\subsection{The function $\col$}

We begin with definitions of the function $\col$, then show that the order of application of $\col$ is unimportant. 



$\col$ collapses the path between two nodes, and replaces it with a new node. All edges into or out of the path are replaced by edges into or out of the new node, and  relabeled according to the function $\relabel$ (Definition~\ref{def:relabel}).
  
\begin{definition}[$\col$]  \label{def:col}
  For nodes $x$, $y$ in a graph $G = (V,E)$, and a new node $v_{N1}$ not in $V$,  $\col(x,y,G)$ is defined provided there is a path between nodes $x$ and $y$. Let $P$ be the set of all nodes on the path from $x$ to $y$. Let $r(l)$ stand for the function $\relabel(l)$. Then $\col(x,y,G) =  (V',E')$, where 
  \begin{eqnarray*}
  V' & = & (V\hide P) \union \{v_{N1}\}     \\
  E' & = & E\hide (
                   \{(v',v,l)|(v',v,l) \in in(v), v\in P\}
                   \union
                   \{(v,v',l)|(v,v',l) \in out(v), v\in P\}
                  )\\
  && \union\ \{(v',v_{N1},r(l))|(v',v,l) \in in(v), v\in P\}\\
  && \union\ \{(v_{N1},v',r(l))|(v,v',l) \in out(v), v\in P\}\\
  \end{eqnarray*}
\end{definition}
If $type(v)$ returns the type of a node $v$, the function $relabel$ (abbreviated above as $r$) is defined as 
\begin{definition}[$relabel$] \label{def:relabel}
  For an edge $(v,v',l)$, 
  \[
   (v,v',relabel(l)) = \left\{
   \begin{array}{l}
      l,    {\emph{if}}\; type(v) \neq type(v') \\
      infl, \emph{otherwise}
   \end{array}   \right.
  \]
\end{definition}
\noindent
$relabel$ is promoted in the obvious way to sets of edges.

\subsection{Proof}

To show that  the order of application of two $\col$ functions does not matter,   we need to show that for any nodes $w,x,y,z$ in a graph $G$, $\col(y,z,\col(w,x,G)) = \col(w,x,\col(y,z,G))$. 


We begin with some abbreviations.

$P_1$ is defined as $\Path(w,x)$.

$P_2$ is defined as $\Path(y,z)$.

$\{v_i\} = \Path(x,w) \inter \Path(y,z)$. Note that in general $\{v_i\}$ is a set of nodes. 



%$E_1$ is the set of nodes after $\col(w,x,G)$ has been carried out.

%$E_2$ is the set of nodes after $\col(y,z,G)$ has been carried out.

% If the new node resulting from $\col(y,z,G)$ is $v_{N1}$,

%$P'_1$ is the path between $w$ and $x$ \textbf{after} $\col(y,z,G)$ has been carried out, resulting in the new node $v_{N1}$. It is defined $P'_1 = (P_1\hide\{v_i\})\union\{v_{N1}\}$.  

%Similarly,  $P'_2$ is the path between $y$ and $z$ \textbf{after} $\col(w,x,G)$ has been carried out, resulting in the new node $v_{N2}$.  It is defined as $P'_2 = (P_2\hide\{v_i\})\union\{v_{N2}\}$.  


{\bf Approach to proof}
\vspace{10pt}

We  calculate the edges and nodes of  $\col(y,z,\col(w,x,G))$,  then $\col(w,x,\col(y,z,G))$, showing that they both result in the same final graph. For readability, we introduce simplifying definitions throughout the proof. 


We begin by calculating the edges:

Following the definitions recorded above, $\col(w,x,G)$ results in graph $G_1=(V_1,E_1)$, where

\begin{eqnarray*}
  E_1 & = & E\hide (
                   \{(v',v,l)|(v',v,l) \in in(v), v\in P_1\}
                   \union
                   \{(v,v',l)|(v,v',l) \in out(v), v\in P_1\}
                  )\\
  && \union\ \{(v',v_{N1},r(l))|(v',v,l) \in in(v), v\in P_1\}\\
  && \union\ \{(v_{N1},v',r(l))|(v,v',l) \in out(v), v\in P_1\}\\
\end{eqnarray*}

\noindent
We now introduce the simplifying definitions:
\[
 in(P_1)   = \{(v',v,l)|(v',v,l) \in in(v), v\in P_1\} \\ 
 out(P_1)  = \{(v,v',l)|(v,v',l) \in out(v), v\in P_1\}\\
 in(v_{N1}) = \{(v',v_{N1},r(l))|(v',v,l) \in in(v), v\in P_1\} \\
 out(v_{N1})= \{(v_{N1},v',r(l))|(v,v',l) \in out(v), v\in P_1\} \\
\]
\noindent
With a slight abuse of notation, $in(\{v_{N1}\})$ (resp. $out(\{v_{N1}\})$)  is written $in(v_{N1})$ (resp. $out(v_{N1})$). These simplifies the definition of $E_1$ to 
\[
  E_1  = ( E\hide(in(P_1) \union out(P_1)) ) \union ( in(v_{N1}) \union out(v_{N1}) )
\]
\noindent  
$P_1$ and $P_2$ intersect (on the nodes $\{v_i\}$), so we record the modification in $P_2$ \textbf{after} $\col(w,x,G)$ has been carried out as $P'_2 = (P_2\hide\{v_i\})\union\{v_{N1}\}$.  If $v_f$ is the replacement abstract node, then


\begin{eqnarray*}
  E_{12} =  & = & E_1\hide (
                   \{(v',v,l)|(v',v,l) \in in(v), v\in P'_2\}
                   \union
                   \{(v,v',l)|(v,v',l) \in out(v), v\in P'_2\}
                  )\\
  && \union\ \{(v',v_f,r(l))|(v',v,l) \in in(v), v\in P'_2\}\\
  && \union\ \{(v_f,v',r(l))|(v,v',l) \in out(v), v\in P'_2\}
\end{eqnarray*}
\noindent
We  introduce some further simplifying abbreviations:
\[
in(P'_2) = \{(v',v,l)|(v',v,l) \in in(v), v\in P'_2\}\\
out(P'_2) = \{(v,v',l)|(v,v',l) \in out(v), v\in P'_2\}\\
in(v_f) = \{(v',v_f,r(l))|(v',v,l) \in in(v), v\in P'_2\}\\
out(v_f) = \{(v_f,v',r(l))|(v,v',l) \in out(v), v\in P'_2\}
\]
\noindent
allowing us to re-write the definition of $E_{12}$ as
\[
  E_{12}  =  (E_1\hide ( in(P'_2) \union out(P'_2)))  \union\ (in(v_f) \union out(v_f))
\]
  

\noindent
and then (by substitution of the $E_1$) as


\[
E_{12}  =  \left( \left(
\begin{array}{l}  E\hide(in(P_1) \union out(P_1)) \\  \union \\ (in(v_{N1}) \union out(v_{N1}) ) \\
\end{array} \right)
   \hide ( in(P'_2) \union out(P'_2)) \right) \\
\hphantom{E_{12}  = \;\; }   \union\ \\
\hphantom{E_{12}  = \;\;}   (in(v_f) \union out(v_f))\\ 

\]

\noindent
and then, by substitution of $P'_2$

\[
E_{12}  =  \left( \left(
  \begin{array}{l}
    E\hide(in(P_1) \union out(P_1)) \\  \union \\ (in(v_{N1}) \union out(v_{N1}) ) \\
  \end{array} \right)
   \hide
   \left( \begin{array}{l}
     in((P_2\hide\{v_i\})\union\{v_{N1}\}) \\\union\\ out((P_2\hide\{v_i\})\union\{v_{N1}\})
   \end{array}
   \right) \right) \\
   \hphantom{E_{12}  = \;\; }   \union\ \\
\hphantom{E_{12}  = \;\;}   (in(v_f) \union out(v_f))\\ 
\]

\noindent
then, since $(P_2\hide\{v_i\}) \inter \{v_{N1}\} = \emptyset$, by distributing $in$ and $out$ over $\union$, we get  

\[
E_{12}  =  \left( \left(
  \begin{array}{l}
    E\hide(in(P_1) \union out(P_1)) \\  \union \\ (in(v_{N1}) \union out(v_{N1}) ) \\
  \end{array} \right)
   \hide
   \left( \begin{array}{l}
     in(P_2\hide\{v_i\})\union in(\{v_{N1}\})) \\\union\\ out(P_2\hide\{v_i\})\union out(\{v_{N1}\})
   \end{array}
   \right) \right) \\
   \hphantom{E_{12}  = \;\; }   \union\ \\
\hphantom{E_{12}  = \;\;}   (in(v_f) \union out(v_f))\\ 
\]

\noindent
We can therefore remove $out(\{v_{N1}\})$ and $out(\{v_{N1}\})$ from both sides of the set hiding operator, leaving 

\[
E_{12}  =  \left( \left(
  \begin{array}{l}
    E\hide(in(P_1) \union out(P_1)) \\  
  \end{array} \right)
   \hide
   \left( \begin{array}{l}
     in(P_2\hide\{v_i\}) \\\union\\ out(P_2\hide\{v_i\})
   \end{array}
   \right) \right) \\
   \hphantom{E_{12}  = \;\; }   \union\ \\
\hphantom{E_{12}  = \;\;}   (in(v_f) \union out(v_f))\\ 
\]
\noindent
and, since $\{v_i\}$ is the set of intersection nodes, and therefore $\{v_i\} \subseteq P_1$, we can write

\[
E_{12}  =  
   E\hide
  (
    in(P_1) \union out(P_1) \union in(P_2) \union out(P_2)
  ) \\
   \hphantom{E_{12}  = \;\;}   \union\ \\
\hphantom{E_{12}  = \;\;}   (in(v_f) \union out(v_f))\\ 
\]


We now need to show that $v_f=v_g$ under $\alpha$-equivalence. We do this by showing that $in(v_f)=in(v_g)$. The demonstration that $out(v_f)=out(v_g)$ proceeds in a similar fashion. 

Below, we expand the definition of $in(v_f)$. 

\[
in(v_f) = \{(v',v_f,r(l))|(v',v,l) \in in(v), v\in P'_2\}
\]
\noindent
by definition of $P'_2$,
\[
in(v_f) = \{(v',v_f,r(l))|(v',v,l) \in in(v), v\in((P_2\hide\{v_i\})\union\{v_{N1}\})\}
\]
\noindent
and,  simplification, 
\[
in(v_f) = \{(v',v_f,r(l))|(v',v,l) \in in((P_2\hide\{v_i\}) \union \{v_{N1}\})\}
\]
\noindent
and, by set manipulation, 
\[
in(v_f) = \{(v',v_f,r(l))|(v',v,l) \in in(P_2\hide\{v_i\}) \union  \{(v',v_{N1},r(l))|(v',v,l)\in in(v), v\in P_1\}\}
\]
\noindent
which, by definition of $in(P_1)$, we can write as
\[
in(v_f) = \{(v',v_f,r(l))|(v',v,l) \in in(P_2\hide\{v_i\}) \union  r(in(P_1))\}
\]
\noindent
where $r(in(P))$ is the relabeling operator $r$ promoted to sets.  Which, since $r$ is idempotent ($r(r(l)) = r(l)$), we can write as
\[
in(v_f) = \{(v',v_f,r(l))|(v',v,l) \in in(P_2\hide\{v_i\}) \union  in(P_1)\}
\]
\noindent
and since $\{v_i\} \subseteq P_1$, 
\[
in(v_f) = \{(v',v_f,r(l))|(v',v,l) \in in(P_2) \union  in(P_1)\}
\]

\pagebreak

We now carry out similar manipulations for $\col(w,x,col(y,z,G))$, thus showing that the edges of $\col(y,z,col(w,x,G))$ are equal to the edges of  $\col(w,x,col(y,z,G))$ (up to $\alpha$-equivalence). 

\section{part 2}

Following the definitions recorded above, $\col(y,z,G)$ results in graph $G_2=(V_2,E_2)$, where the node introduced is called $v_{N2}$. 

\begin{eqnarray*}
  E_2 & = & E\hide (
                   \{(v',v,l)|(v',v,l) \in in(v), v\in P_2\}
                   \union
                   \{(v,v',l)|(v,v',l) \in out(v), v\in P_2\}
                  )\\
  && \union\ \{(v',v_{N2},r(l))|(v',v,l) \in in(v), v\in P_2\}\\
  && \union\ \{(v_{N},v',r(l))|(v,v',l) \in out(v), v\in P_2\}
\end{eqnarray*}
We now introduce the simplifying definitions:
\begin{eqnarray*}
 in(P_2)   &=& \{(v',v,l)|(v',v,l) \in in(v), v\in P_2\} \\ 
 out(P_2)  &=& \{(v,v',l)|(v,v',l) \in out(v), v\in P_2\}\\
 in(v_{N2}) &=& \{(v',v_{N2},r(l))|(v',v,l) \in in(v), v\in P_2\} \\
 out(v_{N2})&=& \{(v_{N2},v',r(l))|(v,v',l) \in out(v), v\in P_2\} 
\end{eqnarray*}

With a slight abuse of notation, $in(\{v_{N2}\})$ (resp\ $out(\{v_{N2}\})$)  is written $in(v_{N2})$ (resp $out(v_{N2})$). Using these re-writes simplifies the definition of $E_2$ to 
\begin{eqnarray*}
  E_2  &=& ( E\hide(in(P_2) \union out(P_2)) ) \\
       &&\union\\
       && ( in(v_{N2}) \union out(v_{N2}) )
\end{eqnarray*}
       
$P_2$ and $P_1$ intersect (on the set of nodes $\{v_i\}$), so after the path $P_2$ has been removed  we record the modification in $P_1$ as $P'_1$, where $P'_1 = (P_1\hide\{v_i\})\union\{v_{N2}\}$.

We now carry out $col(w,x,G')$. 

We now replace the path $P'_1$ with the replacement abstract node $v_g$, and 
\begin{eqnarray*}
  E_{21} =  & = & E_2\hide (
                   \{(v',v,l)|(v',v,l) \in in(v), v\in P'_1\}
                   \union
                   \{(v,v',l)|(v,v',l) \in out(v), v\in P'_1\}
                  )\\
  && \union\ \{(v',v_g,r(l))|(v',v,l) \in in(v), v\in P'_1\}\\
  && \union\ \{(v_g,v',r(l))|(v,v',l) \in out(v), v\in P'_1\}
\end{eqnarray*}
We  introduce some further simplifying abbreviations:
\[
in(P'_1) = \{(v',v,l)|(v',v,l) \in in(v), v\in P'_1\}\\
out(P'_1) = \{(v,v',l)|(v,v',l) \in out(v), v\in P'_1\}\\
in(v_g) = \{(v',v_g,r(l))|(v',v,l) \in in(v), v\in P'_1\}\\
out(v_g) = \{(v_g,v',r(l))|(v,v',l) \in out(v), v\in P'_1\}\\
\]
\noindent
allowing us to re-write the definition of $E_{21}$ as
\[
  E_{21}  =  (E_2\hide ( in(P'_1) \union out(P'_1)))  \union\ (in(v_g) \union out(v_g))
\]
  

\noindent
and then (by substitution of the $E_2$) as


\[
E_{21}  =  \left( \left(
\begin{array}{l}  E\hide(in(P_2) \union out(P_2)) \\  \union \\ (in(v_{N2}) \union out(v_{N2}) ) \\
\end{array} \right)
   \hide ( in(P'_1) \union out(P'_1)) \right) \\
\hphantom{E_{12}  = \;\; }   \union\ \\
\hphantom{E_{12}  = \;\;}   (in(v_g) \union out(v_g))\\ 

\]

\noindent
and then, by substitution of $P'_2$

\[
E_{21}  =  \left( \left(
  \begin{array}{l}
    E\hide(in(P_2) \union out(P_2)) \\  \union \\ (in(v_{N2}) \union out(v_{N2}) ) \\
  \end{array} \right)
   \hide
   \left( \begin{array}{l}
     in((P_1\hide\{v_i\})\union\{v_{N2}\}) \\\union\\ out((P_1\hide\{v_i\})\union\{v_{N2}\})
   \end{array}
   \right) \right) \\
   \hphantom{E_{12}  = \;\; }   \union\ \\
\hphantom{E_{12}  = \;\;}   (in(v_g) \union out(v_g))\\ 
\]

\noindent
then, since $(P_1\hide\{v_i\}) \inter \{v_{N2}\} = \emptyset$, by distributing $in$ and $out$ over $\union$, we get  

\[
E_{21}  =  \left( \left(
  \begin{array}{l}
    E\hide(in(P_2) \union out(P_2)) \\  \union \\ (in(v_{N2}) \union out(v_{N2}) ) \\
  \end{array} \right)
   \hide
   \left( \begin{array}{l}
     in(P_1\hide\{v_i\})\union in(\{v_{N2}\})) \\\union\\ out(P_1\hide\{v_i\})\union out(\{v_{N2}\})
   \end{array}
   \right) \right) \\
   \hphantom{E_{12}  = \;\; }   \union\ \\
\hphantom{E_{12}  = \;\;}   (in(v_g) \union out(v_g))\\ 
\]

\noindent
We can therefore remove $out(\{v_{N2}\})$ and $out(\{v_{N2}\})$ from both sides of the set hiding operator, leaving 

\[
E_{21}  =  \left( \left(
  \begin{array}{l}
    E\hide(in(P_2) \union out(P_2)) \\  
  \end{array} \right)
   \hide
   \left( \begin{array}{l}
     in(P_1\hide\{v_i\}) \\\union\\ out(P_1\hide\{v_i\})
   \end{array}
   \right) \right) \\
   \hphantom{E_{12}  = \;\; }   \union\ \\
\hphantom{E_{12}  = \;\;}   (in(v_g) \union out(v_g))\\ 
\]
\noindent
and, since $\{v_i\}$ is the set of intersection nodes, and therefore $\{v_i\} \subseteq P_2$, we can write

\[
E_{21}  =  
   E\hide
  (
    in(P_2) \union out(P_2) \union in(P_1) \union out(P_1)
  ) \\
   \hphantom{E_{12}  = \;\;}   \union\ \\
\hphantom{E_{12}  = \;\;}   (in(v_g) \union out(v_g))\\ 
\]



Now, we expand the definition of $in(v_g)$. $out(v_g)$ an exercise!

\[
in(v_g) = \{(v',v_g,r(l))|(v',v,l) \in in(v), v\in P'_1\}
\]
\noindent
by definition of $P'_1$,
\[
in(v_g) = \{(v',v_g,r(l))|(v',v,l) \in in(v), v\in((P_1\hide\{v_i\})\union\{v_{N2}\})\}
\]
\noindent
and,  by simplification, 
\[
in(v_g) = \{(v',v_g,r(l))|(v',v,l) \in in((P_1\hide\{v_i\}) \union \{v_{N2}\})\}
\]
\noindent
and, by set manipulation, 
\[
in(v_g) = \{(v',v_g,r(l))|(v',v,l) \in in(P_1\hide\{v_i\}) \union  \{(v',v_{N2},r(l))|(v',v,l)\in in(v), v\in P_2\}\}
\]
\noindent
which, by definition of $in(P_2)$, we can write as
\[
in(v_g) = \{(v',v_g,r(l))|(v',v,l) \in in(P_1\hide\{v_i\}) \union  r(in(P_2))\}
\]
\noindent
where $r(in(P))$ is the relabeling operator $r$ promoted to sets.  Since $r$ is idempotent, we can write this as
\[
in(v_g) = \{(v',v_g,r(l))|(v',v,l) \in in(P_1\hide\{v_i\}) \union  in(P_2)\}
\]
\noindent
and since $\{v_i\} \subseteq P_2$, 
\[
in(v_g) = \{(v',v_g,r(l))|(v',v,l) \in in(P_1) \union  in(P_2)\}
\]


It is thus clear that $in(v_f)=in(v_g)$. 


\subsection{equality of nodes}

 We now need to show that for any nodes $w,x,y,z$ in a graph $G$, the nodes in the graph resulting from $\col(y,z,\col(w,x,G))$ are the same as the nodes resulting from $\col(w,x,\col(y,z,G))$. 

Recall from Definition~\ref{def:col} that if $x$ and $y$ are nodes, and $P$ is the set of all nodes on the path from $x$ to $y$, then  $\col(x,y,G) =  (V',E')$, where 
  \begin{eqnarray*}
  V' & = & (V\hide P) \union \{v_{N1}\}     \\
  \end{eqnarray*}


Let $P_1$ be defined as $\Path(w,x)$ and $P_2$  as $\Path(y,z)$ and $\{v_i\} = \Path(x,w) \inter \Path(y,z)$.

Consider first $\col(y,z,\col(w,x,G))$, and let $v_{N1}$ be the intermediate node, and $v_f$ be the final node. 

Then by Definition~\ref{def:col}, $V_1 = (V\hide P_1) \union \{v_{N1}\}$. $P_2$ is modified, and $P'_2= P_2\hide \{v_i\} \union \{v_{N1}\}$ and the final set of nodes $V''$ is given by $V_{12} = V_1\hide P'_2 \union \{v_f\}$. 

So
\[
V_{12} \\
=\\
V_1\hide P'_2 \union \{v_f\} \\
=\\
V_1\hide ((P_2\hide \{v_i\}) \union \{v_{N1}\}) \union \{v_f\}\\
=\\
((V\hide P_1) \union \{v_{N1}\})\hide ((P_2\hide \{v_i\}) \union \{v_{N1}\}) \union \{v_f\}\\
=(\textrm{removing}\; v_{N1}\; \textrm{from both sides of the hiding operator})\\
((V\hide P_1))\hide ((P_2\hide \{v_i\}) ) \union \{v_f\}\\ 
= (\textrm{since}\; \{v_i\} \subseteq P_1 \inter P_2)\\
((V\hide P_1))\hide ((P_2) ) \union \{v_f\}\\ 
=\\
(V\hide (P_1 \union P_2)) \union \{v_f\}\\ 
\]

A similar argument holds for $V_{21}$, showing $V_{21} = (V\hide (P_2 \union P_1)) \union \{v_g\}$ and so $V_{21}=V_{12}$ (up to $\alpha$-equivalence).

Thus $\col(y,z,\col(w,x,G)) = \col(w,x,\col(y,z,G))$, as required. 

