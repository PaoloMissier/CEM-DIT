\usepackage{amssymb,amsmath}

\usepackage{graphicx}
\usepackage{zed-csp}
%\usepackage{csp}
%\usepackage{algorithm}
%\usepackage{algpseudocode}
\usepackage{float}
\usepackage{url}
\usepackage{cite}
\usepackage{tabularx}
%\usepackage{cases}

% The line below was added as a workaround for spacing ``f'' in mathit. \mathit can now be replaced by \Lmathit to get better spacing. 
\DeclareMathAlphabet{\Lmathit}{\encodingdefault}{\familydefault}{m}{it}

\interdisplaylinepenalty=2500

\newfloat{algorithm}{tbp}{lop}
\floatstyle{ruled}

\newtheorem{lemma}{Lemma}%
\newtheorem{definition}{Definition}%
\newtheorem{conjecture}{Conjecture}%
\newtheorem{theorem}{Theorem}%
\newtheorem{proposition}[theorem]{Proposition}

% include if zed-csp is not included.
% \newcommand{\hide}{\setminus}
% \newcommand{\spot}{\bullet}

%include if zed-csp IS included.
\renewcommand{\inv}{\mathit{inval}}
%otherwise
%\newcommand{\inv}{\mathit{inval}}

\newcommand{\prov}{\mathit{prov}}
\newcommand{\dep}{\mathit{dep}}
\newcommand{\pr}{\mathit{pr}}
\newcommand{\obf}{\mathit{abs}}
\newcommand{\pol}{\mathit{pol}}

\newcommand{\pg}{\mathit{PG}}
\newcommand{\en}{\mathit{En}}
\newcommand{\act}{\mathit{Act}}
\newcommand{\ag}{\mathit{Ag}}

\newcommand{\used}{\Lmathit{used}}
\newcommand{\wgby}{\Lmathit{genBy}}
\newcommand{\influence}{\Lmathit{wasInfluencedBy}}
\newcommand{\wdf}{\Lmathit{wasDerivedFrom}}
\newcommand{\waw}{\Lmathit{waw}}
\newcommand{\attrTo}{\Lmathit{wat}}
\newcommand{\wat}{\Lmathit{wat}}
\newcommand{\delegate}{\Lmathit{abo}}
\newcommand{\wasInfBy}{\Lmathit{wasInformedBy}}
\newcommand{\start}{\Lmathit{start}}
\newcommand{\ed}{\Lmathit{end}}


\newcommand{\Ev}{\mathit{Ev}}
\newcommand{\preorder}{\preceq}

\newcommand{\evmap}{\mathit{evmap}}


\newcommand{\node}{\mathit{Node}}
\newcommand{\type}{\mathit{type}}
\newcommand{\elabel}{\mathit{label}}

\newcommand{\guEA}{\pg_{gu/ea}}  % gen-usage over enties, activities
\newcommand{\guiEA}{\pg_{gui/ea}}
\newcommand{\guaEAG}{\pg_{gu+/eaAg}}  % gen-usage and more over enties, activities plus Agents

%% operators
\newcommand{\group}{\mathit{Group}}
\newcommand{\aggroup}{\mathit{agGroup}}
\newcommand{\sgroup}{\mathit{Group_{str}}}
\newcommand{\clos}{\mathit{pclos}}
\newcommand{\repl}{\mathit{replace}}
\newcommand{\agrepl}{\mathit{agreplace}}
\newcommand{\rewire}{\mathit{rewire}}
\newcommand{\extend}{\mathit{extend}}
\newcommand{\pclos}{\mathit{pclos}}
\newcommand{\rem}{\mathit{remove}}
\newcommand{\agremove}{\mathit{AgRemove}}
\newcommand{\orphanremove}{\mathit{OrphanRemove}}
\newcommand{\allorphanremove}{\mathit{AllOrphanRemove}}
\newcommand{\remIsolated}{\mathit{remIsolated}}

\newcommand{\dclos}{\mathit{dclos}}

\newcommand{\POL}{\mathit{\cal P}}

%\newcommand{\mnote}[1] {\marginpar{\scriptsize \raggedright #1 }}
\newcommand{\mnote}[1] {  \framebox{\begin{minipage}[t]{0.9\linewidth}
 \scriptsize \raggedright #1 \normalsize
    \end{minipage}
 }}

%%%%%%%%%%%%%%%%%%%%%%%%%%%%%%%%%%%%
%Additional commands for BlueCollapse

\newcommand{\col}{\mathit{col}}
\newcommand{\relabel}{\mathit{relabel}}
\newcommand{\Path}{\mathit{path}}

%%%%%%%%%%%%%%%%%%%%%%%%%%%%%%%%%%%%%


%%% Optional notes to highlight the changes in the reviewed document
% \optchange to indicated changed text.
% \optadded to indicate newly text.
% \optremoved to indicate text removed.
% \optnote to write a note which is shown only if changes are on
\newif\ifwithchanges

% Turn on/off change highlighting by un/commenting this toggle
\withchangestrue

\newcommand{\optchange}[1]{%
	\ifwithchanges
	{\leavevmode\color{red}#1}%
	\else
	#1%
	\fi}

\newcommand{\optadded}[1]{%
	\ifwithchanges
	{\leavevmode\color{darkgreen}#1}%
	\else
	#1%
	\fi}

\newcommand{\optremoved}[1]{%
	\ifwithchanges
	{\color{red}\sout{#1}}%
	\fi}

\newcommand{\optnote}[1]{%
	\ifwithchanges
	{\footnotesize\itshape\color{red}\{#1\}}%
	\fi}
%
%%% End of optional note definition


\newenvironment{mydrop}{\begin{array}[t]{@{}l@{}}}{\end{array}}%



\usepackage{color}
\usepackage{pdfpages}
\usepackage{ifthen}
\newcommand{\showComments}{yes} % {yes}
\newcommand{\note}[2]{
    \ifthenelse{\equal{\showComments}{yes}}{\textcolor{#1}{#2}}{}
}
\newcommand{\jwb}[1]{\note{blue}{JWB: #1}}
