\section{Tool Implementation}   \label{sec:summary}

Rather than giving a procedural pseudo-code for the $\group$ algorithm, we summarize its functional specification, comprising of the four functions defined so far: $\clos()$, $\extend()$, $\repl()$, and $\remIsolated()$, in Table~\ref{alg-summary}.

\begin{table*}
  \begin{eqnarray*}
    %%%%%%
    \clos(V_{gr}, V)  &%  
    = &%
     \bigcup_{v_i, v_j \in V_{gr}} V_{ij} \\
    %%%%%%
    \extend(V_{gr}, G ,t) &%
    = & V_{gr} 
      \begin{mydrop}
        ~\cup~ \{ v' | (v, v') \in E \wedge v \in V_{gr} \wedge \type(v') = t) \}  \\
        ~\cup~ \{ v | (v', v) \in E \wedge v \in V_{gr} \wedge \type(v') = t) \}  
      \end{mydrop} \\%
    %%%%%%
    \repl(V_{gr}, v_{new}, G) &% 
    = & %
    (V', E'), \mathrm{where}~~ 
       \begin{mydrop}
         \vartheta_{out}'(V_{gr}') = \{ v \xleftarrow{t}  v_{new} |  \begin{mydrop}
                 v \xleftarrow{t} v' \in \vartheta_{out}(V_{gr}')  \wedge \\%
                 ( (t = \wat \wedge \type(v_{new}) = \en) \vee \\%
                (t = \waw \wedge \type(v_{new}) = \act)  \vee \\%
                (t \neq \wat \wedge t \neq \waw) )\}  
              \end{mydrop} \\
         \vartheta_{in}'(V_{gr}') = \{  v_{new} \xleftarrow{t}  v | \begin{mydrop}
           v' \xleftarrow{t} v \in \vartheta_{in}(V_{gr}')  \wedge  \\%
           ( \begin{mydrop}
             (t = \delegate \wedge \type(v_{new}) = \ag) \\%
             \vee t \neq \delegate )\}
             \end{mydrop}
           \end{mydrop}
       \end{mydrop}\\
    %%%%%%
   V' & %
   = & %
   V  \setminus V_{gr}  \cup \{v_{new}\}  \\
    %%%%%%
   E' &% 
   = & % 
   E \setminus (\vartheta_{out}(V_{gr}) \cup \vartheta_{in}(V_{gr}) \cup \vartheta_{int}(V_{gr}))  \cup \vartheta_{out}'(V_{gr})  \cup \vartheta_{in}'(V_{gr}) \\
    %%%%%%
    \remIsolated(V,E) \mathrm{where} & %
    = &%
    (V', E') \\
    %%%%%%
    \mathit{isolated}(G) & 
    = & %
    \{  v \in V | \type(v) = \ag  \wedge \nexists v' \in V . ((v',v) \in E \vee (v,v') \in E) \} \\
    V' & %
    = & %
    V \setminus \mathit{isolated}(G)\\
    %%%%%%
    E' &
    =&
    E \\
    %%%%%%
    \group(G, V_{gr}, v_{new}, t) &%
    = & %
    \remIsolated(\repl( \extend(\clos(V_{gr},V), V, t), v_{new},  G ))
    %%%%%%
  \end{eqnarray*}
  \caption{Functional summary of the grouping algorithm.}
  \label{alg-summary}
\end{table*}



% \begin{table*}
% \begin{tabularx}{\textwidth}{lXlX}
% %%%%%%
% $\clos(V_{gr}, V) $ &  
% %%%%%%
% $ \bigcup_{v_i, v_j \in V_{gr}} V_{ij}$ \\
% %%%%%%
% $\extend(V_{gr}, G ,t) = V_{gr}:$ &  
% %%%%%%
% $\cup \{ v' | (v, v') \in E \wedge v \in V_{gr} \wedge \type(v') = t) \}  
% \cup \{ v | (v', v) \in E \wedge v \in V_{gr} \wedge \type(v') = t) \} $ \\
% %%%%%%
% $\repl(V_{gr}, v_{new}, G) = (V', E'):$ &  
% $\vartheta_{out}'(V_{gr}') = \{ v \xleftarrow{t}  v_{new} |  v \xleftarrow{t} v' \in \vartheta_{out}(V_{gr}')  \wedge ( (t = \wat \wedge \type(v_{new}) = \en) \vee 
% (t = \waw \wedge \type(v_{new}) = \act)  \vee (t \neq \wat \wedge t \neq \waw) )\} $ \newline
% $\vartheta_{in}'(V_{gr}') = \{  v_{new} \xleftarrow{t}  v |  v' \xleftarrow{t} v \in \vartheta_{in}(V_{gr}')  \wedge ( (t = \delegate \wedge \type(v_{new}) = \ag) \vee t \neq \delegate )\} $ \newline
% $V' = V  \setminus V_{gr}  \cup \{v_{new}\} $ \newline
% $ E' = E \setminus (\vartheta_{out}(V_{gr}) \cup \vartheta_{in}(V_{gr}) \cup \vartheta_{int}(V_{gr}))  \cup \vartheta_{out}'(V_{gr})  \cup \vartheta_{in}'(V_{gr}) $\\
% %%%%%%
% %%%%%%
% $\remIsolated(V,E) = (V', E')$ & 
% $\mathit{isolated}(G) = \{  v \in V | \type(v) = \ag  \wedge \nexists v' \in V . ((v',v) \in E \vee (v,v') \in E) \}$ \newline
% $V ' = V \setminus \mathit{isolated}(G), E' =  E$ \\
% %%%%%%
% $\group(G, V_{gr}, v_{new}, t):$&
% $\remIsolated(\repl( \extend(\clos(V_{gr},V), V, t), v_{new},  G ))$
% \end{tabularx} 
% \caption{Functional summary of the grouping algorithm.}
% \label{alg-summary}
% \end{table*}
% 
 
A procedural implementation of the algorithm, written in Java, is available online (\url{https://github.com/PaoloMissier/ProvAbs}) as part of a tool in which a policy language is used to drive the selection of the set $V_{gr}$ of nodes to be abstracted. Using the tool, policy setters may experiment with abstraction policies on their provenance graphs, observing their effects when abstraction is performed.
%
Noting that $\group()$ is closed wrt composition, the tool also allows for more complex abstraction, by letting users specify compositions of $\group$ operations as part of their policy.

%
Here we only provide a brief description of the tool. A more detailed account can be found in~\citep{Missier2013c} (submitted).
%
The tool operates on provenance graphs written in the provenance notation PROV-N~\citep{w3c-prov-n}.

%
Given $G \in \guaEAG$, the tool lets users specify a grouping set $V_{gr}$ by means of an \textit{abstraction policy}. It operates in two steps. Firstly, path expressions and predicates are used to select a set of nodes, and to assign a numerical \textit{sensitivity} value to them. 
%
For example, the following rule contains a path expression that binds variables \texttt{process} and \texttt{data} to activity and entity nodes $a$, $e$, respectively, such that $\used(a,e)$ holds and $e$ is any node that is reachable from node with id \texttt{d14}:

\vspace{0.5\baselineskip}

\small
\textsf{for all (process used data)}

\textsf{ \;\;\; where (data descendantOf d14)) }

\textsf{ \;\;\;\;\;\; setSensitivity(data, 10)}

\vspace{0.5\baselineskip}

\normalsize

The sensitivity value of the selected data nodes is set to 10.
%
Here \textit{descendantOf} is a built-in query predicate that returns all nodes reachable from a given start node. 

%
As an another example, the  predicate

\vspace{0.5\baselineskip}

\small
\textsf{for all (data wgb process)} 

 \textsf{ \;\;\; where (process.classification $>$ ``conf'' in classifications) }

\textsf{ \;\;\;\;\;\; setSensitivity(data, 9)}

\vspace{0.5\baselineskip}

\normalsize


binds variables \texttt{data} and \texttt{process} to pairs of nodes $d$, $p$ such that $\wgby(d,p)$ and where the classification value $p.classification$ of $p$ (a property of the node) is greater than ``conf''. This requires an ordered set of classification labels to be declared in a user-defined \texttt{classifications} list.
%
During the first step, the tool evaluates the policy rules, resulting in the annotation of the selected nodes with sensitivity values. 

%
This model operates on the same principle as the Bell-LaPadula security model. Each known receiver of a graph is pre-assigned a clearance level. This is determined by factors outside the scope of the tool, e.g. how much a receiver is trusted to the provenance owner. 
%
In the second step, node sensitivities are compared to the clearance level $cl$ of the receiver, and $V_{gr}$ is defined as the set of nodes whose sensitivity is lower than $cl$.
%

The rationale for these two steps is that sensitivity can be defined largely independently of the specific receiver, while the exact level of abstraction is relative to a receiver, who in this case is represented simply by a clearance level.

%Such predicates 
%For example, if certain activities had a property named \emph{sensitivity} which contained numeric values, then ``\emph{all entities produced by processes that sensitivity} $\ge$ \emph{5}'' is a valid clause within the policy language.  
%
%
%
%The evaluation of a policy over $G$ results in the identification of the set of nodes $V_{gr}$. The user can then  apply $\group(G, V_{gr}, v_{new}, t)$ to a graph $G \in \guaEAG$, where $t$ is again selected by the user. 
%
