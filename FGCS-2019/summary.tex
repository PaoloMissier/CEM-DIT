
\section{Summary and further research}
\label{sec:further}

We have proposed a model for the principled \jwbtwo{hiding} of provenance based on a formal definition of abstraction, in which elements of a provenance graph are grouped together and replaced by a single abstract node.  
%The presentation is incremental: first we develop the model for the case when the nodes of provenance graphs are restricted to entities and activities, and then consider the case where nodes may also be agents.  
A guiding principle  throughout is that we avoid the introduction of \emph{\jwbtwo{unjustified} dependencies}: our abstraction will reduce the information content of a provenance graph, but it will not introduce information \jwbtwo{that is not justified by the combination of the abstraction and the information in the original graph}.  
The abstraction acts on and results in provenance graphs which are PROV compliant.   A separate paper~\citep{Missier2014} presents the tool implementing this model in detail.

%\comment{summarise contributions}

\jwbtwo{Our grouping operators ensure the validity of the resulting abstracted graph: if $\pg$ is a PROV graph, that is, it conforms to the PROV data model, meeting the constraints  outlined in ~\citep{w3c-prov-dm}.} 
%$\pg'$ is also a PROV graph, and constraint preservation: if $\pg$ satisfies all PROV constraints~\citep{w3c-prov-dm}, then $\pg'$  also satisfies these. 
\jwbtwo{Also, no \jwbtwo{unjustified} dependencies are introduced into $\pg'$: a  relationship involving $v_{new}$ is only created as a result of a mapping from an existing relationship involving elements of $V_{gr}$.}~ \jwbtwo{Strictly, if $a'$ and $e'$ are  \textit{directly} related in $\pg'$, we guarantee that for each of $a'$ and $e'$, either they or an element in their abstracted set are directly related in $\pg$. }


The work described in this paper is progressing in \jwbtwo{two} main directions.
%
First, we are aware that the fragment of PROV to which $\group$ applies does not cover all relation types. Nevertheless, the method described in the paper for reasoning about PROV graph transformation can be used as a guideline to extend the work to the missing parts of PROV.
\jwbtwo{A first extension would consider agents, possibly beginning with an initial simplifying assumption that agents are disjoint from entities and activities.}

\jwbtwo{Secondly, although in this work we have predominantly chosen not to allow the grouping operators to change types of edges, in Section~\ref{sec:influence} we have explored the use of the \emph{wasInfluencedBy} relation, which is a supertype of both $\wgby$ and $\used$. 
%
Allowing the operator to change the node types has the advantage of retaining more information from the original PROV graph, as we point out, and we should explore more completely the implications of allowing these types of changes more generally.  
}

%\jwbtwo{Finally, our initial work on the} implications of abstraction on the space of events that are used to characterize the semantics of PROV \jwbtwo{in Section~\ref{sec:events} should be extended to include invalidation.}


%As constraints over the relative ordering of events are defined in detail in the \jwbtwo{PROV-CONSTRAINTS} document, there is an obligation to extend the notion of validity of PROV graphs to include those constraints. Thus, grouping must be shown to be validity-preserving relative to those constraints as well. 

%\comment{we've done this according to sec 5.}


%\comment{Address: The authors have not considered what domains enforce a validity constraint and what if it is relaxed to show an partially inconsistent graph?}

