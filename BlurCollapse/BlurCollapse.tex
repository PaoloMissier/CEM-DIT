%%-*- mode: LaTeX; mode: FlySpell; -*-

\documentclass{elsarticle}

\usepackage{amssymb,amsmath}

\usepackage{graphicx}
\usepackage{zed-csp}
%\usepackage{csp}
%\usepackage{algorithm}
%\usepackage{algpseudocode}
\usepackage{float}
\usepackage{url}
\usepackage{cite}
\usepackage{tabularx}
%\usepackage{cases}

% The line below was added as a workaround for spacing ``f'' in mathit. \mathit can now be replaced by \Lmathit to get better spacing. 
\DeclareMathAlphabet{\Lmathit}{\encodingdefault}{\familydefault}{m}{it}

\interdisplaylinepenalty=2500

\newfloat{algorithm}{tbp}{lop}
\floatstyle{ruled}

\newtheorem{lemma}{Lemma}%
\newtheorem{definition}{Definition}%
\newtheorem{conjecture}{Conjecture}%
\newtheorem{theorem}{Theorem}%
\newtheorem{proposition}[theorem]{Proposition}

% include if zed-csp is not included.
% \newcommand{\hide}{\setminus}
% \newcommand{\spot}{\bullet}

%include if zed-csp IS included.
\renewcommand{\inv}{\mathit{inval}}
%otherwise
%\newcommand{\inv}{\mathit{inval}}

\newcommand{\prov}{\mathit{prov}}
\newcommand{\dep}{\mathit{dep}}
\newcommand{\pr}{\mathit{pr}}
\newcommand{\obf}{\mathit{abs}}
\newcommand{\pol}{\mathit{pol}}

\newcommand{\pg}{\mathit{PG}}
\newcommand{\en}{\mathit{En}}
\newcommand{\act}{\mathit{Act}}
\newcommand{\ag}{\mathit{Ag}}

\newcommand{\used}{\Lmathit{used}}
\newcommand{\wgby}{\Lmathit{genBy}}
\newcommand{\influence}{\Lmathit{wasInfluencedBy}}
\newcommand{\wdf}{\Lmathit{wasDerivedFrom}}
\newcommand{\waw}{\Lmathit{waw}}
\newcommand{\attrTo}{\Lmathit{wat}}
\newcommand{\wat}{\Lmathit{wat}}
\newcommand{\delegate}{\Lmathit{abo}}
\newcommand{\wasInfBy}{\Lmathit{wasInformedBy}}
\newcommand{\start}{\Lmathit{start}}
\newcommand{\ed}{\Lmathit{end}}


\newcommand{\Ev}{\mathit{Ev}}
\newcommand{\preorder}{\preceq}

\newcommand{\evmap}{\mathit{evmap}}


\newcommand{\node}{\mathit{Node}}
\newcommand{\type}{\mathit{type}}
\newcommand{\elabel}{\mathit{label}}

\newcommand{\guEA}{\pg_{gu/ea}}  % gen-usage over enties, activities
\newcommand{\guiEA}{\pg_{gui/ea}}
\newcommand{\guaEAG}{\pg_{gu+/eaAg}}  % gen-usage and more over enties, activities plus Agents

%% operators
\newcommand{\group}{\mathit{Group}}
\newcommand{\aggroup}{\mathit{agGroup}}
\newcommand{\sgroup}{\mathit{Group_{str}}}
\newcommand{\clos}{\mathit{pclos}}
\newcommand{\repl}{\mathit{replace}}
\newcommand{\agrepl}{\mathit{agreplace}}
\newcommand{\rewire}{\mathit{rewire}}
\newcommand{\extend}{\mathit{extend}}
\newcommand{\pclos}{\mathit{pclos}}
\newcommand{\rem}{\mathit{remove}}
\newcommand{\agremove}{\mathit{AgRemove}}
\newcommand{\orphanremove}{\mathit{OrphanRemove}}
\newcommand{\allorphanremove}{\mathit{AllOrphanRemove}}
\newcommand{\remIsolated}{\mathit{remIsolated}}

\newcommand{\dclos}{\mathit{dclos}}

\newcommand{\POL}{\mathit{\cal P}}

%\newcommand{\mnote}[1] {\marginpar{\scriptsize \raggedright #1 }}
\newcommand{\mnote}[1] {  \framebox{\begin{minipage}[t]{0.9\linewidth}
 \scriptsize \raggedright #1 \normalsize
    \end{minipage}
 }}

\newenvironment{mydrop}{\begin{array}[t]{@{}l@{}}}{\end{array}}%



\usepackage{color}
\usepackage{pdfpages}
\usepackage{ifthen}
\newcommand{\showColour}{yes} % {yes}
\newcommand{\showComments}{yes} % {yes}
\newcommand{\note}[2]{\ifthenelse{\equal{\showColour}{yes}}{\textcolor{#1}{#2}}{#2}}
\newcommand{\jwb}[1]{\note{blue}{#1}}

\newcommand{\paolo}[1]{\note{magenta}{#1}}


\newcommand{\com}[2]{\ifthenelse{\equal{\showComments}{yes}}{\textcolor{#1}{#2}}{}}
\newcommand{\comment}[1]{\com{red}{#1}}



\begin{document}

\title{Blur And Collapse}


\author[ncl]{P. ~Missier \corref{cor1}}
\ead{Paolo.Missier@newcastle.ac.uk}

\author[cov]{J. ~Bryans\corref{cor2}}
\ead{Jeremy.Bryans@coventry.ac.uk}


\address[ncl]{School of Computing Science, Newcastle University, UK}
\address[cov]{Institute for Future Transport and Cities, Coventry University, UK}

\cortext[cor1]{Principal Corresponding Author}
\cortext[cor2]{Corresponding Author}

%\markboth{IEEE Transaction on Knowledge and Data Engineering,~Vol.~X, No.~Y, DATE}%
%{Shell \MakeLowercase{\textit{et al.}}: happy}


\begin{abstract}

\end{abstract}

\begin{keyword}
Provenance \sep Provenance metadata \sep provenance abstraction 
\end{keyword}

\maketitle


\section{Introduction and motivation}


%\subsection{The function $\col$}
\mnote{still to do:
\begin{itemize}
  \item calc edges of $E_{21}$ and show equal to $E_{12}$.
  \item whole proof of equality of nodes. 
  \item so $(G_{12} = G_{21})$. 
\end{itemize}
}




$\col$ collapses the path between two nodes, and replaces it with a new node. All edges into or out of the path are relabeled according to the function $\relabel$.
  
\begin{definition}[$\col$]  \label{def:col}
  For nodes $x$, $y$ in a graph $G = (V,E)$, and a new node $v_{N}$ not in $V$,  $\col(x,y,G)$ is defined provided there is a path between nodes $x$ and $y$. Let $P$ be the set of all nodes on the path from $x$ to $y$. Let $r(l)$ stand for the function $\relabel(l)$. Then $\col(x,y,G) =  (V',E')$, where 
  \begin{eqnarray*}
  V' & = & (V\hide P) \union \{v_N\}     \\
  E' & = & E\hide (
                   \{(v',v,l)|(v',v,l) \in in(v), v\in P\}
                   \union
                   \{(v,v',l)|(v,v',l) \in out(v), v\in P\}
                  )\\
  && \union\ \{(v',v_{N},r(l))|(v',v,l) \in in(v), v\in P\}\\
  && \union\ \{(v_{N},v',r(l))|(v,v',l) \in out(v), v\in P\}\\
  \end{eqnarray*}
\end{definition}
If $type(v)$ returns the type of a node $v$, the function $relabel$ (abbreviated above as $r$) is defined as 
\begin{definition}[$relabel$] \label{def:relabel}
  For an edge $(v,v',l)$, 
  \[
   (v,v',relabel(l)) = \left\{
   \begin{array}{l}
      l,    {\emph{if}}\; type(v) \neq type(v') \\
      infl, \emph{otherwise}
   \end{array}   \right.
  \]
\end{definition}
\noindent
$relabel$ is promoted in the obvious way to sets of edges. 

To show that  the order of application of two $\col$ functions does not matter, (ie. that $\col$ commutes with itself)  we need to show that for any nodes $w,x,y,z$ in a graph $G$, $\col(y,z,\col(w,x,G)) = \col(w,x,\col(y,z,G))$. 


We begin with some definitions.

$P_1$ is defined as $\Path(w,x)$.

$P_2$ is defined as $\Path(y,z)$.

$\{v_i\} = \Path(x,w) \inter \Path(y,z)$. Note that in general this is a set of nodes. 



%$E_1$ is the set of nodes after $\col(w,x,G)$ has been carried out.

%$E_2$ is the set of nodes after $\col(y,z,G)$ has been carried out.

% If the new node resulting from $\col(y,z,G)$ is $v_{N1}$,

%$P'_1$ is the path between $w$ and $x$ \textbf{after} $\col(y,z,G)$ has been carried out, resulting in the new node $v_{N1}$. It is defined $P'_1 = (P_1\hide\{v_i\})\union\{v_{N1}\}$.  

%Similarly,  $P'_2$ is the path between $y$ and $z$ \textbf{after} $\col(w,x,G)$ has been carried out, resulting in the new node $v_{N2}$.  It is defined as $P'_2 = (P_2\hide\{v_i\})\union\{v_{N2}\}$.  



We  calculate the nodes and edges of  $\col(y,z,\col(w,x,G))$,  then $\col(w,x,\col(y,z,G))$, showing that they both result in the same final graph. For readability, we introduce simplifying definitions through the proof. 


Edges:

Following the definitions recorded above, $\col(w,x,G)$ results in graph $G_1=(V_1,E_1)$, where

\begin{eqnarray*}
  E_1 & = & E\hide (
                   \{(v',v,l)|(v',v,l) \in in(v), v\in P_1\}
                   \union
                   \{(v,v',l)|(v,v',l) \in out(v), v\in P_1\}
                  )\\
  && \union\ \{(v',v_{N},r(l))|(v',v,l) \in in(v), v\in P_1\}\\
  && \union\ \{(v_{N},v',r(l))|(v,v',l) \in out(v), v\in P_1\}\\
\end{eqnarray*}

\noindent
We now introduce the simplifying definitions:
\[
 in(P_1)   = \{(v',v,l)|(v',v,l) \in in(v), v\in P_1\} \\ 
 out(P_1)  = \{(v,v',l)|(v,v',l) \in out(v), v\in P_1\}\\
 in(v_{N1}) = \{(v',v_{N1},r(l))|(v',v,l) \in in(v), v\in P_1\} \\
 out(v_{N1})= \{(v_{N1},v',r(l))|(v,v',l) \in out(v), v\in P_1\} \\
\]
\noindent
With a slight abuse of notation, $in(\{v_{N1}\})$ (resp. $out(\{v_{N1}\})$)  is written $in(v_{N1})$ (resp. $out(v_{N1})$). These simplifies the definition of $E_1$ to 
\[
  E_1  = ( E\hide(in(P_1) \union out(P_1)) ) \union ( in(v_{N1}) \union out(v_{N1}) )
\]
\noindent  
$P_1$ and $P_2$ intersect (on the nodes $\{v_i\}$), so we record the modification in $P_2$ \textbf{after} $\col(w,x,G)$ has been carried out as

It is defined as $P'_2 = (P_2\hide\{v_i\})\union\{v_{N1}\}$.  If $v_f$ is the replacement abstract node, then


\begin{eqnarray*}
  E_{12} =  & = & E_1\hide (
                   \{(v',v,l)|(v',v,l) \in in(v), v\in P'_2\}
                   \union
                   \{(v,v',l)|(v,v',l) \in out(v), v\in P'_2\}
                  )\\
  && \union\ \{(v',v_f,r(l))|(v',v,l) \in in(v), v\in P'_2\}\\
  && \union\ \{(v_f,v',r(l))|(v,v',l) \in out(v), v\in P'_2\}\\
\end{eqnarray*}

\noindent
We  introduce some more simplifying definitions:
\[
in(P'_2) = \{(v',v,l)|(v',v,l) \in in(v), v\in P'_2\}\\
out(P'_2) = \{(v,v',l)|(v,v',l) \in out(v), v\in P'_2\}\\
in(v_f) = \{(v',v_f,r(l))|(v',v,l) \in in(v), v\in P'_2\}\\
out(v_f) = \{(v_f,v',r(l))|(v,v',l) \in out(v), v\in P'_2\}\\
\]
\noindent
allowing us to re-write the defintion of $E_{12}$ as
\[
  E_{12}  =  (E_1\hide ( in(P'_2) \union out(P'_2)))  \union\ (in(v_f) \union out(v_f))
\]
  

\noindent
and then (by substitution of the $E_1$) as


\[
E_{12}  =  \left( \left(
\begin{array}{l}  E\hide(in(P_1) \union out(P_1)) \\  \union \\ (in(v_{N1}) \union out(v_{N1}) ) \\
\end{array} \right)
   \hide ( in(P'_2) \union out(P'_2)) \right) \\
\hphantom{E_{12}  = \;\; }   \union\ \\
\hphantom{E_{12}  = \;\;}   (in(v_f) \union out(v_f))\\ 

\]

\noindent
and then, by substitution of $P'_2$

\[
E_{12}  =  \left( \left(
  \begin{array}{l}
    E\hide(in(P_1) \union out(P_1)) \\  \union \\ (in(v_{N1}) \union out(v_{N1}) ) \\
  \end{array} \right)
   \hide
   \left( \begin{array}{l}
     in((P_2\hide\{v_i\})\union\{v_{N1}\}) \\\union\\ out((P_2\hide\{v_i\})\union\{v_{N1}\})
   \end{array}
   \right) \right) \\
   \hphantom{E_{12}  = \;\; }   \union\ \\
\hphantom{E_{12}  = \;\;}   (in(v_f) \union out(v_f))\\ 
\]

\noindent
then, since $(P_2\hide\{v_i\}) \inter \{v_{N1}\} = \emptyset$, by distributing $in$ and $out$ over $\union$, we get  

\[
E_{12}  =  \left( \left(
  \begin{array}{l}
    E\hide(in(P_1) \union out(P_1)) \\  \union \\ (in(v_{N1}) \union out(v_{N1}) ) \\
  \end{array} \right)
   \hide
   \left( \begin{array}{l}
     in(P_2\hide\{v_i\})\union in(\{v_{N1}\})) \\\union\\ out(P_2\hide\{v_i\})\union out(\{v_{N1}\})
   \end{array}
   \right) \right) \\
   \hphantom{E_{12}  = \;\; }   \union\ \\
\hphantom{E_{12}  = \;\;}   (in(v_f) \union out(v_f))\\ 
\]

\noindent
We can therefore remove $out(\{v_{N1}\})$ and $out(\{v_{N1}\})$ from both sides of the set hiding operator, leaving 

\[
E_{12}  =  \left( \left(
  \begin{array}{l}
    E\hide(in(P_1) \union out(P_1)) \\  
  \end{array} \right)
   \hide
   \left( \begin{array}{l}
     in(P_2\hide\{v_i\}) \\\union\\ out(P_2\hide\{v_i\})
   \end{array}
   \right) \right) \\
   \hphantom{E_{12}  = \;\; }   \union\ \\
\hphantom{E_{12}  = \;\;}   (in(v_f) \union out(v_f))\\ 
\]
\noindent
and, since $\{v_i\}$ is the set of intersection nodes, and therefore $\{v_i\} \subseteq P_1$, we can write

\[
E_{12}  =  
   E\hide
  (
    in(P_1) \union out(P_1) \union in(P_2) \union out(P_2)
  ) \\
   \hphantom{E_{12}  = \;\;}   \union\ \\
\hphantom{E_{12}  = \;\;}   (in(v_f) \union out(v_f))\\ 
\]

Now, we  revisit the defintion of $v_f$. and compare to $v_g$...

Below, we expand the definition of $in(v_f)$. $out(v_f)$ an exercise!

\[
in(v_f) = \{(v',v_f,r(l))|(v',v,l) \in in(v), v\in P'_2\}
\]
\noindent
by defition of $P'_2$,
\[
in(v_f) = \{(v',v_f,r(l))|(v',v,l) \in in(v), v\in((P_2\hide\{v_i\})\union\{v_{N1}\})\}
\]
\noindent
and,  simplification, 
\[
in(v_f) = \{(v',v_f,r(l))|(v',v,l) \in in((P_2\hide\{v_i\}) \union \{v_{N1}\})\}
\]
\noindent
and, by set manipulation, 
\[
in(v_f) = \{(v',v_f,r(l))|(v',v,l) \in in(P_2\hide\{v_i\}) \union  \{(v',v_{N1},r(l))|(v',v,l)\in in(v), v\in P_1\}\}
\]
\noindent
which, by definition of $in(P_1)$, we can write as
\[
in(v_f) = \{(v',v_f,r(l))|(v',v,l) \in in(P_2\hide\{v_i\}) \union  r(in(P_1))\}
\]
\noindent
where $r(in(P))$ is the relabeling operator $r$ promoted to sets.  Which, since $r$ is idempotent ($r(r(l)) = r(l)$), we can write as
\[
in(v_f) = \{(v',v_f,r(l))|(v',v,l) \in in(P_2\hide\{v_i\}) \union  in(P_1)\}
\]
\noindent
and since $\{v_i\} \subseteq P_1$, 
\[
in(v_f) = \{(v',v_f,r(l))|(v',v,l) \in in(P_2) \union  in(P_1)\}
\]

\mnote{still need to repeat this for $E_{21}$ and show two results are equal. This proof might be best in a TR version}

\pagebreak







\mnote{Running examples:  we may need two, one for each operator.
	
	PM: suggest picking p-graphs that are plausible when viewed as the trace of a program, ie using abstract function names for activities.
}
	
\mnote{main contributions:
	\begin{itemize}
		\item two graph rewriting operators for abstracting over provenance graphs, formalised and with proof that the rewriting preserves PROV validity
		\item algebraic properties of the operators (commutative??)
		\item small PROV extension to introduce new relationships when the existing ones are generalised i.e. to ``influence'', that is the new relationship types qualify ``influence''
		
		\item demonstration of how the operators support a variety of abstraction policies and how we cna use a variety of graph metrics to measure the effect of applying a specific policy to a graph
		\item empirical evaluation of the rewriting i.e. using (1) a variety of synthetic graphs, generated using ProvGen, (2) a variety of policies chosen as representative examples and (3) a variety of provenance network metrics
	\end{itemize}
	
}
	
	
\section{The Blur and Collapse operators}
	
	
	\mnote{HOW ABOUT CUTTING RELATIONSHIPS}
	
	\subsection{Formal definitions}
\mnote{we decided:
	
	\begin{itemize}
		\item two separate formal definitions, one for each op.
		\item blur does not follow from collapse
		\item blur +/- does not imply blur (CHECK)
		\item collapse limited to a-a  (supported by intuition on what it means to collapse. collapsing e-e means you merge inputs and outputs which does not make much sense)
  		\item collapse requires using the PROV top-level ``influence'' relationships (only between two activities) --> show an example. This is well supported by intuition: when $a$ uses the intermediate data produced by $a_1$ and consumed by $a_2$, and we collapse $a_1, a_2$ into $a'$, then $a$ becomes related to $a'$. There influence has a natural interpretation.
  			
	\end{itemize}
}

  \subsection{Composing Blur operations}
  
  	\begin{itemize}
	\item blur is compositional, not commutative
	\item thus we define a explicit semantics for dealing with a set of Blur operations --> \textit{union semantics}
\end{itemize}


  \subsection{Composing Collapse operations}

\mnote{  
  	\begin{itemize}
  	  \item collapse is compositional, probably commutative  --> \textbf{proof??}
  	  \item if this is the case, then no need for union semantics
  	\end{itemize}
  	}
  	
  \subsection{Composing Blur and Collapse operations}  
		
	\mnote\textbf{{?? not worked out.} }

\section{Abstraction policies}

\mnote{We want to show how different families of abstraction policies can be specified and enforced using these two operators and their composition.}

\subsection{Policy definition}

\mnote{What is the most abstract definition of an abstraction policy?
	
it consists of two elements:

  	\begin{itemize}
	\item specifying the nodes that are arguments to the operators (and the radius in the case of blur). This specification is either explicit (mentioning specific nodes) or intensional, ie through conditions that predicate on properties of the nodes, eg sensitivity, utility.
	
	\item apply the set of operations. If collapse only then probably commutative (order irrelevant), blur: defined by union semantics for blur. if a mix: not clear.
	
  \end{itemize}
}

\subsection{Policy Examples}
	
		\mnote{1. sensitivity threshold on single node (see old paper). 2. average sensitivity within a set of neighbouring nodes.. 3....}
		
	\mnote{apply each policy to our running example and show the resulting graph}

\section{Empirical evaluation}

\mnote{the aim of the evaluation is to show how the framework allows for the easy specification of policies and comparing their effects on a collection of test graphs, using a variety of provenance network metrics}

\mnote{introduce network metrics: sensitivity, utility, plus provenance metrics as proposed by Moreau 2018}

\mnote{experimental testbed consists of a set of PROV graphs generated using provGen}

\mnote{apply each policy to each graph, measure metrics before /  after, compare as charts / tables}

\section*{Acknowledgements}

The support of the ONRG and the EPSRC in funding this research is gratefully acknowledged. 

\appendix

\section{Proofs}




\section*{References}

%\bibliographystyle{plain}
%\bibliography{prov-abstraction-foundations,p3s-JB}

%\appendix


\end{document}
