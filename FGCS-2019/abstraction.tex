%%-*- mode: LaTeX; mode: FlySpell; -*-

\documentclass{elsarticle}

\usepackage{amssymb,amsmath}

\usepackage{graphicx}
\usepackage{zed-csp}
%\usepackage{csp}
%\usepackage{algorithm}
%\usepackage{algpseudocode}
\usepackage{float}
\usepackage{url}
\usepackage{cite}
\usepackage{tabularx}
%\usepackage{cases}

% The line below was added as a workaround for spacing ``f'' in mathit. \mathit can now be replaced by \Lmathit to get better spacing. 
\DeclareMathAlphabet{\Lmathit}{\encodingdefault}{\familydefault}{m}{it}

\interdisplaylinepenalty=2500

\newfloat{algorithm}{tbp}{lop}
\floatstyle{ruled}

\newtheorem{lemma}{Lemma}%
\newtheorem{definition}{Definition}%
\newtheorem{conjecture}{Conjecture}%
\newtheorem{theorem}{Theorem}%
\newtheorem{proposition}[theorem]{Proposition}

% include if zed-csp is not included.
% \newcommand{\hide}{\setminus}
% \newcommand{\spot}{\bullet}

%include if zed-csp IS included.
\renewcommand{\inv}{\mathit{inval}}
%otherwise
%\newcommand{\inv}{\mathit{inval}}

\newcommand{\prov}{\mathit{prov}}
\newcommand{\dep}{\mathit{dep}}
\newcommand{\pr}{\mathit{pr}}
\newcommand{\obf}{\mathit{abs}}
\newcommand{\pol}{\mathit{pol}}

\newcommand{\pg}{\mathit{PG}}
\newcommand{\en}{\mathit{En}}
\newcommand{\act}{\mathit{Act}}
\newcommand{\ag}{\mathit{Ag}}

\newcommand{\used}{\Lmathit{used}}
\newcommand{\wgby}{\Lmathit{genBy}}
\newcommand{\influence}{\Lmathit{wasInfluencedBy}}
\newcommand{\wdf}{\Lmathit{wasDerivedFrom}}
\newcommand{\waw}{\Lmathit{waw}}
\newcommand{\attrTo}{\Lmathit{wat}}
\newcommand{\wat}{\Lmathit{wat}}
\newcommand{\delegate}{\Lmathit{abo}}
\newcommand{\wasInfBy}{\Lmathit{wasInformedBy}}
\newcommand{\start}{\Lmathit{start}}
\newcommand{\ed}{\Lmathit{end}}


\newcommand{\Ev}{\mathit{Ev}}
\newcommand{\preorder}{\preceq}

\newcommand{\evmap}{\mathit{evmap}}


\newcommand{\node}{\mathit{Node}}
\newcommand{\type}{\mathit{type}}
\newcommand{\elabel}{\mathit{label}}

\newcommand{\guEA}{\pg_{gu/ea}}  % gen-usage over enties, activities
\newcommand{\guiEA}{\pg_{gui/ea}}
\newcommand{\guaEAG}{\pg_{gu+/eaAg}}  % gen-usage and more over enties, activities plus Agents

%% operators
\newcommand{\group}{\mathit{Group}}
\newcommand{\aggroup}{\mathit{agGroup}}
\newcommand{\sgroup}{\mathit{Group_{str}}}
\newcommand{\clos}{\mathit{pclos}}
\newcommand{\repl}{\mathit{replace}}
\newcommand{\agrepl}{\mathit{agreplace}}
\newcommand{\rewire}{\mathit{rewire}}
\newcommand{\extend}{\mathit{extend}}
\newcommand{\pclos}{\mathit{pclos}}
\newcommand{\rem}{\mathit{remove}}
\newcommand{\agremove}{\mathit{AgRemove}}
\newcommand{\orphanremove}{\mathit{OrphanRemove}}
\newcommand{\allorphanremove}{\mathit{AllOrphanRemove}}
\newcommand{\remIsolated}{\mathit{remIsolated}}

\newcommand{\dclos}{\mathit{dclos}}

\newcommand{\POL}{\mathit{\cal P}}

%\newcommand{\mnote}[1] {\marginpar{\scriptsize \raggedright #1 }}
\newcommand{\mnote}[1] {  \framebox{\begin{minipage}[t]{0.9\linewidth}
 \scriptsize \raggedright #1 \normalsize
    \end{minipage}
 }}


%%% Optional notes to highlight the changes in the reviewed document
% \optchange to indicated changed text.
% \optadded to indicate newly text.
% \optremoved to indicate text removed.
% \optnote to write a note which is shown only if changes are on
\newif\ifwithchanges

% Turn on/off change highlighting by un/commenting this toggle
\withchangestrue

\newcommand{\optchange}[1]{%
	\ifwithchanges
	{\leavevmode\color{red}#1}%
	\else
	#1%
	\fi}

\newcommand{\optadded}[1]{%
	\ifwithchanges
	{\leavevmode\color{darkgreen}#1}%
	\else
	#1%
	\fi}

\newcommand{\optremoved}[1]{%
	\ifwithchanges
	{\color{red}\sout{#1}}%
	\fi}

\newcommand{\optnote}[1]{%
	\ifwithchanges
	{\footnotesize\itshape\color{red}\{#1\}}%
	\fi}
%
%%% End of optional note definition


\newenvironment{mydrop}{\begin{array}[t]{@{}l@{}}}{\end{array}}%



\usepackage{color}
\usepackage{pdfpages}
\usepackage{ifthen}
\newcommand{\showComments}{yes} % {yes}
\newcommand{\note}[2]{
    \ifthenelse{\equal{\showComments}{yes}}{\textcolor{#1}{#2}}{}
}
\newcommand{\jwb}[1]{\note{blue}{JWB: #1}}


\begin{document}

\title{Abstracting PROV provenance graphs: a validity-preserving approach}

 
\author[ncl]{P. ~Missier \corref{cor1}}
\ead{Paolo.Missier@newcastle.ac.uk}

\author[cov]{J. ~Bryans\corref{cor1}}
\ead{Jeremy.Bryans@coventry.ac.uk}

\author[ncl]{C. ~Gamble}
\ead{Carl.Gamble@newcastle.ac.uk}
    
\author[kcl]{V. ~Curcin}
\ead{Vasa.Curcin@kcl.ac.uk}

\address[ncl]{School of Computing, Newcastle University, UK}
\address[cov]{Institute for Future Transport and Cities, Coventry University, UK}
\address[kcl]{Kings College, London, UK}

\cortext[cor1]{Corresponding Author}


\markboth{IEEE Transaction on Knowledge and Data Engineering,~Vol.~X, No.~Y, DATE}%
{Shell \MakeLowercase{\textit{et al.}}: happy}


\begin{abstract}
  Data provenance is a structured form of metadata designed to record the activities and datasets involved in data production, as well as their dependency relationships. The PROV data model, released by the W3C in 2013, defines a schema and constraints that together provide a structural and semantic foundation for provenance. This enables the interoperable exchange of provenance between data producers and consumers. When the provenance content is sensitive and subject to disclosure restrictions, however, a way of partially obfuscating provenance in a principled way before communicating it to certain parties is required. In this paper we present a provenance abstraction operator that achieves this goal. It maps a PROV document $P_1$  to a new abstract version $P_2$, with the properties that (i) if $P_1$ satisfies a certain constraint $C$ then $P_2$ also satisfies $C$, (ii) the dependencies that appear in $P_2$ are \textit{justified} by those in $P_1$, i.e., no spurious dependencies are introduced in $P_2$,  \jwbtwo{and (iii) the resultant graph is ``altered'' as little as possible, while ensuring that  the operator is closed with respect to composition.} \jwbtwo{Closure under composition} makes further abstraction of abstract PROV documents possible.  The operator is implemented as part of a user tool, described in a separate paper, that lets owners of sensitive provenance information control the abstraction by specifying an abstraction policy.
\end{abstract}

\begin{keyword}
Provenance \sep Provenance metadata \sep provenance abstraction 
\end{keyword}

\maketitle

% -*- Mode: LaTeX; Mode: Flyspell -*-

\section{Introduction}

\paolo{
The provenance of data is a form of structured metadata that records the processes involved in data production. In addition to references to data generation or transformation processes, a \textit{provenance trace} typically includes input  or intermediate data products as well as references to \textit{agents}, that is the humans or software systems who were responsible to enact those processes.
%
	In multi-party collaboration settings that involve data sharing, as well as in third party auditing of data and processes, there is a broad expectation that 
shipping the available provenance to collaborators, or more generally publishing it along with the data, may help data consumers, including auditors, form judgements regarding the reliability of the data itself.
}

%
\paolo{
	Offering to disclose the provenance of a data as evidential basis for establishing data quality and reliability is particularly important when data products are exchanged as part of transactions that involve parties with limited mutual trust. 
This is the case for instance of \textit{dynamic coalitions}~\citep{BFJM06}, ad hoc collaborative partnerships that are created 
to pursue a common goal, in scenarios such as multi-agency emergency/threat responses, as well as the exchange of intelligence information. 
Despite the need to share data of a possibly sensitive nature, these coalitions are characterized by a lack of established interaction protocols and by limited trust amongst the partners. 
}
\paolo{
This situation creates a tension between data providers and consumers, when it comes to negotiating the level of detail of the provenance that providers are prepared to offer to consumers. 
On the one hand, consumers will require as much provenance detail as possible, to use as a basis for establishing data credibility.
Data providers, on the other hand, will be reticent to offer detailed provenance traces, because those may contain sensitive information regarding their own internal processes as well as any proprietary data used by those processes. 
%
In fact, the provenance of a data product will typically contain more sensitive information than then data product itself.}


\paolo{\subsection{Motivating scenario: provenance of intelligence information}
	%
To appreciate how such tension may arise, consider a scenario where a public agency PA  wants to buy intelligence reports, say about potential threats to the public, from an intelligence provider, IP.
%
Under assumption of limited trust, PA will want to mitigate the risk of acting upon information provided by IP, which is potentially unreliable. 
% 
At the same time, IP has a business incentive to supply PA with additional evidence that facilitates PA's risk assessment and thus increases the chance of a successful transaction.}

%
\paolo{The  key assumption that motivates our work is that the provenance of each intelligence report is relevant in contributing, at least in part, the required evidence. 
%
A fictional but realistic example of provenance for an intelligence report such information is shown in Fig.~\ref{fig:graph-example}.
%
As depicted in the figure, provenance can be visually depicted as a digraph whose nodes represent either \textit{entities} (ovals in the figure), i.e., data, documents, etc., \textit{activities} (rectangles), which represent the execution of some process over a period of time, or \textit{agents} (pentagons), which represent  humans or computing systems. 
%
The edges represent various types of directed relationships, the most common being ``activity $a$ used entity $e$'', ``entity $e$ was generated by activity $a$'', ``activity $a$ was associated with agent $ag$'' (i.e., $ag$ was responsible for $a$), and more.}

\paolo{Formally, such a graph is a depiction of a PROV document, which in turn conforms to the W3C PROV data model~\citep{w3c-prov-dm}, introduced in Sec.~\ref{sec:prov-core}. 
We will use the graph representation of provenance throughout the paper, as it facilitates reasoning about the mechanisms for provenance abstraction, which are at the core of our work.}

\paolo{
The process Reading the graph from top to bottom retains the temporal dependencies amongst its nodes, which are generated as the process 
In this example, the topmost entities \texttt{TwitterFeed-batch1}, \texttt{TwitterFeed-batch2}, \texttt{TwitterFeed-batch3}
	%%%
	%
	}
	

%

\paolo{TBD -- rewriting the example in 1.1. removed fig 1} \\


\paolo{OLD TEXT BELOW}

This may contain a wealth of details regarding how the report was produced, including the raw input data used for the analysis, the analytical tools and their configuration parameters, as well as the identity of the analysts involved, and their position within the business organization.  


%Second, regarding \textit{selective disclosure}, we are motivated by the emerging notion of .
%
%
%
\mnote{In this paper we develop a model and algorithm for performing  abstraction over provenance metadata. These provide the theoretical underpinning, defined on top of PROV, for applications that require provenance exchange while obfuscating some of its content.}

%The work presented in this paper is motivated 

\mnote{The recent standardisation of a data model for exchanging provenance in an interoperable way, namely the PROV model from the W3C~\citep{w3c-prov-dm}, makes this assumption realistic.}
%


%controlling the selective disclosure of potentially sensitive provenance content


%
%
%\begin{figure}
%\centering
%\includegraphics[scale=.35]{figures/intelligence-scenario.png} 
%\caption{Simple two-way partnership involving one sender (IR) and one receiver (PA)}
%\label{fig:scenario}
%\end{figure}

%

\begin{figure*}
\begin{center}
\includegraphics[width=\textwidth]{figures/IR-agents-delegation.pdf}
\caption{Example provenance graph for the reference scenario (Fig.~\ref{fig:scenario})}
\label{fig:graph-example}
\end{center}
\end{figure*}

%
While both IR and PA are interested in sending and receiving such provenance information, some of it may be sensitive and proprietary. 
%
Input datasets may include a combination of public social media streams, shown in the figure, and private databases. Likewise, analytical tools may include proprietary algorithms.
%
In a setting where mutual trust is limited, the contrast between the need to exchange provenance as a form of evidence on one side, and the need to protect confidential information that can be found in the provenance on the other, generates a tension amongst the partners.

%
We believe that such tension can be resolved by providing IR, the \textit{provenance owner}, with ways to control the disclosure of provenance to third parties. This consists of (i) a model of abstraction over provenance, whereby some of its elements are grouped together and replaced with a new, abstract element, and (ii) a policy model that the provenance owner can use to specify the  elements which are to be abstracted.
%
% These are used by the provenance owner to support a process of \emph{negotiation} with the third parties, 

The work presented here is focused exclusively on (i), while (ii)  is the subject of a separate report, focused on a user tool for the specification and enforcement of provenance abstraction policies~\citep{MBGCD14}.  
%
%
%For completeness, however, we provide a brief overview of the tool  and an example of an abstraction policy  in Sec.~\ref{sec:summary}. % Sec~\ref{sec:further} concludes.




\subsection{Options for provenance abstraction }  \label{sec:contributions}

Removal of information from a provenance graph can be achieved in a number of ways.
%
 One could simply remove the labels as well as the annotations from individual nodes and relationships, i.e., anonymize part of the graph. Doing so, however, does not hide any of the structure of the process of data production. One could further remove nodes and relationships, for instance nodes \texttt{consolidateAJC} and \texttt{consolidateBNC} in the graph of Fig.~\ref{fig:graph-example}, or  indeed entire sub-graphs. The new graph will be disconnected, however, making it difficult to reconstruct the lineage of the end data product, that is, the sequence of data derivations from the initial inputs to the outcome of the process.
 
Instead, our abstraction operator replaces a sub-graph with a new abstract node, and then ``re-wires'' the new node to the remaining original graph. This has the effect of hiding parts of the process structure as it was represented in the original provenance, while maintaining connectivity. One can still query the lineage, but some of the provenance elements returned by the query will now be an abstraction of the actual data production process.

The main challenge is to guarantee that abstraction produces PROV-compliant  graphs,   maintaining the interoperability guarantees provided from having standardized PROV and ensuring that the results can be consumed by standard PROV tools. We provide a proof of this in the Appendix.
%demonstrate this 



%compatibility with standard PROV tools can consume and 

% would be compromised.
 
%  A second option, which we take here, is to define a rewriting that generates a PROV-compliant graph.




% We view provenance abstraction as a form of graph rewriting. Conceptually, two approaches can be taken when defining the rewriting rules.
 %
% One option is to extend the PROV model  with additional elements (for instance, new types to denote abstract nodes) and a corresponding system of constraints, resulting in an extension PROV' of PROV.
 %
%  In this setting, a legal rewriting is one that transforms a valid PROV graph $\pg$ into a valid PROV' graph $\pg'$, where validity is interpreted as conformance to the PROV' schema.
  %
%  While this approach leads to the definition of a new and possibly interesting model for \textit{abstract provenance}, the interoperability guarantees provided from having standardized PROV would be compromised.
 
%  A second option, which we take here, is to define a rewriting that generates a PROV-compliant graph.
 

 
\subsection{Contributions}
Our specific contribution is the formal definition of a provenance abstraction operator that rewrites  a PROV graph $\pg$ into a new graph $\pg'$, by mapping a set $V_{gr}$ of nodes (for ``vertex in a group'') in $\pg$ to a new abstract node $v_{new}$, and then mapping each relationship involving elements of $V_{gr}$ to a new relationship involving $v_{new}$ in $\pg'$. 
We prove several formal properties of $\pg'$: firstly, schema preservation: if $\pg$ is a PROV graph, that is, it conforms to the PROV data model~\citep{w3c-prov-dm}, then $\pg'$ is also a PROV graph. Secondly, validity preservation: if $\pg$ is \textit{valid}, that is, it satisfies all PROV constraints~\citep{w3c-prov-dm}, then $\pg'$ is also valid. Finally, no spurious dependencies are introduced into $\pg'$: a  relationship involving $v_{new}$ is only created as a result of a mapping from an existing relationship involving elements of $V_{gr}$. Strictly, if $a$ and $e$ are not \textit{directly} related in $\pg$, we guarantee that they are not directly related in $\pg'$. 
Note that new indirect dependencies between two nodes in $\pg'$, manifested as new paths in the graph, may be introduced, however we show that these are always justified by the topology of the underlying graph $\pg$.

Note that, by making the abstraction operator closed with respect to the set of valid PROV graphs, abstraction can be naturally composed, i.e., one can abstract $\pg'$ into some $\pg''$.

Note also that $\pg'$ itself has also an associated provenance graph, that is, a record of the provenance abstraction process as it was applied to $\pg$. 
PROV provides a syntactic facility to maintain the association between a provenance graph and its own provenance, namely using the ``provenance of provenance'' mechanism (i.e., bundles~\citep{w3c-prov-dm}).



%
%The remainder of the paper is structured as follows. After a review of related work, in Sec.~\ref{sec:prov-core} we introduce the fragment of the PROV model that defines the scope of our work, followed by an overview of the approach and summary of contributions (Sec.~\ref{sec:overview}).
%%
%The core technical material is in Sec.~\ref{sec:grouping}, where we define abstraction over provenance in terms of a \textit{grouping} operator, present its functional specification, and show that grouping maps a graph into a new graph that conforms to the same relational schema.


 
\section{Related Work}  \label{sec:related}

%Work on provenance abstraction generally combines two elements, namely a technique or algorithm for graph editing, and a policy to drive the algorithm. As mentioned, in this paper we focus exclusively on the former, while the latter is described in a separate paper\jwb{~\citep{Missier2014}.}

\paolotwo{Multiple  strands of research relate to our work.
These include creating views over a provenance graph to reduce its complexity, redacting a graph (including non-provenance graphs, such as a social media network) to obfuscate or remove some its sensitive elements, and summarising a collection of graphs. 
We also mention graph access control and anonymisation techniques, which are more peripheral to our work.}

\paolotwo{\subsection{Provenance views and graph redaction}}

\paolotwo{
	Graph redaction is used in ~\citep{Blaustein:2011:SPP:2002974.2002979} to rewrite a graph where particular nodes, node properties, or relationship instances are sensitive. The technique relies on the notion of \textit{surrogates}, which are less sensitive versions of the graph where some information has been either removed or replaced, depending on the clearance level of the user who has access to the graph. 
If a measure of \textit{utility} is associated with elements of the graph, removing or redacting graph elements may reduce the residual utility, and may also result in loss of connectivity. The paper 
paper describes techniques for generating surrogates that achieve a desired protection level, while maximizing graph connectivity and minimising utility loss. 
The technique applies to generic graphs, for instance a social media network, and is also demonstrated on a provenance graph.}

\paolotwo{Similar elements, namely a technique for graph editing, a user clearance policy based on nodes sensitivity and user clearance levels, and a quantitative measure of utility, are indeed at the core of our own ProvAbs system~\citep{Missier2014}. 
This paper extends ProvAbs, but policies and utility maximisation are not discussed as the focus is on proving validity-preservation properties of the redaction operator (i.e., grouping). This is indeed the main distinctive feature that sets our work apart from  ~\citep{Blaustein:2011:SPP:2002974.2002979}, where there are no such notion of a \textit{valid} graph.}

\paolotwo{Also close to our abstraction model, both in motivation and in its technical approach, is the ProPub system~\citep{springerlink:10.1007/978-3-642-22351-8_13}, which computes views over provenance graphs that are suitable for publication by meeting  certain privacy requirements. In ProPub, users specify edit operations on a graph, such as anonymizing, abstracting, and hiding certain parts of it.
% (here the term ``abstracting'' is interpreted as ``zooming out'', much as in~\citep{DBLP:conf/icde/BitonBDH08} mentioned above).
%
The operations are specified as logic rules, and are interpreted natively by the Datalog-based prototype implementation. ProPub adopts an ``apply--detect--repair'' approach, whereby user rules are applied to the graph first, then consistency violations that may occur in the resulting new graph are detected, and a final set of edits are applied to the graph in order to repair such violations. In some cases, this causes nodes that the user wanted removed to be reintroduced, and it is not always possible to satisfy all rules. 
%
In contrast, our grouping involves more simply a set of nodes to be abstracted (but note that anonymization is a particular case, when the group contains a single element). In return for this simplicity in the specification of the nodes to be grouped, our method always produces a valid abstract graph while ensuring that the nodes specified in the policy are removed. 
%, as described in detail in the next section. In order to further clarify the relationship between the two approaches, in Appendix~\ref{sec:appendix} we show how our algorithm computes an abstraction over the same example graph used in the ProPub paper.
}

\paolotwo{Techniques for \textit{provenance redaction} that are based on graph grammars in combination with a redaction policy language are discussed in~\citep{Cadenhead:2011:TPU:1998441.1998456}, where they are deployed to edit provenance that is expressed using the Open Provenance Model~\citep{Moreau2010a} (a precursor to PROV).
%
Although the authors claim that the redaction operators ensure that specific relationships are preserved, this critical issue is not addressed formally in the paper, i.e., with reference to the OPM semantics.
%
In contrast, the formal schema and set of constraints that come with PROV~\citep{w3c-prov-dm,w3c-prov-constraints} provide the necessary grounding for reasoning about the validity-preservation properties of the editing operations.
%Cadenhead, Tyrone, Vaibhav Khadilkar, Murat Kantarcioglu, and Bhavani Thuraisingham. “Transforming Provenance Using Redaction.” In Proceedings of the 16th ACM Symposium on Access Control Models and Technologies, 93–102. New York, NY, USA: ACM, 2011. doi:10.1145/1998441.1998456.
}

\paolo{ZOOM -- discuss ``sound'' workflow views and discuss wrt correctness of abstraction.}

%Work related to our research is broadly motivated either by the need to (simplify provenance graphs to facilitate their understanding by humans, or to enforce access control over parts of the graph, or to summarise a collection of similar provenance graphs in order to reduce both space and query complexity.


was arguably pioneered by the Zoom system~\citep{DBLP:conf/icde/BitonBDH08}. In Zoom, the main assumption is that  the graph is a trace that specifically represents the execution of a dataflow. This is a common occurrence in e-science, where workflows that follow the  dataflow model are a popular high level programming paradigm.
%
In this setting views over provenance are effectively a form of abstraction and are computed based on the user's indication of which workflow modules (tasks) are relevant, or perhaps based on which modules the user has access to. Thus, key to this approach is knowledge of the underlying workflow structure, which is used to specify the nodes in the graphs to be abstracted. This sets Zoom apart from our work, which instead investigates the properties of a grouping operator \textit{independently of the origins of the trace to which it is applied}. 

Also specific to workflow-generated provenance, and thus too narrow in scope for our purposes, is a  strand of research that investigates the problem of preserving the privacy of functions used in workflows, when a large number of input/output pairs for those functions is revealed through the provenance traces of multiple workflow executions. This work on  \textit{module privacy}~\citep{Davidson:2011:PP:1938551.1938554,Davidson2010a,Davidson:2011:PVM:1989284.1989305} is concerned with protecting the semantics of workflow modules. 
It applies anonymization techniques specifically to provenance graphs and is again centred around a workflow-specific form of provenance and  is thus also peripheral to our interest.

% that contain traces of dataflow execution (a ``run''). . This is similar to our work in two ways. Firstly, a view is an answer to a provenance query, which accounts for users preferences and privileges. While provenance sharing policies are not a part of the model and thus are not mentioned explicitly, it is easy to imagine our policies providing input to the view generator.
%Secondly, Zoom too has a notion of consistency, i.e., views must be valid provenance graphs. The main difference with our work is that in Zoom, provenance views are really a by-product of user views over the workflow whose execution the provenance graph represents, whereas our approach is based on a ``PROV-lite'' provenance model that makes no assumptions on the structure of the process that generates the graph.
%



\subsection{Provenance Access Control}

Most of the work on protecting access to sensitive provenance includes policy models that extend traditional data access control frameworks (RBAC), with a distinction made between PBAC (Provenance-Based Access Control) and PAC (Provenance Access Control).
%
PBAC is about policy to specify access rights to data objects based on their provenance.
%
An example, from~\citep{nguyen2012dependency}, is a rule of the form ``only the student submitter can access the graded homework object''. 
%
This rule can be enforced by looking for a dependency path in a provenance graph, whereby a given homework is attributed to a specific student (i.e., relation \textit{IsAuthoredBy} in the Open Provenance Model). This assumes that the object's attribution is explicit in the provenance graph. It is less clear how such a rule would be evaluated when the provenance is incomplete with respect to such attribution dependency, however.

PAC, or how to enforce access control on parts of a provenance graph, is more directly relevant to our work. An analysis of some of the  challenges associated with secure provenance exchange can be found in  \citep{Braun:2008:SP:1496671.1496675}, where 
%[2] U. Braun, A. Shinnar, and M. Seltzer. Secure provenance. In The 3rd USENIX Workshop on Hot Topics in Sec., pages 1–5, Berkeley, CA, USA, 2008.
examples are presented that show how the provenance of data can be more sensitive than the data itself.
%
Another position paper \citep{Hasan:2007:ISP:1314313.1314318}
% [9] R. Hasan, R. Sion, and M. Winslett. Introducing secure provenance: problems and challenges. StorageSS ’07, pages 13–18, 2007.
describes the challenges associated with the exchange of provenance across multiple partners, in a setting where forgery of provenance by malicious users is a possibility, and where users may collude to reveal sensitive provenance to others. These are all common and complex security problems. Unfortunately, the paper stops short of providing any hints at technical solutions, and indeed it is not clear how these problems are specific to provenance, as opposed to data sharing in general.

A concrete specification of an access control system or provenance~\citep{Cadenhead:2011:LPA:1943513.1943532} consists of a XACML-based policy language, in which path queries are used to specify target elements of the graph, as well as an implementation architecture and a prototype.
%Cadenhead, Tyrone, Vaibhav Khadilkar, Murat Kantarcioglu, and Bhavani Thuraisingham. “A Language for Provenance Access Control.” In Proceedings of the First ACM Conference on Data and Application Security and Privacy, 133–144. New York, NY, USA: ACM, 2011. doi:10.1145/1943513.1943532.
%
%\item \citep{Sun:2013:EAC:2435349.2435390}
%Sun, Lianshan, Jaehong Park, and Ravi Sandhu. “Engineering Access Control Policies for Provenance-aware Systems.” In Proceedings of the Third ACM Conference on Data and Application Security and Privacy, 285–292. New York, NY, USA: ACM, 2013. doi:10.1145/2435349.2435390.


%\mnote{Cite as reference: \citep{Altintas2010a}}

\subsection{Summarisation of provenance graphs}

A loosely related strand of research in this area aims at summarising a collection of provenance graphs by constructing a ``super-graph'' that captures the common features across a collection of similar graphs, such as those that are produced by repeating execution of a process with different inputs and parameters. This has been addressed with an aim to improve provenance queries \citep{DBLP:journals/jidm/El-JaickML14}, as well as to provide a compact but approximate representation of provenance at the possible cost of information loss \citep{Ainy:2015:ASD:2806416.2806429}. This work is only peripherally relevant here, as our approach only operates on one graph at a time. 

\paolo{In \citep{moreau2015aggregation} a mechanism is proposed to automatically construct aggregations from a single PROV graph.
This relies on the concept of \textit{provenance types}, which are fixed-length paths in the graph that occur more than once. 
The aggregation is defined as a mapping from provenance nodes to the provenance types, and there is a way to connect these types into a new PROV-like graph, by similarly mapping the graph edges to new weighted edges.
The result is a new graph that is meant to capture the ``essence'' of a fine-grained set of provenance  statements by observing regularities in the original graph.
This is substantially different from our approach, namely (i) the choice of nodes to aggregate is driven by the discovery of provenance types, which is entirely driven by graph topology and not by a user choice, and (ii) there is no intent to generate \textit{valid} PROV graphs, which is instead the main goal of our transformation.  
Thus, the approach is not suitable to support policy-driven (or other user-oriented) selective disclosure, and the aggregation operation produces a graph that may violate PROV constraints.
}

\subsection{General graph anonymization}

For completeness, we briefly mention more general techniques for graph editing, largely motivated by the need to preserve privacy in social network data. This body of work, which is not specific to provenance, extends the well-known data anynomization framework developed for relational data to graph data structures~\citep{springerlink:10.1007/978-3-540-78478-4_9,Bhagat:2009:CGA:1687627.1687714,Liu:2008:TIA:1376616.1376629}. The main idea is to randomly remove arcs between two nodes and replace them with new ones. As arcs in PROV graphs represent relationships with a given semantics, this approach generally results in false dependencies being created in the edited graph, and is therefore not viable. 
%
The main value of this body of work in this setting, as summarised in~\citep{Zhou:2008:BSA:1540276.1540279}, is to ensure that various forms of anonymization are provably robust to attacks from adversaries who can potentially leverage their partial information about fragments of the graph, to infer additional knowledge. In this paper we do not discuss the robustness of abstraction by grouping, indeed we do not consider any specific threats, and so the challenge of preventing the reconstruction of the abstracted fragments of provenance graphs is left for future work.
%\citep{Bhagat:2009:CGA:1687627.1687714}  // anon social network graphs -- shrivastava

%\citep{Liu:2008:TIA:1376616.1376629}  // Towards identity anonymization on graphs

%\citep{Zhou:2008:BSA:1540276.1540279}  // survey of anon graph data for social network applications



%%%%%%%%%%%%
%%
%%%%%%%%%%%%

\section{Background}

\subsection{Core PROV model} \label{sec:prov-core}

We now introduce the core elements of the PROV model, which forms the basis for the grouping operator.
%
We maintain a dual view of provenance, both as a relational model (with binary relations) and as a graph model. Viewed as a relational model, PROV includes three types of elements: Entities ($\en$), Activities ($\act$), and Agents ($\ag$), and several types of relations amongst them. 
In line with the description in~\citep{w3c-prov-dm} (sec. 2), PROV is defined by the following core relations, with common abbreviations in brackets. 

\begin{eqnarray*}
Used~~(\used)  & \subseteq & \act \times \en \\
WasGeneratedBy~~(\wgby) & \subseteq  & \en \times \act \\
WasDerivedFrom~~(\wdf) & \subseteq   & \en \times \en \\
WasInvalidatedBy~~(\inv) &  \subseteq &  \en \times \act \\
WasAssociatedWith~~(\waw) & \subseteq & \act \times \ag \\
ActedOnBehalfOg~~(\delegate) & \subseteq & \ag \times \ag \\ 
WasAttributedTo~~(\attrTo) & \subseteq & \en \times \ag \\
WasInformedBy~~(\wasInfBy) & \subseteq & \act \times \act
\end{eqnarray*}


% \begin{figure}
% \centering
% \includegraphics[scale=.45]{figures/prov-essentials.pdf} 
% \caption{Core elements of the PROV model, adapted from~\citep{w3c-prov-dm}}
% \label{fig:prov-core}
%  \end{figure}
% %

%\subsection{Bipartite PROV: $\guEA$}  \label{sec:prov-guea}
%These are summarized in Fig.~\ref{fig:prov-core}.
%
Initially, we are going to restrict ourselves to an even simpler model, consisting only of $\en$, $\act$, and relations $\used$ and $\wgby$.
%Agents and the relations that involve them are introduced in Sec.~\ref{sec:agents-abstraction}.
%
%Further extensions to the additional relations --- $\wdf$ and $\wasInfBy$ --- are straightforward and are not considered in detail.

%
An instance  of the model is a provenance document $D$, consisting of sets $en \in \en$ and $act \in \act$ of symbols, and sets of relation instances $\{ \wgby(e,a)  | e \in \en, a \in \act \} \cup   \{ \used(a,e)  | e \in \en, a \in \act\}$. 

%
As these relations are binary, we view $D$ as a digraph $G=(V,E)$, where $V= \en \cup \act$, and each relation instance maps to a labelled directed edge. By convention, we orient these edges from right to left, to denote that the relation ``points back to the past''. Thus:
$a \xleftarrow{\wgby} e \in E$ iff $\wgby(e,a) \in D$, and $e \xleftarrow{\used} a \in E$ iff $\used(a,e) \in D$.
%
We denote the label associated to edge $(v_i, v_j)$ as $\elabel(v_i,v_j)$. 

%
Note that, by definition of the relations, $G$ is a bipartite graph.
We denote the set of all such graphs, containing only $\en$ and $\act$ nodes, and $\wgby$ and $\used$ edges,  by $PG$. 
%In Sec.~\ref{sec:agents-abstraction} we are going to extend this set to include agents as well as additional relations.
%Fig.{fig:baseline-ug-ae} portrays a simple $\guEA$ graph that we will be using as a running example.

% \begin{figure}
% \centering
% \includegraphics[scale=.6]{figures/baseline-ug-ae.pdf} 
% \caption{$\guEA$ provenance graph used as a running example to illustrate abstraction by grouping}  \label{fig:baseline-ug-ae}
% \end{figure}
% 


\section{Grouping Provenance graph nodes}  \label{sec:grouping}

\comment{discuss (somewhere) implications of relaxing the assumption about connectivity in Section~\ref{sec:closure}.}



As mentioned in Sec.~\ref{sec:contributions}, our goal is to define graph editing operators that selectively remove information from a graph $G \in \guEA$, yielding a new graph $G' \in \guEA$.
%
The first of these, namely the removal of labels or annotations associated with a node or an edge, is straightforward. 
%
Regarding the removal of a node, we note that simply reconnecting the remaining nodes generally may lead to an invalid graph. 
%
A simple example is a graph defined by: $\{\used(a, e_1)$, $\wgby(e_2, a) \}$ where activity $a$ is removed. This results simply in two disconnected nodes $e_1$, $e_2$, because no relationship can be inferred between them from the original graph.


Rather than delving into the possible consequences of such node and edge elision, we are going to focus exclusively on the $\group$ graph transformation operator as the prime way to achieve abstraction over provenance graphs.
%
$\group$ takes a graph $G = (V, E) \in \guEA$ and a subset $V_{gr} \subset V$ of its nodes \jwb{that the user wishes to obfuscate} and produces a modified graph $G' \in \guEA$. The nodes in $V_{gr}$ are ``grouped'' together and replaced by a new single node.
%
  
\[ \group : \guEA \times \mathbb{P}(V) \rightarrow \guEA \]
  
%Here we define $\group$ over $\guEA$ graphs. This will be extended to abstraction over Agents in Sec.~\ref{sec:agents-abstraction}.

%The $\group$ operator has the following signature:

%
As the operator is closed under composition, further abstraction can be achieved by repeated grouping, either on multiple disjoint sets $V_{gr}$, or on sets that include abstract nodes (abstraction of abstraction).



%
To get a quick intuition of the  problems faced in the definition of the grouping operator, consider the transformation in Fig.~\ref{fig:non-convex-ex-1}, where nodes $V_{gr} = \{e_1, e_3, e_4, e_5\}$ are simply replaced with new node $e'$ in the example graph of Fig.~\ref{fig:baseline-ug-ae}, and all edges in and out of nodes in $V_{gr}$ are just ``rewired'' in and out of $e'$. 



\begin{figure*}
\centering
\includegraphics[scale=.5]{figures/non-convex-ex-1.pdf} 
\caption{A naive replacement of a set of nodes may lead to a non-$\guEA$ graph.} \label{fig:non-convex-ex-1}
\end{figure*}

This naive replacement leads to problems. Firstly, it introduces two cycles: $e'  \leftrightarrow a_1$ and $e'  \leftrightarrow a_3$. Furthermore, the two edges 
$e'  \leftarrow a_1$ and $e'  \leftarrow a_3$ cannot be of type $\wgby$, while at the same time one cannot arbitrarily introduce $\used$ relations, which would be false dependencies.
%the new node $e'$ appears to have been generated by two distinct activities, $a_1$ and $a_3$.
Thus, the resulting graph is not a valid PROV graph. Note that the former of these problems had been already pointed out in the description of the ProPub system~\citep{springerlink:10.1007/978-3-642-22351-8_13}, mentioned above.

\subsection{Closure and homogeneous grouping}
\label{sec:closure}

The example suggests that the \jwb{issue} is caused by nodes $a_1$ and $a_3$, which both lie on the paths between two of the nodes in $V_{gr}$. Intuitively, set $V_{gr}$ is not ``convex'', that is, there are paths in $G$ that lead out of $V_{gr}$ and then back in again. This observation suggests the introduction of a preliminary closure operation \textit{pclos}, which ensures acyclicity by capturing \jwb{and including these paths}. It is defined as follows.

%%%%
%% closure
%%%%
\vspace*{10pt}
\begin{definition}[Path Closure]
\label{def:clos}
Let $G = (V,E) \in \guEA$ be a provenance graph, and let $V_{gr} \subset V$.  
For each pair  $v_i, v_j \in V_{gr}$ such that there \jwb{one or more directed paths} $v_i \leadsto v_j$ in $G$, let $V_{ij} \subset V$ be the set of all nodes in \jwb{all paths $v_i \leadsto v_j$}.
The Path Closure of $V_{gr}$ in $G$ is
\[\clos(V_{gr}, \jwb{G})  =  \bigcup_{v_i, v_j \in V_{gr}} V_{ij} \]
\end{definition}
\jwb{At this point, we make the assumption that, if we remove typing and directional information from $G$, and view $G$ as an \emph{undirected} graph, then  $\clos(V_{gr}, G)$ results in a single \emph{connected component} of the undirected graph. In Section~\ref{} we explore the implications of relaxing this assumption.} \comment{reference foward needed.} \comment{need to demonstrate later that the order of operations is irrelevant}. 


\begin{figure*}
\centering
\includegraphics[scale=.5]{figures/convex-ex-1.pdf} 
\caption{Path closure and replacement with extension on a set of entity nodes.} \label{fig:convex-ex-1}
\end{figure*}

Fig.~\ref{fig:convex-ex-1} shows a continuation of the previous example. This time the replacement is performed on $\clos(\{e_1, e_3, e_4, e_5\},G) = \{e_1, e_3, e_4, e_5, a_1, a_3\}$, \jwb{(in Fig.}~\ref{fig:convex-ex-1}(b)),  resulting in graph \jwb{in Fig.~\ref{fig:convex-ex-1}(c).}
%
However, while this solves the cycle problem, the graph is no longer bipartite, \jwb{because} the new edges $e' \rightarrow e_2$ and $e' \rightarrow e_6$ connect nodes of the same type.
%
In this example, we can construct a new group of nodes, $\{ e', e_2, e_6\}$, on the graph that results from the first replacement, and replace it with a new node $e''$. The resulting graph \jwb{Fig.~\ref{fig:convex-ex-1}(d)} is a valid $\guEA$ graph.

%
The same result can be obtained by first \textit{extending} the closure in \jwb{Fig.~\ref{fig:convex-ex-1}(b)} to include e-nodes $e_2$, $e_6$, and then replacing the resulting set with $e''$ (this is indicated by the ``extend and replace'' arrow from \jwb{Fig.~\ref{fig:convex-ex-1}(b)} to \jwb{Fig.~\ref{fig:convex-ex-1}(d) as} shown).
%
\jwb{The \textit{extend} operator will have the role of ``gathering up'' certain nodes into the set to be replaced. We require all the set boundary nodes (nodes connected to nodes not in the set) to be of the same type, so in Fig.~\ref{fig:convex-ex-1}(b) we must  include nodes $e_2$ and $e_6$.}

Following this approach, we are going to define grouping as a composition of three functions: \textit{closure}, defined above, \textit{extension}, and \textit{replacement}, as follows.

%
The \textit{extension} of a set $V_{gr} \subset V$ relative to type $t \in \{ \en, \act \}$ is $V_{gr}$ augmented with all its adjacent nodes, in either direction, of type $t$. Formally:


%%%%
%% extension
%%%%
% \vspace*{10pt}
% \begin{definition}[$\extend$]
% Let $G = (V,E) \in \guEA$, $t \in \{ \en, \act \}$.
% \[
% \begin{array}{l}
% \extend(V_{gr}, G ,t) =  \\
% \quad V_{gr} \;\cup \\ 
% \quad    \{ v' | (v \leftarrow v') \in E \wedge v \in V_{gr} \wedge \type(v') = t \} \;\cup \\
% \quad   \{ v | (v' \leftarrow v) \in E \wedge v \in V_{gr} \wedge \jwb{\type(v) = t} \}  \\
% \end{array}
% \]
%\end{definition}
\comment{The nodes we to which we extend must have the type $t$. It isn't possible to include a node of type other than t.}

\comment{To make this more clear, we include $v_s(v_d \nin V_{gr}$ in the definition of $extend$. This would also make this definition more compatible with the definitions of incut and outcut.}


\vspace*{10pt}
\begin{definition}[$\extend$]
Let $G = (V,E) \in \guEA$, $t \in \{ \en, \act \}$. $v_s$ and $v_d$ are the source and destination nodes of a relationship. 
\[
\begin{array}{l}
\extend(V_{gr}, G ,t) =  \\
\quad V_{gr} \;\cup \\ 
\quad    \{ v_d | (v_d \leftarrow v_s) \in E \wedge v_s \in V_{gr} \jwb{~\wedge\ v_d \nin V_{gr}} \wedge \type(v_d) = t \} \;\cup \\
\quad   \{ v_s | (v_d \leftarrow v_s) \in E  \jwb{~\wedge\ v_s \nin V_{gr}} \wedge v_d \in V_{gr}  \wedge \jwb{\type(v_s) = t} \}  \\
\end{array}
\]


\end{definition}

\comment{I've replaced the phrase ``sink node'' with ``boundary node'' in the following.}

%
In our example, $\extend(\{e_1, e_3, e_4, e_5, a_1, a_3\}, G, \en) = \{e_1, e_3, e_4, e_5, a_1, a_3, e_2, e_6\}$.  Note that all boundary nodes in $\extend(V_{gr}, G, t)$ are of type $t$ by construction.
%\footnote{A boundary node is any node in the extended set with a link to a node not in the set.}

Next, we consider replacing the collected nodes with a new abstract node.
%
Let $V^* \subset V$ be obtained using $\pclos$ then $\extend$, as outlined above, and let $v_{new}$ be a new node symbol that does not appear in $V$, created using $\pclos$ then $\extend$. 
Function $\repl$ replaces $V^*$ with $v_{new}$ in $V$, and connects $v_{new}$ to the rest of the graph, as follows.

Let $\vartheta_{out}(V^*)$ denote the \textit{outcut} of $G$ associated with $V^*$:
\jwb{\[ \vartheta_{out}(V^*) = \{  (v_d \leftarrow v_s) |   v_s \in  V^*, v_d \in V \setminus V^*\}\]}
This is the set of arcs of $G$ pointing out of $V^*$,
%\jwb{(i.e., those  whose destinations lie in $V^*$ and whose sources lie in $V \setminus V^*$ recalling that relationships point backwards in time). }

Symmetrically, let $\vartheta_{in}(V^*)$ denote the \textit{incut}  of $G$ associated with $V^*$, i.e., the set of arcs of $G$ leading into $V^*$:
\jwb{\[ \vartheta_{in}(V^*) = \{  (v_d\leftarrow v_s) |  v_d \in V^*, v_s \in  V \setminus V^* \}\]}

Finally, let $\vartheta_{int}(V^*)$ denote internal arcs, that connect two nodes inside $V^*$:
\[ \vartheta_{int}(V^*) = \{  (v_s\leftarrow v_d) | v_s, v_d \in V^*\}\]


\jwb{Function $\repl$ replaces each arc $(v_d \leftarrow v_s) \in \vartheta_{out}(V^*)$ with a new arc $(v_{new} \leftarrow v_s)$ of the same type, and replaces each arc $(v_d \leftarrow v_s) \in \vartheta_{in}(V^*)$ with a new arc $(v_s \leftarrow v_{new})$ of the same type. Arcs in $\vartheta_{int}(V^*)$ simply disappear along with the nodes in $V^*$.}
%


%
\jwb{The definitions of $\vartheta_{out}'(V^*)$ and $\vartheta_{in}'(V^*)$ below define the final part of the ``rewiring'' carried out by $\repl$. }

\begin{definition}
  Let $ty \in \{\used,  \wgby\}$. Then:
  \begin{align*}
\vartheta_{out}'(V^*) = \{ & \jwb{(}v \xleftarrow{ty}  v_{new} \jwb{)}|  v \xleftarrow{ty} v' \in \vartheta_{out}(V^*)  \}  \\
\vartheta_{in}'(V^*) = \{ & \jwb{(}v_{new} \xleftarrow{ty} v \jwb{)} | v' \xleftarrow{ty} v \in \vartheta_{in}(V^*)  \}   
\end{align*}
\label{def:eq:outcut}
\end{definition}

% \begin{align}
% \vartheta_{out}'(V_{gr}') = \{ & v \xleftarrow{t}  v_{new} |  v \xleftarrow{t} v' \in \vartheta_{out}(V_{gr}')  \}  \\
% \vartheta_{in}'(V_{gr}') = \{ & v_{new} \xleftarrow{t} v | v' \xleftarrow{t} v \in \vartheta_{in}(V_{gr}')  \}   \label{eq:outcut}
% \end{align}
% 

\noindent
\jwb{And the full definition of $\repl$ is}
\vspace*{10pt}
\begin{definition}[replace]
\label{def:group-replace}

\[ \repl (V^*, v_{new}, G) = (V', E'), \mbox{ where: } \]
\begin{eqnarray*}
V' & = & V  \setminus V^*  \cup \{v_{new}\}\\
E' & = & E  \setminus (\vartheta_{out}(V^*) \cup \vartheta_{in}(V^*) \cup \vartheta_{int}(V^*))  \\
   & & \qquad \cup\  \vartheta_{out}'(V^*)  \cup \vartheta_{in}'(V^*)
\end{eqnarray*}
% \begin{align*}
% V' = V & \setminus V_{gr}  \cup \{v_{new}\}\\
% E' = E & \setminus (\vartheta_{out}(V_{gr}) \cup \vartheta_{in}(V_{gr}) \cup \vartheta_{int}(V_{gr}))  \\
%     & \cup \vartheta_{out}'(V_{gr})  \cup \vartheta_{in}'(V_{gr})
% \end{align*}
\end{definition}

It is easy to verify that the resulting graph is type-correct. All boundary nodes in \jwb{$V^*$}  are of the \jwb{same type $t \in \{En,Act\}$,} as noted above, and   $v_{new}$ \jwb{is of type $t$}  by construction.
%Thus, boundary nodes are replaced by a node $v_{new}$ of the same type.
Since the arcs have the same type as those they replace, it follows that $\repl$ preserves type correctness.



\begin{figure*}
\centering
\includegraphics[scale=.5]{figures/convex-a-only.pdf} 
\caption{Grouping on a set of Activity nodes} \label{fig:convex-a-only}
\end{figure*}

We can now provide an initial definition of our $\group$ operator, under the simplifying assumption that all nodes in $V_{gr}$ are of the same type, either $\en$ or $\act$, which we denote by $\type(V_{gr})$ (with a slight abuse of notation). \jwb{Definitions~\ref{def:t-grouping} and~\ref{def:strict-t-grouping} remove this assumption.  }


%\comment{I don't think we should assume $V_{gr}$ is of homogeneous type. This is too restrictive. All we need is to identify in advance the ``target'' type of $v_{new}$.}
Fig.~\ref{fig:convex-a-only} shows a progression similar to that of Fig.~\ref{fig:convex-ex-1}, but this time $\type(v) = \act$ for each $v \in V_{gr} = \{a_1, a_2, a_3\}$, and $V_{gr}$ is replaced by another activity node, $a'$.
%
Under assumption of type homogeneity, the grouping operator is a functional composition of $\clos$, $\extend$, and $\repl$ functions, defined as follows.

\vspace*{10pt}
\begin{definition}[Homogeneous Grouping]
Let $G=(V,E) \in \guEA$, $V_{gr} \in V$ be a type-homogeneous set, and let $v_{new}$ be a new node with $\type(v_{new}) = \type(V_{gr})$.
\begin{align*}
\group_{hom} &  (G, V_{gr}, v_{new}) = \\
 & \repl(  \\
 & \extend( \\
 & \clos(V_{gr},\jwb{G}), V, \type(V_{gr})), v_{new},  G ) 
\end{align*}
\label{def:homo-group}
\end{definition}

As an illustration, in our running example in Figure~\ref{fig:convex`-ex-1} we have: %Ive altered this example: see latex for orig.
\begin{align*}
V_{gr} &= \{e_1, e_3, e_4, e_5\} \\
V_{cl} & = \clos(V_{gr},G) = \{e_1, e_3, e_4, e_5, a_1, a_3\} \\
V^{*} & = \extend(V_{cl}, \en) = V_{cl} \cup  \{e_2, e_6 \} \\
\group_e(G, V_{gr}, v_{new}) & = \repl(V^{*}, v_{new}, G) \\
& = (\{a_1,a_2,a_3,e''\}, \{ (a_2, e''), (a_4, e''), (e'',a_5) \})
\end{align*}

\jwb{In this section we have described the grouping operator in terms of the component functional parts.}
%
\jwb{We made two assumptions that have to be discharged. The first, made in Definition~\ref{def:clos}, was that, if we removed typing and directional information from $G$,  that $\pclos(V_{gr},G)$ resulted in a single connected component. We show the implications of relaxing this this in Section~\ref{sec:?}. The second assumption, that all nodes in the initial selected set are the same type, is dealt with in the following  Section.}


\subsection{Generalization to \textit{e-grouping} and \textit{a-grouping}}
\label{sec:generalisation}

So far we have considered grouping over sets $V_{gr}$ of type-homogeneous nodes (before closure). Additional care must be taken if we allow $V_{gr}$ to include nodes of mixed types.  First, the type of the replacement node must now be specified, as it is no longer implied from the type of the nodes in $V_{gr}$. Indeed, the choice of such type leads to different abstracted graphs. Thus, we will now refer to grouping as \textit{t-grouping}, where $t \in \{ \en, \act\}$, i.e., \textbf{e-grouping} or \textbf{a-grouping}. Fig.~\ref{fig:e2-a4}(a-1, a-2) illustrates the application of the $\group_{hom}$ operator (Def.~\ref{def:homo-group}), assuming \textit{a-grouping} and $V_{gr} = \{ e_4, a_2\}$. Note that the extension incorporates activity node $a_1$. 

Second, observe that a new pattern arises in the case of \textit{e-grouping} as shown in Fig.~\ref{fig:e2-a4}(e-1, e-2). Now the extension leads to $V_{cl} = V_{gr} \cup \{ e_5\}$, which in turn leads to the pattern shown in Fig.~\ref{fig:e2-a4}(e-2), involving two generation events for the new entity $e_{N}$.

Although this is a valid pattern, the two generation events must be simultaneous (this is one of the temporal constraints defined in~\citep{w3c-prov-constraints}):
\begin{align*}
ev(\wgby(e_N, a_1)) & \preorder ev(\wgby(e_N, a_3))  \quad \wedge \\
ev(\wgby(e_N, a_3)) & \preorder ev(\wgby(e_N, a_1))
\end{align*}

The intuitive interpretation for this pattern is that each of the two activities generated one entity in the group represented by $e_N$, and that the abstraction makes these two events indistinguishable. Formally, nothing further needs to be done to the graph. However note that one can restore, if desired, the more natural pattern whereby one single generation event is recorded for $e_N$. This is achieved simply by propagating the grouping to the set of generating activities. In the example, this leads to the graph in Fig.~\ref{fig:e2-a4}(e-3).  


\begin{figure*}
\centering
\includegraphics[scale=.5]{figures/e2-a4.pdf} 
\caption{e-grouping and a-grouping on mixed type nodes} \label{fig:e2-a4}
\end{figure*}

We now formalize these considerations by introducing two definitions for $\group$. The first, which we call \textbf{t-grouping}, is agnostic of multiple generation patterns, while the second (\textbf{strict t-grouping}) applies propagation to ensure that the graph is free from multiple generation patterns.

\vspace{10pt}
\begin{definition}[t-Grouping]
\label{def:t-grouping}
Let $G=(V,E) \in \guEA$, $V_{gr} \in V$, $t \in \{\en, \act\}$, and let  $v_{new}$ be a new node with $\type(v_{new}) = t$.
%
Then:
\begin{align*} 
\group & (G, V_{gr}, v_{new}, t) = \\
& \repl( \extend(\clos(V_{gr},\jwb{G}), V, t), v_{new},  G )
\end{align*}
\label{eq:t-grouping}
\end{definition}

Note that the assumption that \jwb{boundary} nodes in the closure are homogeneous and are replaced by a node of the same type $t$, which is necessary for $\repl$ to perform correctly, still holds in this case.

\begin{definition}[Strict t-Grouping]
Given 
$G=(V,E) \in \guEA$, $V_{gr} \in V$, $t \in \{\en, \act\}$, and a new node $v_{new}$ with $\type(v_{new}) = t$, let
\[ G' = (V',E') = \group(G, V_{gr}, v_{new}, t). \]
Let 
$V_{gen} = \{ a \in V' |  a \xleftarrow{\wgby} v_{new} \in E' \}$ be the set of activity nodes that generate $v_{new}$ according to $G'$, and let $a_{new}$ be a new activity node. Then:
\begin{equation*}
%\footnotesize
\sgroup(G, V_{gr},v_{new}, t)=
\begin{cases}
G' & \!\text{if}  |\!V_{gen}\!| \leq 1  \\
\repl(V_{gen}, a_{new}, G') & \!\text{otherwise } 
\end{cases}
%\normalsize
\end{equation*}
\label{def:strict-t-grouping}
\end{definition}

% \begin{align*}
% \sgroup & (G, V_{gr},v_{new}, t)=\\%
% & \begin{cases}
%     G' & \!\text{if}  |\!V_{gen}\!| \leq 1  \\
%     \repl(V_{gen}, a_{new}, G') & \!\text{otherwise } 
%   \end{cases}
% \end{align*}
% 
% 

%\jwb{if a1 and a3 were linked via an intermediate entity node, would this be a problem?}

It is straightforward to show that strict t-grouping reduces to normal grouping if the grouping is a homogeneous a-grouping:
\begin{align*}
&\text{if } \type(t) =  \act \\
 &\qquad\text{then } \sgroup(G, V_{gr},t) = \group(G, V_{gr},t). 
\end{align*}

\comment{Here is one possible place to relax assumption made in definition 2, and talk about ordering of two overlapping group events. } 

\comment{Two overlapping group events can produce the same result only if they both reusult in the same node type. }

\comment{Round off the section by summarising and pointing forward.}

\jwb{summary, look ahead}

%
%
%In addition, however, they are also subject to ordering constraints relative to events associated to elements in in $V_{cl}$ (the set of nodes to be grouped, resulting from Path Closure on source graph $G'$), which have now been replaced by the grouping nodes. To illustrate this reasoning, consider for instance the simple graph in Fig.~\ref{fig:simple-prototype-pattern-events}(a), and let $V_{gr}= \{ e_1, e_2\}$. The ordering constraints on $G$ are as follows:
%
%
%\begin{figure}
%\centering
%\includegraphics[scale=.5]{figures/simple-prototype-pattern-events} 
%\caption{Simple graph patterns to illustrate ordering constraints on events after grouping} \label{fig:simple-prototype-pattern-events}
%\end{figure}
%
%
%\begin{align*}
%start_{ev}(a_1) \preorder gen_{ev}(\wgby(e_i, a_1)) \preorder end_{ev}(a_1)   \text{ for } i=1, i=2\\
%start_{ev}(a_2) \preorder usage_{ev}(\used(a_2,e_1)) \preorder end_{ev}(a_2) \\
%start_{ev}(a_j) \preorder usage_{ev}(\used(a_j,e_2)) \preorder end_{ev}(a_j) \text{ for } j=2, j=3 \\
%gen_{ev}(\wgby(e_2, a_1))  \preorder usage_{ev}(\used(a_j,e_2))  \text{ for } j=1, j=2 \\
%gen_{ev}(\wgby(e_1, a_1))  \preorder usage_{ev}(\used(a_2,e_1)) \\
%\end{align*}
%%
%The corresponding ordering constraints on $G'$ are as follows.
%%
%\begin{align}
%start_{ev}(a_1) \preorder gen_{ev}(\wgby(e_N, a_1)) \preorder end_{ev}(a_1)  \label{eq:constraints-example-1}  \\
%start_{ev}(a_2) \preorder usage_{ev}(\used(a_2, e_N)) \preorder end_{ev}(a_2) \\
%start_{ev}(a_3) \preorder usage_{ev}(\used(a_3, e_N)) \preorder end_{ev}(a_3) \\
%gen_{ev}(\wgby(e_N, a_1))  \preorder usage_{ev}(\used(a_2,e_N)) \\
%gen_{ev}(\wgby(e_N, a_1))  \preorder usage_{ev}(\used(a_3,e_N)) \label{eq:constraints-example-n} 
%\end{align}
%
%The following additional ordering constraints further characterize events in $G'$ in terms of events in $G$. It is easy to see that such constraints are sufficient conditions for the constraints \ref{eq:constraints-example-1}-\ref{eq:constraints-example-n} above to hold.
%\begin{align*}
%usage_{ev}(\used(a_2,e_1)) \preorder usage_{ev}(\used(a_2,e_N))  \\
%usage_{ev}(\used(a_2,e_2)) \preorder usage_{ev}(\used(a_2,e_N)) \\
%usage_{ev}(\used(a_3,e_3)) \preorder usage_{ev}(\used(a_3,e_N)) \\
%gen_{ev}(\wgby(e_N, a_1)) \preorder gen_{ev}(\wgby(e_1, a_1)) \\
%gen_{ev}(\wgby(e_N, a_1)) \preorder gen_{ev}(\wgby(e_2, a_1)) )
%\end{align*}
%
%A similar reasoning applies to the case of a-grouping, illustrated in Fig.~\ref{fig:simple-prototype-pattern-events}(b), where the following definitions are consistent with the temporal orderings:
%\begin{align*}
%usage_{ev}(\used(a_1,e_1)) \preorder usage_{ev}(\used(a_N,e_1))  \\
%usage_{ev}(\used(a_2,e_1) ) \preorder usage_{ev}(\used(a_N,e_1))  \\
%gen_{ev}(\wgby(e_2, a_N))  \preorder  gen_{ev}(\wgby(e_2, a_1)) \\
%gen_{ev}(\wgby(e_3, a_N))  \preorder gen_{ev}(\wgby(e_3, a_2)) \\
%start_{ev}(a_N) \preorder start_{ev}(a_1) \\
%start_{ev}(a_N) \preorder  start_{ev}(a_2) ) \\
%end_{ev}(a_1) \preorder end_{ev}(a_N)  \\
%end_{ev}(a_2) ) \preorder end_{ev}(a_N)  
%\end{align*}
%
%Following this reasoning, we define the following additional ordering constraints amongst events in $G'$ and $G$ events.

%\begin{figure}
%\centering
%\includegraphics[scale=.5]{figures/a-grouping-pattern-proofs} 
%\caption{$\guEA$ graph patterns for e- and a-grouping} \label{fig:a-grouping-pattern-proofs}
%\end{figure}

\jwb{
\subsection{relaxing teh second assumption}
\label{sec:relaxing}
}


%%-*- mode: LaTeX; mode: FlySpell; -*-

\section{Abstraction over events}
\label{sec:event}

\begin{figure*}
	\centering
	\includegraphics[scale=.5]{figures/e4-e5.pdf} 
	\caption{Abstraction over a document content, and associated abstracted events} \label{fig:e4-e5}
\end{figure*}

In Sec.~\ref{sec:prov-constraints} we recalled the definition of PROV ordering  constraints C2-C7, given in the PROV-CONSTRAINT document, which must be satisfied by any valid PROV graph.
We now want to extend such formal definition of validity to include \textit{abstract} graphs $G'$. 
To achieve this, we must first define suitable events on $G'$. 
When $G'$ is obtained using either e-grouping or a-grouping over some base graph $G$,  in general these are not the same events as $G$'s, because both entities and relationships may have changed. 
Specifically, when $a_{new}$ is created through a-grouping, $G'$s events include 
 $start(a_{new})$, $end(a_{new})$, as well as 
$ev(\used(e, a_{new}))$, $ev(\wgby(a_{new}, e))$ for all $e$ that are generated by or used by $a_{new}$. 
For e-grouping, the new events are $ev(\wgby(e_{new}, a))$ and  $ev(\used(e_{new},a))$ for any $a$ that has generated (resp. used) $e_{new}$.
%
We are going to refer to the two sets of events in $G$ and $G'$ as $EV_{G}$, $EV_{G'}$, respectively.

Note: we have been using symbols like $g_{41}$ in the figure to indicate relationships like $wgby(e_4, a_1)$. 
With slight abuse of notation, but in the interest of simplicity, in the following we are going to use $g_{41}$ to also denote $ev(\wgby(e_4, a_1))$ when it is clear from the context that we refer to the event rather than to the relationship itself.

To fix ideas, consider $G$ in Fig.~\ref{fig:e4-e5}(a), where two sections of a document are independently generated by two editing activities, and then they are independently used by four more activities. Note that this is a slight extension of the abstract pattern of Fig.~\ref{fig:e2-a4}, where the document sections are $e_4$, $e_5$.
%
The e-grouping set $V_{gr} = \{ e_4, e_5\}$ represents the whole document. 

Let $G'$ be the result of (non-strict) e-grouping, as depicted in Fig.~\ref{fig:e4-e5}(b), where the abstract generation and usage events are given new names, namely $g_{Ni}$ as a shorthand for $\wgby(e_N,a_i)$, and $u_{Ni_j}$ for each usage $j$ of the form $\used(a_i, e_N)$. 
Thus, $EV_G = \{ g_{41}, g_{53}, u_{42}, u_{52}, u_{54} \}$ and $EV_{G'} = \{ g_{N1}, g_{N3}, u_{N2_1}, u_{N2_2}, u_{N_4} \}$.
%
If $G$ is valid, the following must hold (constraint C3):
\begin{align}
\label{eq:c3-G}
g_{41} \preorder u_{42}, \quad g_{53} \preorder u_{52}, \quad g_{53} \preorder u_{54} 
\end{align}
where $\preorder$ is the preorder relationship introduced in Sec. ~\ref{sec:prov-events}.

Similarly, for $G'$ to be valid we must have:
\begin{align}
g_{N1} \preorder u_{N2_1}, \quad g_{N1} \preorder u_{N2_2}, \quad g_{N1} \preorder u_{N_4}  \label{eq:c3-global1} \\  
g_{N3} \preorder u_{N2_1}, \quad g_{N3} \preorder u_{N2_2}, \quad g_{N3} \preorder u_{N_4} \label{eq:c3-global2} 
\end{align}

Recall that the PROV-DM recommendation document~\citep{w3c-prov-dm}, defines generation and usage events as follows:
%\begin{description}
\begin{itemize}
	\item\textbf{Generation} \textit{is the completion of production of a new entity by an activity} (Sec. 5.1.3)
	\item\textbf{Usage} \textit{is the beginning of utilizing an entity by an activity} (Sec. 5.1.4)
%	\item\textbf{Invalidation} \textit{is the start of the destruction, cessation, or expiry of an existing entity by an activity. The entity is no longer available for use (or further invalidation) after invalidation. Any generation or usage of an entity precedes its invalidation.} (Sec. 5.1.8)
\end{itemize}

Thus, $e_{N}$ is generated when both generation events $g_{N1}, g_{N3}$ have occurred (incidentally, this implies that these abstract events must be simultaneous), and it start being used when the ``earliest'' of the usage events takes place, keeping in mind that no ordering relationships amongst the usage events is necessarily defined.

Intuitively, it should be possible to map abstract events to corresponding  original events in $G$, in such a way that validity of $G'$ follows from the validity of $G$.
%
To formalise this idea, we propose to define the  events in $EV_{G'}$ in terms of events in $EV_{G}$, that is, by means of a function $\psi$ that maps each $ev' \in EV_{G'}$ to a corresponding event in $EV_{G}$:
\begin{align}
 \psi: EV_{G'} \rightarrow EV_{G} 
\end{align}
Furthermore, we want $\psi$ to be order-preserving, so that the validity of $G'$ relative to temporal constraints can be derived from the validity of $G$ using $\psi$: 
\begin{align}
 ev_1' \preorder ev_2' \Rightarrow \psi(ev_1') \preorder \psi(ev_2')  
\end{align}
where the same preorder $\preorder$ is used for both sets (because that is defined by the constraints).

We now look for a suitable mapping $\psi$. Consider for instance:
\begin{equation}
\psi(g_{N1}) = g_{41} ,  \quad  \psi(g_{N3}) =  g_{53},  \quad   \psi(u_{N2_1}) = u_{42}  \quad \psi(u_{N2_2}) = u_{52} \quad \psi(u_{N4}) = u_{54} \label{eq:psi-zero}
\end{equation}
This is a ``natural'' mapping, which follows the mapping between abstract and original relationships induced by the grouping operator.
Since $\psi$ is order-preserving, from (\ref{eq:psi-zero}) and (\ref{eq:c3-global2}), it follows that:
\begin{equation}
  g_{41} \preorder   u_{52}, \quad  g_{53} \preorder   u_{42}
  \label{eq:over-constraints}
  \end{equation}
However, we observe that neither of these relationships hold on $EV_G$, in fact the ordering of $g_{41}$ (resp $g_{53}$) relative to $u_{52}$ (resp $u_{42}$) is undefined.
Thus, if we consider all the possible total orderings in $EV_G$ that are consistent with $\preorder$, we see that this choice of mapping function restricts the possible interpretations of $EV_{G'}$, when we assume that $G'$ is valid, to only a subset of these orderings, namely those where the additional relationships (\ref{eq:over-constraints}) also hold.

We use this example to motivate a broader definition of $\psi$ that does not incur such restriction. For this, we first redefine $\psi$ to map each event in $EV_{G'}$ to a \textit{set} of events in $EV_{G}$:
\begin{align}
 \psi': EV_{G'} \rightarrow {\cal P}(EV_{G}) 
\end{align}
(where ${\cal P}(EV_{G})$ denotes a powerset).
%
Then, we introduce a new preorder $\preorderprime$ on ${\cal P}(EV_{G})$, such that:
\begin{align}
 S_1 \preorderprime S_2 \text{ iff for each } s_1 \in S_1, \exists s_2 \in S_2 \text{ such that } s_1 \preorder s_2 
 \end{align}
We can now define mapping $\psi'$ that is order-preserving:
\begin{align} 
ev_1' \preorder ev_2' \Rightarrow \psi(ev_1') \preorderprime \psi(ev_2')  
\end{align}
as follows:
\begin{equation}
\psi(g_{N1}) = \{ g_{41}, g_{53} \},  \quad  \psi(u_{N2_1}) =  \psi(u_{N2_4}) = \{ u_{42}, u_{52} \}, \quad \psi(u_{N4}) = \{u_{54} \} \label{eq:psi-real}
\end{equation}
It is easy to see for example that, with this mapping, the constraints 
$g_{N1} \preorder u_{N2_1}$, $g_{N3} \preorder u_{N2_2}$ both imply $ \{ g_{41}, g_{53} \} \preorderprime  \{ u_{42}, u_{52} \} $,
which includes interpretations on $G$ where the new constraints (\ref{eq:over-constraints}) need not hold.
The specification of $\psi'$ follows easily from the computation of the grouping operator, and details are omitted.

In summary, we have proposed to introduce (i) a set of abstract events on $G'$ so we can verify its validity relative to temporal constraints, (ii) a mapping function that defines abstract events in terms of original events in $G$, and (iii) a set-based preorder such that the mapping can be order-preserving without imposing additional constraints on $G$. 


%%OBSOLETE. retrieve from git is we change our minds.
%\subsection{Mappings for abstract events }
%
%The following questions provide a useful starting point for reasoning about events. Firstly, given the generation events $g_{41}$, $g_{53}$  of each of the document sections, when was the entire document $e_N$ generated? 
%%
%When did $a_2, a_4$ use the document? 
%%
%Secondly,  suppose after the first grouping one performs two additional a-groupings, first with $V_{gr} = \{a_1, a_3\}$, and then with $V_{gr} = \{a_2, a_4, a_5, a_6\}$.
%%
% This results in the abstraction depicted in Fig.~\ref{fig:e4-e5}(c), which reads simply ``(abstract) document $e_N$ was used by (abstract) activity $a_N$''. 
% In this abstraction, what happens to the original generation and usage events?
%
%Initial help in answering these questions comes from the PROV-DM recommendation document~\citep{w3c-prov-dm}, namely:
%%\begin{description}
%\begin{itemize}
%\item\textbf{Generation} \textit{is the completion of production of a new entity by an activity} (Sec. 5.1.3)
%\item\textbf{Usage} \textit{is the beginning of utilizing an entity by an activity} (Sec. 5.1.4)
%\item\textbf{Invalidation} \textit{is the start of the destruction, cessation, or expiry of an existing entity by an activity. The entity is no longer available for use (or further invalidation) after invalidation. Any generation or usage of an entity precedes its invalidation.} (Sec. 5.1.8)
%\end{itemize}
%%\end{description}
%
%%
%Let us consider these definitions in the context of our example. Firstly, the generation of the whole document is only complete upon generation of the last section. Thus, each of the generation events of $e_N$, denoted $g_{N1}$ and $g_{N3}$ in Fig.~\ref{fig:e4-e5}(b), cannot precede $g_{41}$, $g_{53}$. This can be written as the ordering constraints:
%\begin{align}
% \label{eq:gprime-order1}
%max\{g_{41}, g_{53}\} \leq g_{N1} \\
%max\{g_{41}, g_{53}\}  \leq g_{N3}
%\end{align}
%Furthermore, we know from constraint C2 that $g_{N1}$ and $g_{N3}$ must be simultaneous:
%$g_{N1} = g_{N3}$.
%
%
%Secondly, symmetrically to generation, usage of the document ($u_{N2_1}, u_{N2_2}, u_{N4}$) in Fig.~\ref{fig:e4-e5}(b) begins with the earliest usage by any of the consuming activities:
%\begin{align}
%u_{N2_1} \leq u_{42} \label{eq:gen-usage1}\\
%u_{N2_2} \leq u_{52}  \label{eq:gen-usage2}\\
%u_{N4} \leq u_{54}  \label{eq:gen-usage3}
%\end{align}
%%
%Finally, from the definition above, the rule for invalidation follows the same pattern as usage:
%%
%\begin{align}
%i_{N6} &\leq i_{46} \label{eq:inv1}\\
%i_{N5} &\leq i_{55}  \label{eq:inv2}
%\end{align}
%%
%Furthermore, C3 requires each generation to precede each usage:
%%
%\begin{align}
%g_{N1} &= g_{N13}  \leq  u_{N2_1}  \quad  \\
%g_{N1} &= g_{N13}  \leq  u_{N2_2}  \quad  \\
%g_{N1} &= g_{N13}  \leq  u_{N4}  \label{eq:gprime-order2}
%\end{align}
%%
%and C4, C5 require both generation and usage to precede invalidation:
%%
%\begin{align}
%g_{N1} &= g_{N3} \leq i_{N6}  \quad g_{N1} = g_{N3} \leq i_{N5} \\
%u_{N2_1} &\leq i_{N6}  \quad u_{N2_1} \leq i_{N5} \\
%u_{N2_2} &\leq i_{N6}  \quad u_{N2_2} \leq i_{N5} \label{eq:inv-last}
%\end{align}
%%
%In order to avoid excessive clutter in the example, start and generation constraints are only discussed in the next section, along with all general ordering constraints on $G'$.
%
%
%Now, consider the  linear orderings in $G$ under the assumption that \textit{every generation event precedes every usage event} and \textit{every usage event precedes every invalidation event}, that is, there is a ``generation phase'' followed by a ``usage phase'' and by an ``invalidation phase'' (Fig.~\ref{fig:e-grouping-orderings}(a)). It is easy to see that, with this assumption, all these orderings are consistent with constraints (\ref{eq:gprime-order1}) through (\ref{eq:inv-last}), provided that we redefine $G'$ events to be \textit{simultaneous} to corresponding $G$ events, as follows:
%
%\begin{figure*}
%\centering
%\includegraphics[scale=.5]{figures/e-grouping-orderings.pdf} 
%\caption{Two possible orderings on $G$ (top), and corresponding orderings on $G'$ (bottom)} \label{fig:e-grouping-orderings}
%\end{figure*}
%
%
%\begin{align}
%g_{N1} = g_{N3} = max\{g_{41}, g_{53}\}   \label{eq:max} \\
%u_{N2_1} = u_{N2_2}  = u_{N4} = min\{u_{42},  u_{52},  u_{54}\}   \label{eq:usage-min-simple} \\
%i_{N5} = i_{N6}  = min \{   i_{46},  i_{55} \}   \label{eq:inval-min-simple}
%\end{align}
%In this case we conclude that $G'$ is valid by our assumption that each of its generation events precedes each of its usage events.
%
%Consider now the more general case where generation, usage and invalidation events are interleaved for different entities in $V_{gr}$. Fig.~\ref{fig:e-grouping-orderings}(b) shows such an interleaving for our example. In this case, 
%$min \{   u_{42},  u_{52},  u_{54} \} = u_{42} \leq g_{53} = max\{g_{41}, g_{53}\}$.
%%
%This violates (\ref{eq:gen-usage1}) through (\ref{eq:gen-usage3}).  In other words, this more general family of interpretations over $G$ is not represented in $G'$ when the abstract events in $G'$ are defined using the inequalities above. 
%
%In order to account for this general case, we modify  (\ref{eq:usage-min-simple}) and (\ref{eq:inval-min-simple}) as follows:
%\begin{align}
%u_{N2_1} &= u_{N2_2}  = u_{N4} = max\{ g_{41}, g_{53}, min \{u_{42},  u_{52},  u_{54}\}\}  \label{eq:usage-min} \\
%i_{N5} &= i_{N6}  = max\{g_{41}, g_{53}, u_{42},  u_{52},  u_{54}, min\{i_{46},  i_{55}\}\}   \label{eq:inval-min}
%\end{align}
%%
%
%In the example of Fig.~\ref{fig:e-grouping-orderings}(b), this stricter constraint results in  generation and usage events in $G'$ to all be simultaneous to $g_{53}$, while the invalidation events are shifted later in the event line, to the latest usage. Note that in the special case of Fig.~\ref{fig:e-grouping-orderings} (a), constraints (\ref{eq:usage-min-simple}), (\ref{eq:usage-min}) and (\ref{eq:inval-min-simple}), (\ref{eq:inval-min})  are pairwise equivalent.
%
%The reasoning used in the examples just presented justifies the following definitions for the general inequalities which define abstract events in terms of events in $G$.
%
%\subsection{Abstract events for e-grouping}
%\label{sec:abstract-events-for-e-grouping}
%%
%Let $G=(V,E) \in \guEA$, $V_{gr} \subset V$ be the set of nodes that are to be grouped, and  $e_{new} \in \en$ be the new entity node introduced through e-grouping as per Def.~\ref{eq:t-grouping}.
%% 
%
%\paragraph*{\textbf{Abstract Generation events}}
%Let $V^*$ to denote $\extend(\clos(V_{gr},G), \en)$, and let $\wgby_{out}$ denote the set of generation relations involving entity nodes in the extension
%$V^*$, and activity nodes outside of the extension:
%\begin{align*} 
%\wgby_{out} = \{ \wgby(e,a) |  e \in V^*,  a \notin V^* \}
%\end{align*} 
%In the example of Fig.~\ref{fig:e4-e5}, $\wgby_{out} = \{ g_{41}, g_{53} \}$.
%
%%
%Correspondingly, let $\wgby'_{out}$ denote the generation relations that involve $e_{new}$:
%\begin{align*} 
%\wgby'_{out} = \{ \wgby(e_{new},a) | a \notin V^* \}
%\end{align*} 
%In the example, $\wgby'_{out} = \{ g_{N1}, g_{N3} \}$.
%
%In general, we will denote values in the abstracted prov graph by primed versions of their counterparts in the original graph. The exception to this will be relationships and events involving $e_{new}$, since $e_{new}$ is  a new entity that does not appear in the old graph.
%
%The following equalities, define the orderings of the events associated with the relations in $\wgby'_{out}$.
%
%
%\vspace*{10pt}
%\begin{definition}[Abstract generation events - e-grouping]
%\label{def:abstract-gen-e}
%
%\paolotwo{Replace max with a new event that dominates all of the $ev(g)$ and adjust the def below accordingly}
%
%For each $g' \in \wgby'_{out}$:
%\[
%ev(g') = max \{ ev(g) | g \in \wgby_{out} \}  
%\]
%For all activities $a$ that participate in generating the new event $e_{new}$, we set the generation event to
%\[
%ev(\wgby(e_{new},a)) = max \{ ev(g) | g \in \wgby_{out} \}
%\]
%\end{definition}
%
%%\vspace*{10pt}
%%\begin{definition}[Abstract usage events - e-grouping] 
%%\label{def:abstract-usage-e}
%%Let
%%\[u'_{min} = min_{ u \in \used_{in} } ( ev(u) \}\] and let 
%%$g'_{max} = max_{g \in \wgby_{out}} ( ev(g) \} $.\\
%%For each $u' \in \used'_{in}$:
%%\begin{equation}
%%v(u') = max \{ g'_{max} , u'_{min} \}
%%\end{equation}
%%\end{definition}
%
%\paragraph*{\textbf{Abstract Usage events}}
%Usage events for e-grouping are defined similarly by generalization from (\ref{eq:usage-min}), as follows.
%%
%Let $\used_{in}$ denote the set of usage relations involving entity nodes in the extension
%$V^*$, and activity nodes outside of the extension:
%\begin{align*} 
%\used_{in} = \{ \used(a,e) |  e \in V^*,  a \notin V^* \}
%\end{align*} 
%In the example of Fig.~\ref{fig:e4-e5}, $\used_{in} = \{ u_{42}, u_{52}, u_{54} \}$.
%
%%
%Correspondingly, let 
%$\used'_{in}$ denote the usage relations that involve $e_{new}$:
%\begin{align*} 
%\used'_{in} = \{ \used(a, e_{new}) | a \notin V^* \}
%\end{align*} 
%In the example, $\used'_{in} = \{ u_{N2_1}, u_{N2_2}, u_{N4}  \}$.
%
%The following equalities, which generalise (15), define the events associated with the relations in $\used'_{in}$.
%
%\vspace*{10pt}
%\begin{definition}[Abstract usage events - e-grouping] 
%\label{def:abstract-usage-e}
%
%
%\paolotwo{min does not always exist so as for max, we need to introduce a new $u$ that is $\leq$ than all of the usage events, and adjust the def below accordingly}
%Let
%\[u'_{min} = min \{ ev(u) | u \in \used_{in} \}\] and let 
%$g'_{max} = max \{ ev(g) | g \in \wgby_{out}\} $.\\
%For each $u' \in \used'_{in}$:
%\begin{equation*}
%ev(u') = max \{ g'_{max} , u'_{min} \}
%\end{equation*}
%and so
%\begin{equation}
%ev(\used(a,e_{new})) = max \{ g'_{max} , u'_{min} \}
%\end{equation}
%\end{definition}
%
%\paolotwo{remove invalidation altogether}
%\paragraph*{\textbf{Abstract Invalidation events}}
%The events equalities for invalidation, exemplified in (\ref{eq:inval-min}), follow the pattern used above for usage. The only difference is that $\used'_{in}$ is replaced by 
%\[ \inv'_{in} = \{ \inv(a, e_{new}) | a \notin V^* \} \]
%In our example, $ \inv_{in} = (  i_{46}, i_{55} \}$,  $\inv'_{in} = ( i_{N6}, i_{N5} \}$.
%%
%The corresponding definition is as follows.
%
%\vspace*{10pt}
%\begin{definition}[Abstract invalidation events] 
%\label{def:abstract-inv}
%Let
%\[i'_{min} = min \{ ev(i) | i \in \inv_{in} \}\]
%and 
%\[u'_{max} = max \{ ev(u) | u \in \used_{in} \}\]
%For each $i' \in \inv'_{in}$:
%\[
%ev(i') = max \{ g'_{max} , u'_{max},  i'_{min}\}
%\]
%and thus for each activity $a$  that participates in the invalidation of $e_{new}$, invalidation is the latest of the new generation event $g'_{max}$, the latest usage event $u'_{max}$ and the minimum of the original invalidation events.  
%\begin{equation}
%\inv(a,e_{new}) = max \{ g'_{max} , u'_{max},  i'_{min} \}
%\end{equation}
%\end{definition}
%
%
%{\bf Start events:} Now that the event of usage and generation of $e_{new}$ has been fixed, we need to ensure that the activities involved in these events continue to meet the constraints that apply to them. This includes start and end events for activities related to our new entity $e_{new}$ by either $\wgby$ or $\used$.  
%
%
%%In addition, a-grouping also produces new abstract start and end events for the new activity $a_{new}$. 
%% The situation is illustrated in the timelines of Fig.~\ref{fig:a1-a2}. The right side of figure (a) shows a possible interleaving of events in $G$, which is consistent with constraints C2-C7. Intuitively, the start (resp. end) event for the abstract activity $a_N$ cannot follow (resp. precede) the earliest (resp., latest) usage/generation event associated with $a_1, a_2$.
%% %
%% Similar to the case for e-grouping,
%We appeal to the informal definitions of start and end in~\citep{w3c-prov-dm} to derive inequalities for the abstract start and end events for our new activity $a_{new}$.
%% 
%\begin{itemize}
%\item \textbf{Start} \textit{is when an activity is deemed to have been started by an entity, known as trigger. The activity did not exist before its start. Any usage, generation, or invalidation involving an activity follows the activity's start} (See~\citep{w3c-prov-dm},  Section 5.1.6)
%
%\item \textbf{End} \textit{is when an activity is deemed to have been ended by an entity, known as trigger. The activity no longer exists after its end. Any usage, generation, or invalidation involving an activity precedes the activity's end} (See~\citep{w3c-prov-dm}, Section 5.1.7)
%\end{itemize}
%(For simplicity we are going to leave the trigger entity implicit, and simply refer to the start and end events as $\start(a)$ and $\ed(a)$).
%
%\paolotwo{do not think we need (27) because it is implied by the mapping for the abstract generation events. probably (28) not needed either }
%\begin{definition}[Start events - e-grouping] 
%\label{def:abstract-start-e}
%Consider events $a$ such that $\wgby(e_{new},a)$. For each such $a$, we set the new start event $\start'(a)$ as the lesser of the original start event and the generation event $ev(\wgby(e_{new},a))$. Thus
%\begin{equation}
%\start'(a) = min\{\start(a),ev(\wgby(e_{new},a))\}
%\end{equation}
%\end{definition}
%
%{\bf End events:} For all activities $a$ that use the newly created entity, we must ensure that ensure that the end of the activity does not precede any $\used(a,e_{new})$ events.
%\begin{definition}[End events - e-grouping]
%  \label{def:abstract-end-e}
%  If the set of all usage events by $a$ of $e_{new}$ is denoted $\{ev(\used(a,e_{new}))\}$, we set the new end events $\ed'(a)$ to be
%  \begin{equation}
%  \ed'(a) = max\{\ed(a), max\{ev(\used(a,e_{new}))\}\}
%\end{equation}
%\end{definition}
%
%A proof of that, given the definitions above, the  constraints of Section~\ref{sec:prov-constraints} are satisfiable, is given in~\ref{sec:consistency-constraints-e-grouping}.
%
%
%\subsection{Abstract events for a-grouping}
%\label{sec:abstract-events-for-a-grouping}
%% \begin{figure}
%% \centering
%% \includegraphics[scale=.5]{figures/a1-a2.pdf} 
%% \caption{Abstraction for start/end events} \label{fig:a1-a2}
%% \end{figure}
%
%Generation and usage abstract events follow a very similar pattern as those for e-grouping, except that the new node introduced by grouping is an activity node: $a_{new} \in \act$.
%%
%%This is illustrated in Fig.~\ref{fig:a1-a2}.
%As a consequence, the abstract event definitions given in the previous section also follow the same pattern, but with the roles of entities and activities reversed. They are summarized here below.  We now use $V^*$ to mean the group of nodes collected by $\extend(\clos(V_{gr},G), \act)$.
%
%%\jwb{I (personally) find the notations below confusing, and would rather get rid of them. I haven't used them much in the rest of this section.}
%%
%%\begin{align*} 
%%\wgby_{in} & =  \{ \wgby(e,a) |  e \notin V^*, a \in V^* \} \\
%%\wgby'_{in} &  = \{ \wgby(e, a_{new}) | e \notin V^* \} \\
%%\used_{out} & = \{ \used(a,e) |  e \notin V^*,  a \in V^* \} \\
%%\used'_{out} & = \{ \used(a_{new}, e) | e \notin V^* \}
%%\end{align*} 
%%
%%
%%
%
%\paolotwo{def 14 makes no sense as we again need to introduce artificial min and max to account for the abstract start and end events}
%The new start (resp. end) event is taken to be the minimum (resp. maximum) relevant start (resp. end)  event.
%\begin{definition}[Abstract start and end events - a-grouping] 
%\label{def:abstract-start-and-end-a}
%\begin{align*}
%  \start(a_{new}) & = min\{\start(a) | a \in V^*\} \\
%  \ed(a_{new}) & = max\{\ed(a) | a \in V^*\} \\
%\end{align*}
%\end{definition}
%
%\begin{definition}[Abstract generation events - a-grouping]
%\label{def:abstract-gen-a}
%%Since generation events must be simultaneous, we can assume that they are simultaneous in the original graph. 
%%
%%For any entity $e$ not in $V^*$ that is generated by an activity $a$ in $V^*$, the new generation event cannot be before the start of the abstracted activity $a_{new}$, taken as the minimum of the original start events s in Definition~\ref{def:abstract-start-and-end-a}. However,
%Constraint C2 applies to the original graph, so for all activities $a,b$ that participate in the generation of $e$, $ev(\wgby(e,a)) = ev(\wgby(e,b))$. 
%
%
%%\jwb{Should we do start and end definitions first? Start and end only apply to ctivities, so the definition below already excludes entities}
%The new generation event is thus given as
%%  \begin{align*}
%%   ev(\wgby(e,a_{new})) =  max\{ &  \start(a_{new}), \\ 
%%                                & min\{ev(\wgby(e,a))  | a \in V^*\} \}
%% \end{align*}
%  \begin{align*}
%   ev(\wgby(e,a_{new})) =  ev(\wgby(e,a))
% \end{align*}
%
%\end{definition}
%
%\vspace*{10pt}
%\begin{definition}[Abstract usage events - a-grouping] 
%\label{def:abstract-usage-a}
%For an entity $e$ not in $V^*$ which is used by an activity $a$ in $V^*$, the assigned usage event for $a_{new}$ ($ev(\used(a_{new},e))$) is given by  
%\begin{align*}
%ev(\used(a_{new},e)) = max\{ & \start(a_{new}) , \\
%                            & min\{ev(\used(a,e))) | a \in V^*\}\}
%\end{align*}
%\end{definition}
%
%
%\vspace*{10pt}
%\begin{definition}[Abstract invalidation events - a-grouping] 
%  \label{def:abstract-inv-a}
%  By Constraint C8, invalidation events from two distinct activities are simultaneous.
% %
%For an entity $e$ which is invalidated by an activity $a$ in $V^*$, the event of the invalidation event remains the same when the activity is abstracted.
%\[
%ev(\inv(a_{new},e)) = min\{ev(\inv(a,e)) | a \in V^*\}
%\]
%\end{definition}
%
%A proof of that, given the definitions above, the  constraints of Section~\ref{sec:prov-constraints} are satisfiable, is given in~\ref{sec:consistency-constraints-e-grouping}.




%%%-*- mode: LaTeX; mode: FlySpell; -*-

\section{Abstraction with Agents}  \label{sec:agents-abstraction}

\comment{outer border should be ``boundary'' in this section. }

Having laid the foundations for abstraction on the core $\guEA$ model, extending grouping to a model that also includes agents, the third pillar of the PROV model, is quite straightforward. Agents may be humans or software systems. 
%
Specifically, we now consider the node type $\ag$ and the following additional relation types from the PROV schema of Sec.~\ref{sec:prov-core}:

\begin{eqnarray*}
\waw      & \subseteq & \act \times \ag \\
\attrTo   & \subseteq & \en \times \ag \\
\delegate & \subseteq & \ag \times \ag  
\end{eqnarray*}
% \begin{align*}
% \waw \subseteq \act \times \ag \\
% \attrTo \subseteq \en \times \ag \\
% \delegate \subseteq \ag \times \ag  
% \end{align*}
% %Instances of this extended model now include the three relation types above.
%$\{\attrTo(e,ag)|e \in \en, ag \in \ag\} \cup \{\waw(a,ag)|a \in \act, ag \in \ag\} \cup \{ \delegate(ag_1, ag_2) | ag_1, ag_2 \in \ag\}$. 
%

We use the shorthand relation names $\waw$,  $\attrTo$ and $\delegate$  for \textit{wasAssociatedWith},  \textit{wasAttributedTo} and \textit{actedOnBehalfOf}. These denote responsibility of an agent for an activity ($\waw$), responsibility of an agent for an entity ($\attrTo$), and delegation between two agents ($\delegate$). 
%

Note that PROV admits an additional optional activity element to $\delegate$, which is used to  qualify the delegation as occurring within the scope of that activity. For simplicity, we are not going to consider this qualified version of the relation.
%
Thus, we can still assume that these new relations are binary, and so we continue to view an instance of a provenance graph as a directed graph $G=(V,E)$, where new $V= \en ~\cup~ \act ~\cup~ \ag$, and where each relation instance maps to a labelled directed edge.  We denote the set of all such graphs as $\guaEAG$.

The main implications of adding agents to our abstraction model are that (i) a new \textit{ag-grouping} operator must be introduced, and (ii) the existing definitions of e-grouping and a-grouping must be modified slightly. 

In order to incorporate agents into the definition of $\group$, observe that, for all three relations involving agents, the agent node is always the \textit{target} of the directed edge.
%
This means that agents can be viewed as part of an ``outer layer'' in the provenance graph.
%
This is illustrated in Fig.~\ref{fig:agents-baseline}, where the agent nodes and new relations are shown in bold lines in the outer side of the digraph.

\begin{figure}
\centering
\includegraphics[scale=.5]{figures/agents-baseline}
\caption{$\guaEAG$ provenance graph. Agents lie on the outer border of the digraph. Shaded nodes show a possible grouping set in the general case.}  \label{fig:agents-baseline}
\end{figure}

This observation suggests we can break down the analysis of grouping with agents into the following three parts.
%
\begin{enumerate}
\item $V_{gr} \subset \ag$. This is the case for ag-grouping, which only involves the outer layer of the graph. Since agents are only related to each other through delegation: $\delegate(ag_1 , ag_2)$, grouping in this case is akin to \textit{homogeneous grouping} from Sec.~\ref{sec:closure}, \jwb{ in the sense that} no nodes of other types are ever involved, and $v_{new} \in \ag$. %\comment{this is NOT what type-homogeneity means. Nodes of other types ARE (or at least can) involved there, because of $\pclos$. }

\item $V_{gr} \subset \en \cup \act$ as in Sec.~\ref{sec:grouping}. This is the case of t-grouping (Def.~\ref{def:t-grouping}), where the nodes involved in the abstraction are in the inner layer, but they may be related to agent nodes via $\waw$ and $\wat$ relations.

\item $V_{gr} \subset \en \cup \act \cup \ag$.  Here, the group set may contain any combination of nodes. However, the peripheral role played by agents relative to entities and activities suggests that it may be reasonable to restrict this case to e-grouping or a-grouping, i.e., a combination of node types may include agents, but it should not be abstracted by a new agent node.
%
\end{enumerate}



\subsection{Ag-grouping: abstracting agents}  \label{sec:ag-grouping}

We begin with the case where abstraction is performed over a set of agents, i.e., $V_{gr} \subset \ag$.
%
In this case, the existing definition of t-grouping (Def.~\ref{eq:t-grouping}) extends naturally to $\guaEAG$ graphs.
%
%
T-grouping involves three operators: $\clos$, $\extend$, and $\repl$.
%
Since $\clos(V_{gr}, \jwb{G})$ operates only on $\delegate$ relations, it follows that its result is also homogeneous, i.e., $\clos(V_{gr}, \jwb{G}) \subset \ag$. Also, there is no need to restore type validity by extending the closure, i.e., $\extend$ is the identity: $\extend(\clos(V_{gr},  \jwb{G}), V, \ag) = \clos(V_{gr},  \jwb{G})$.
%
Finally, it is easy to see that our original definition of group replace (Def.~\ref{def:group-replace}) is general enough to accommodate the ``rewiring'' of the new abstract agent node. 
%
We illustrate this informally using the patterns of Fig.~\ref{fig:d-chains-abstracted}.
%

In pattern (a), $V_{gr} = \{ ag_1, ag_4 \}$, and $\clos(V_{gr},  \jwb{G}) =  \{ ag_1, ag_2, ag_3, ag_4 \}$.
%
In this case, the closure includes all the intermediate agents in the delegation chain between $ag_1$ and $ag_4$.
%
Replacement trivially transforms $G$ into the single abstract agent $ag_N$.


\begin{figure}
\centering
\includegraphics[scale=.5]{figures/d-chains-abstracted}
\caption{ag-grouping involving delegation.}
\label{fig:d-chains-abstracted}
\end{figure}

In pattern (b), $V_{gr} = \{ ag_2, ag_5, ag_4 \}$.
%
Note that not all agent nodes in $V_{gr}$ are related, either directly or through a path. This is not a problem, as we have $V_{clos} = \clos(V_{gr},  \jwb{G}) =   \{ ag_2, ag_5, ag_4, ag_3 \}$. Replacement applies as follows, where $\vartheta_{int}(V_{clos})$ is the set of relations beginning and ending inside $V_{clos}$:
\begin{eqnarray*}
\vartheta_{int}(V_{clos}) & = & \{ \delegate(ag_2, ag_3), \delegate(ag_3, ag_5) \} \\
\vartheta_{in}(V_{clos}) & = & \{ \wat(e, ag_2), \delegate(ag_1, ag_4) \} \\
\vartheta_{out}(V_{clos}) & = & \{ \delegate(ag_4, ag_6) \}
\end{eqnarray*}

% \begin{align*}
% &\vartheta_{int}(V_{clos}) = \{ \delegate(ag_2, ag_3), \delegate(ag_3, ag_5) \} \\
% &\vartheta_{in}(V_{clos}) = \{ \wat(e, ag_2), \delegate(ag_1, ag_4) \} \\
% &\vartheta_{out}(V_{clos}) = \{ \delegate(ag_4, ag_6) \}
% \end{align*}
% \

%
Thus, $\repl(V_{clos}, ag_N, G)$ maps relations in the original graph to those in the abstracted graph as follows:
%
\begin{eqnarray*}
\delegate(ag_4, ag_6) & \rightarrow &  \delegate(ag_N, ag_6) \\
\wat(e, ag_2) & \rightarrow &  \wat(e, ag_N) \\
\delegate(ag_1, ag_4) & \rightarrow &   \delegate(ag_1, ag_N)
\end{eqnarray*}

% \begin{align*}
% &\delegate(ag_4, ag_6) \rightarrow \delegate(ag_N, ag_6) \\
% &\wat(e, ag_2) \rightarrow \wat(e, ag_N) \\
% &\delegate(ag_1, ag_4) \rightarrow  \delegate(ag_1, ag_N)
% \end{align*}
% %
In practice, replacement preserves agents $ag_1$ and $ag_6$ and restores their delegation relations relative to the new abstract agent, $ag_N$. It also maps relation $\wat(e,ag_2)$, which involves the untouched $e$ node, to a new relation of the same type: $\wat(e,ag_N)$. 

We conclude that, in this first case, Def.~\ref{def:homo-group} applies without changes.

%\mnote{
%Points we must include somewhere
%\begin{itemize}
%  \item $V_{ag}$ means only type ag nodes included..
%  \item short forms of relations for general use
%  \item $\delegate(a,b) \in E$ means $(a,b) \in E \land label(a,b) = \delegate$
%  \item like-for-like: agents are abstracted to agents, not anything else.
%\end{itemize}
%}\\
%\mnote{
%  \begin{itemize}
%  \item close all delegate chains up, so we avoid cycles. ref sec on path closure.
%  \item remove and replace with $v_{new}$. 
%  \item rewire graph
%    \begin{itemize}
%    \item $\delegate$: agnets can't delagate to themselves \fbox{PM?}, otherwise retain delegate links
%    \item $\waw$ and $\wat$ remove all links: replace those that cross the abstract node boundary
%    \end{itemize}
%  \end{itemize}
%}
%

\subsection{a-grouping and e-grouping with agents}  \label{sec:t-grouping-agents}

The second case, where $V_{gr} \subset \en ~\cup~ \act$, is t-grouping with added agents relations.
%
Here the $\repl$ operator (Def.~\ref{def:group-replace}) must now consider how edges that involve agents are mapped to new edges in the abstract graph. 
%
We have already observed that agent nodes are always the targets of directed edges.
%
It follows that the closure of a set of nodes $V_{gr} \subset \en \cup \act$ never adds agent nodes to $V_{gr}$, because this would require the added agent to be on a path between two nodes from $\en ~\cup~ \act$, and therefore to be the source of a directed edge. 
%
%there can be no paths of the form $x \leftarrow ag \leftarrow y$  for $x,y \in \en \cup \act$.

%
A second observation is that if $\waw(a, ag)$ holds, and $a$ is involved in a-grouping, then $a$ is replaced by $a_{new}$, and thus $\waw(a_{new}, ag)$ also holds (Fig.~\ref{fig:agents-relations-patterns}(a1)).
%
Similarly for e-grouping, if $\wat(e, ag)$ holds, and $e$ is involved in e-grouping, then $e$ is replaced by $e_{new}$, and $\wat(e_{new}, ag)$ holds  (Fig.~\ref{fig:agents-relations-patterns}(b1)).
%
On the other hand, suppose $\waw(a,ag)$ holds and e-grouping is performed. A simple case is shown in Fig.~\ref{fig:agents-relations-patterns}(c1).
%
In this case, $a$ is replaced by $e_{new} \in \en$, therefore $\waw(e_{new},ag)$ is type-incorrect. Similarly, $\wat(e,ag)$ after a-grouping would become, incorrectly,  $\wat(a_{new}, ag)$.
%
These two patterns are summarised in Fig.~\ref{fig:agents-relations-patterns}(a2, b2 resp.). Note that one cannot simply replace association with attribution, i.e., replace relation $\waw(a, ag)$ with $\wat(e_N, ag)$, because there is no guarantee that any of the entities represented by the new $e_N$ had been attributed to $ag$ in the original graph. Similarly, one cannot replace $\wat(e, ag)$ with $\waw(a_N, ag)$.
%
Instead, in pattern (a) we simply remove the incorrect $\waw$ relations following e-grouping, and similarly, in pattern (b) we remove the incorrect $\wat$ relations following a-grouping.

\begin{figure}
\centering
\includegraphics[scale=.5]{figures/agents-relations-patterns}
\caption{$\waw$ and $\wat$ edges involving nodes in $\clos(V_{gr},G)$ may need to removed following e-grouping and a-grouping, respectively.}  \label{fig:agents-relations-patterns}
\end{figure}

%
These considerations suggest that the definition of the $\repl$ function for grouping needs to be adapted for the case where agents are involved. 
%
To understand why, recall from Sec.~\ref{sec:closure} that $\repl$ replaces a type-homogeneous set, computed by the $\extend$ function, with an abstract node of the same type. An extension of type $t$ augments the closure of a grouping set by adding all adjacent nodes of the same type to it. This ensures that replacing the nodes in the extension with an abstract node of the same type preserves the type correctness of the relations. 
%
However it should be clear from the example above (parts c1, c2 of the figure), that both e-grouping and a-grouping on the set  $V_{gr} = \{ a, e\}$  result in one of the two agent relations being incorrect. Intuitively, this is because the extension function fails to incorporate agents, leaving the agent relations exposed on the outcut of $V_{gr}$ (in fact, $\extend$ in this example does nothing at all).

The pattern in (c2) (or its symmetric, for a-grouping), can be obtained simply by ensuring that $\repl$ deletes the incorrect relations.
%
The following variation on Definition~\ref{def:eq:outcut} ensures these deletions are enforced.
\begin{align*}
\vartheta_{out}'(V_{gr}') = \{ & v \xleftarrow{t}  v_{new} |  v \xleftarrow{t} v' \in \vartheta_{out}(V_{gr}') ~\wedge \\
      ( & (t = \wat \wedge \type(v_{new}) = \en) ~\vee \\
        & (t = \waw \wedge \type(v_{new}) = \act)  ~\vee \\
        & (t \neq \wat \wedge t \neq \waw) )\} 
\end{align*}


%Agents may also be abstracted. Here, a naive approach raises again the problem of cycles in the graph.   To illustrate, suppose the agents $ag2$, $ag4$ and $ag5$ are to be abstracted from the two delegation chains in Figure~\ref{fig:d-chains-abstracted}. 
%
%
% 
%
%This leads to agent $ag3$ both delegating to and being a delegate of the abstract agent node $a''$, a situation which we disallow 
%
%\mnote{
%@PM there doesn't seem to be  a constraint that excludes this.  Should we disallow it?  Why/Why not?
%}
%
% 
%We introduce \emph{delegation chain closure}, which is akin to path closure and includes   agents which, although not part of the original request, must also be abstracted. 
%
%\begin{definition}[Delegate Chains]
%\label{def:del-chain}
%A \emph{delegate chain} is a directed chain of agents $a_1,a_2,\ldots,a_n$, such that $\forall i<n\spot \delegate(a_i,a_{i+1}) \in E$. 
%\end{definition}
%
%
%In Figure~\ref{fig:d-chains-abstracted}, agent $ag3$ must be included in the path.  The function $\dclos$, given in Definition~\ref{def:dclos}, is the result of restricting function $\clos$ (Definition~\ref{def:clos}) to operate only on agents and traverse only $\delegate$ links.
% 
%\begin{definition}[Delegate Chain Closure]
%\label{def:dclos}
%Let $G = (V,E) \in \guaEAG$ be a provenance graph, and let $V_{ag} \subset V$ be a set of agents.   
%For each pair  $a_i, a_j \in V_{ag}$ such that there is a delegate chain beginning at  $a_i$ and ending at $a_j$,  let $V_{ij} \subset V$ be the set of all nodes in the chain.
%The \emph{delegate chain closure} of $V_{gr}$ in $G$ is 
%\[\dclos(V_{ag}, G)  =  \bigcup_{v_i, v_j \in V_{ag}} V_{ij} \]
%\end{definition}
%
%
%For any set of agents $V_{ag}$ to be abstracted, $\dclos(V_{ag},G)$ returns the set of agents which need to be replaced by a new abstract agent.  Replacing these agents is again a similar task to the one carried out by the $\repl$ operator.
%
%
%\begin{definition}[Agents Replace]
%\label{def:agent-replace}
%\[ \agrepl(V_{ag}, ag_{abs}, G) = (V', E'), \mbox{ where: }\]
%\begin{align*}
%V' & = V \setminus V_{ag}  \cup \{ag_{abs}\} \\
%E' & = E \setminus & \\
%   & \{ \delegate(ag_{abs},ag)  | \\
%   & \quad  \exists ag' \in V_{ag} \spot ag \in V\hide V_{ag} \\
%   & \quad \land  \delegate(ag',ag) \in E \} \\
%   & \cup \{ \delegate(ag,ag_{abs})  | \\
%   & \quad \exists ag' \in V_{ag} \spot ag \in V\hide V_{ag} \land  \\
%   & \quad \delegate(ag,ag') \in E \} \\
%   & \cup \{ \waw(a,ag_{abs})| \\
%   & \quad \exists ag'\in V_{ag}\spot \waw(a,ag') \in E\}\\
%   & \cup \{ \wat(e,ag_{abs})| \\
%   & \quad \exists ag'\in V_{ag}\spot \wat(e,ag') \in E\}\\
%\end{align*}
%\end{definition}
%
%
%This definition replaces all relations between agents in the set $V_{ag}$ and agents in $V\hide V_{ag}$, and therefore we cannot create  orphan agents using $\agrepl$. 
%
%
%\begin{definition}[Grouping Agents]
%\label{def:ag-group-by}
%\begin{align*}
%\aggroup(G,V_{ag},ag_{abs}) = \agrepl(\dclos(V_{ag},G),ag_{abs},G) 
%\end{align*}
%
%\end{definition}


\subsection{The general case: grouping on any node type}

The third and more general case, where \mbox{$V_{gr} \subset \en ~\cup~ \act ~\cup~ \ag$}, presents one further  difficulty. Consider performing e-grouping on the pattern of Fig.~\ref{agents-baseline-obo-problem} (left), with \mbox{$V_{gr} = \{ e_4, ag_5 \}$}.
%
We have \mbox{$\clos(V_{gr}, \jwb{G}) = \{ e_4, ag_5, ag_3 \}$}, resulting in the abstraction on the right.
%
Clearly, the former $\delegate(ag_4, ag_3)$ relation should not be mapped in the final graph. 
%
This issue arises because there the extension function, which guarantees type consistency for e- and a-nodes, does not include agents. 
%

Once again we deal with the issue by changing the definition of $\repl$ to ensure that the incorrect relation is not mapped. 
%
Note that a change is required to $\vartheta_{in}'(V_{gr}')$ rather than $\vartheta_{out}'(V_{gr}')$ as in the previous case. The new definition is as follows.
\begin{align*}
\vartheta_{in}'(V_{gr}') = \{ & v_{new} \xleftarrow{t}  v |  v' \xleftarrow{t} v \in \vartheta_{in}(V_{gr}') \\
    & \wedge ( (t = \delegate \wedge \type(v_{new}) = \ag) \vee t \neq \delegate )\} 
\end{align*}
Thus, a delegation relation is mapped in the abstract graph only if the target abstract node is an agent.

\begin{figure}
\centering
\includegraphics[scale=.5]{figures/agents-baseline-obo-problem}
\caption{e-grouping leads to incorrect $\delegate$ relation when agents are part of a closure.}
\label{agents-baseline-obo-problem}
\end{figure}

With the new version of the $\repl$ function, introduced in the previous two sections, some of the relations in the original graph $G$ are not mapped to the abstracted graph $G'$. This makes it possible for some of the nodes in $G'$ to end up disconnected from the rest of the graph. 
%
As an example, the graph in Fig.~\ref{agents-baseline-abstracted} shows the result of e-grouping over the shaded nodes in the graph of Fig.~\ref{fig:agents-baseline}. The combination of closure, extensions and replacement results in $ag_5$ being isolated, or ``orphaned'' in the abstract graph. 

%
As isolated agent nodes may not be significant to consumers of the abstracted graphs, for completeness we provide a simple function to optionally remove them at the end of the abstraction process, as follows.

\begin{figure}
\centering
\includegraphics[scale=.5]{figures/agents-baseline-abstracted}
\caption{Abstracted version of the graph in Fig.~\ref{fig:agents-baseline} after e-grouping involving agents.}
\label{agents-baseline-abstracted}
\end{figure}


\begin{definition}[Removing isolated agents]
\label{def:orphanremove}
Let $G = (V,E) \in \guaEAG$, and 
\begin{align*}
 \mathit{isolated}(G) = \{  & v \in V | \type(v) = \ag \\
   & \wedge \nexists v' \in V . ((v',v) \in E \vee (v,v') \in E) \}
\end{align*}
Then:
\[ \remIsolated((V,E)) = (V \setminus \mathit{isolated}(G), E) \] 
\end{definition}




%\section{Tool Implementation}   \label{sec:summary}

Rather than giving a procedural pseudo-code for the $\group$ algorithm, we summarize its functional specification, comprising of the four functions defined so far: $\clos()$, $\extend()$, $\repl()$, and $\remIsolated()$, in Table~\ref{alg-summary}.

\begin{table*}
  \begin{eqnarray*}
    %%%%%%
    \clos(V_{gr}, V)  &%  
    = &%
     \bigcup_{v_i, v_j \in V_{gr}} V_{ij} \\
    %%%%%%
    \extend(V_{gr}, G ,t) &%
    = & V_{gr} 
      \begin{mydrop}
        ~\cup~ \{ v' | (v, v') \in E \wedge v \in V_{gr} \wedge \type(v') = t) \}  \\
        ~\cup~ \{ v | (v', v) \in E \wedge v \in V_{gr} \wedge \type(v') = t) \}  
      \end{mydrop} \\%
    %%%%%%
    \repl(V_{gr}, v_{new}, G) &% 
    = & %
    (V', E'), \mathrm{where}~~ 
       \begin{mydrop}
         \vartheta_{out}'(V_{gr}') = \{ v \xleftarrow{t}  v_{new} |  \begin{mydrop}
                 v \xleftarrow{t} v' \in \vartheta_{out}(V_{gr}')  \wedge \\%
                 ( (t = \wat \wedge \type(v_{new}) = \en) \vee \\%
                (t = \waw \wedge \type(v_{new}) = \act)  \vee \\%
                (t \neq \wat \wedge t \neq \waw) )\}  
              \end{mydrop} \\
         \vartheta_{in}'(V_{gr}') = \{  v_{new} \xleftarrow{t}  v | \begin{mydrop}
           v' \xleftarrow{t} v \in \vartheta_{in}(V_{gr}')  \wedge  \\%
           ( \begin{mydrop}
             (t = \delegate \wedge \type(v_{new}) = \ag) \\%
             \vee t \neq \delegate )\}
             \end{mydrop}
           \end{mydrop}
       \end{mydrop}\\
    %%%%%%
   V' & %
   = & %
   V  \setminus V_{gr}  \cup \{v_{new}\}  \\
    %%%%%%
   E' &% 
   = & % 
   E \setminus (\vartheta_{out}(V_{gr}) \cup \vartheta_{in}(V_{gr}) \cup \vartheta_{int}(V_{gr}))  \cup \vartheta_{out}'(V_{gr})  \cup \vartheta_{in}'(V_{gr}) \\
    %%%%%%
    \remIsolated(V,E) \mathrm{where} & %
    = &%
    (V', E') \\
    %%%%%%
    \mathit{isolated}(G) & 
    = & %
    \{  v \in V | \type(v) = \ag  \wedge \nexists v' \in V . ((v',v) \in E \vee (v,v') \in E) \} \\
    V' & %
    = & %
    V \setminus \mathit{isolated}(G)\\
    %%%%%%
    E' &
    =&
    E \\
    %%%%%%
    \group(G, V_{gr}, v_{new}, t) &%
    = & %
    \remIsolated(\repl( \extend(\clos(V_{gr},V), V, t), v_{new},  G ))
    %%%%%%
  \end{eqnarray*}
  \caption{Functional summary of the grouping algorithm.}
  \label{alg-summary}
\end{table*}



% \begin{table*}
% \begin{tabularx}{\textwidth}{lXlX}
% %%%%%%
% $\clos(V_{gr}, V) $ &  
% %%%%%%
% $ \bigcup_{v_i, v_j \in V_{gr}} V_{ij}$ \\
% %%%%%%
% $\extend(V_{gr}, G ,t) = V_{gr}:$ &  
% %%%%%%
% $\cup \{ v' | (v, v') \in E \wedge v \in V_{gr} \wedge \type(v') = t) \}  
% \cup \{ v | (v', v) \in E \wedge v \in V_{gr} \wedge \type(v') = t) \} $ \\
% %%%%%%
% $\repl(V_{gr}, v_{new}, G) = (V', E'):$ &  
% $\vartheta_{out}'(V_{gr}') = \{ v \xleftarrow{t}  v_{new} |  v \xleftarrow{t} v' \in \vartheta_{out}(V_{gr}')  \wedge ( (t = \wat \wedge \type(v_{new}) = \en) \vee 
% (t = \waw \wedge \type(v_{new}) = \act)  \vee (t \neq \wat \wedge t \neq \waw) )\} $ \newline
% $\vartheta_{in}'(V_{gr}') = \{  v_{new} \xleftarrow{t}  v |  v' \xleftarrow{t} v \in \vartheta_{in}(V_{gr}')  \wedge ( (t = \delegate \wedge \type(v_{new}) = \ag) \vee t \neq \delegate )\} $ \newline
% $V' = V  \setminus V_{gr}  \cup \{v_{new}\} $ \newline
% $ E' = E \setminus (\vartheta_{out}(V_{gr}) \cup \vartheta_{in}(V_{gr}) \cup \vartheta_{int}(V_{gr}))  \cup \vartheta_{out}'(V_{gr})  \cup \vartheta_{in}'(V_{gr}) $\\
% %%%%%%
% %%%%%%
% $\remIsolated(V,E) = (V', E')$ & 
% $\mathit{isolated}(G) = \{  v \in V | \type(v) = \ag  \wedge \nexists v' \in V . ((v',v) \in E \vee (v,v') \in E) \}$ \newline
% $V ' = V \setminus \mathit{isolated}(G), E' =  E$ \\
% %%%%%%
% $\group(G, V_{gr}, v_{new}, t):$&
% $\remIsolated(\repl( \extend(\clos(V_{gr},V), V, t), v_{new},  G ))$
% \end{tabularx} 
% \caption{Functional summary of the grouping algorithm.}
% \label{alg-summary}
% \end{table*}
% 
 
A procedural implementation of the algorithm, written in Java, is available online (\url{https://github.com/PaoloMissier/ProvAbs}) as part of a tool in which a policy language is used to drive the selection of the set $V_{gr}$ of nodes to be abstracted. Using the tool, policy setters may experiment with abstraction policies on their provenance graphs, observing their effects when abstraction is performed.
%
Noting that $\group()$ is closed wrt composition, the tool also allows for more complex abstraction, by letting users specify compositions of $\group$ operations as part of their policy.

%
Here we only provide a brief description of the tool. A more detailed account can be found in~\citep{Missier2013c} (submitted).
%
The tool operates on provenance graphs written in the provenance notation PROV-N~\citep{w3c-prov-n}.

%
Given $G \in \guaEAG$, the tool lets users specify a grouping set $V_{gr}$ by means of an \textit{abstraction policy}. It operates in two steps. Firstly, path expressions and predicates are used to select a set of nodes, and to assign a numerical \textit{sensitivity} value to them. 
%
For example, the following rule contains a path expression that binds variables \texttt{process} and \texttt{data} to activity and entity nodes $a$, $e$, respectively, such that $\used(a,e)$ holds and $e$ is any node that is reachable from node with id \texttt{d14}:

\vspace{0.5\baselineskip}

\small
\textsf{for all (process used data)}

\textsf{ \;\;\; where (data descendantOf d14)) }

\textsf{ \;\;\;\;\;\; setSensitivity(data, 10)}

\vspace{0.5\baselineskip}

\normalsize

The sensitivity value of the selected data nodes is set to 10.
%
Here \textit{descendantOf} is a built-in query predicate that returns all nodes reachable from a given start node. 

%
As an another example, the  predicate

\vspace{0.5\baselineskip}

\small
\textsf{for all (data wgb process)} 

 \textsf{ \;\;\; where (process.classification $>$ ``conf'' in classifications) }

\textsf{ \;\;\;\;\;\; setSensitivity(data, 9)}

\vspace{0.5\baselineskip}

\normalsize


binds variables \texttt{data} and \texttt{process} to pairs of nodes $d$, $p$ such that $\wgby(d,p)$ and where the classification value $p.classification$ of $p$ (a property of the node) is greater than ``conf''. This requires an ordered set of classification labels to be declared in a user-defined \texttt{classifications} list.
%
During the first step, the tool evaluates the policy rules, resulting in the annotation of the selected nodes with sensitivity values. 

%
This model operates on the same principle as the Bell-LaPadula security model. Each known receiver of a graph is pre-assigned a clearance level. This is determined by factors outside the scope of the tool, e.g. how much a receiver is trusted to the provenance owner. 
%
In the second step, node sensitivities are compared to the clearance level $cl$ of the receiver, and $V_{gr}$ is defined as the set of nodes whose sensitivity is lower than $cl$.
%

The rationale for these two steps is that sensitivity can be defined largely independently of the specific receiver, while the exact level of abstraction is relative to a receiver, who in this case is represented simply by a clearance level.

%Such predicates 
%For example, if certain activities had a property named \emph{sensitivity} which contained numeric values, then ``\emph{all entities produced by processes that sensitivity} $\ge$ \emph{5}'' is a valid clause within the policy language.  
%
%
%
%The evaluation of a policy over $G$ results in the identification of the set of nodes $V_{gr}$. The user can then  apply $\group(G, V_{gr}, v_{new}, t)$ to a graph $G \in \guaEAG$, where $t$ is again selected by the user. 
%



\section{Summary and further research}
\label{sec:further}

We have proposed a model for the principled \jwbtwo{hiding} of provenance based on a formal definition of abstraction, in which sensitive elements of a provenance graph are grouped together and replaced by a single abstract node.  The presentation is incremental: first we develop the model for the case when the nodes of provenance graphs are restricted to entities and activities, and then consider the case where nodes may also be agents.  A guiding principle  throughout is that we avoid the introduction of \emph{false dependencies}: abstraction will reduce the information content of a provenance graph, but it will not introduce false information.  
The abstraction acts on and results in provenance graphs which are PROV compliant.   A separate paper presents the tool implementing this model in detail.


The work described in this paper is progressing in two main directions.
%
First, we are aware that the fragment of PROV to which this version of $\group$ applies does not cover all relation types. Nevertheless, the method described in the paper for reasoning about PROV graph transformation can be used as a guideline to extend the work to the missing parts of PROV. We are going to address these in the future.

Second, so far we have ignored the implications of abstraction on the space of events that are used to characterize the semantics of PROV. As constraints over the relative ordering of events are defined in detail in the \jwbtwo{PROV-CONSTRAINTS} document, there is however an obligation to extend the notion of validity of PROV graphs to include those constraints. Thus, grouping must be shown to be validity-preserving relative to those constraints as well. 

\comment{Address: The authors have not considered what domains enforce a validity constraint and what if it is relaxed to show an partially inconsistent graph?}



\section*{Acknowledgements}

The support of the ONRG and the EPSRC in funding this research is gratefully acknowledged. 

\section*{References}

\bibliographystyle{plain}
\bibliography{prov-abstraction-foundations}

\appendix

%\section{Proofs}
\subsection{The function $\col$}
\mnote{still to do:
\begin{itemize}
  \item calc edges of $E_{21}$ and show equal to $E_{12}$.
  \item whole proof of equality of nodes. 
  \item so $(G_{12} = G_{21})$. 
\end{itemize}
}




$\col$ collapses the path between two nodes, and replaces it with a new node. All edges into or out of the path are relabeled according to the function $\relabel$.
  
\begin{definition}[$\col$]  \label{def:col}
  For nodes $x$, $y$ in a graph $G = (V,E)$, and a new node $v_{N}$ not in $V$,  $\col(x,y,G)$ is defined provided there is a path between nodes $x$ and $y$. Let $P$ be the set of all nodes on the path from $x$ to $y$. Let $r(l)$ stand for the function $\relabel(l)$. Then $\col(x,y,G) =  (V',E')$, where 
  \begin{eqnarray*}
  V' & = & (V\hide P) \union \{v_N\}     \\
  E' & = & E\hide (
                   \{(v',v,l)|(v',v,l) \in in(v), v\in P\}
                   \union
                   \{(v,v',l)|(v,v',l) \in out(v), v\in P\}
                  )\\
  && \union\ \{(v',v_{N},r(l))|(v',v,l) \in in(v), v\in P\}\\
  && \union\ \{(v_{N},v',r(l))|(v,v',l) \in out(v), v\in P\}\\
  \end{eqnarray*}
\end{definition}
If $type(v)$ returns the type of a node $v$, the function $relabel$ (abbreviated above as $r$) is defined as 
\begin{definition}[$relabel$] \label{def:relabel}
  For an edge $(v,v',l)$, 
  \[
   (v,v',relabel(l)) = \left\{
   \begin{array}{l}
      l,    {\emph{if}}\; type(v) \neq type(v') \\
      infl, \emph{otherwise}
   \end{array}   \right.
  \]
\end{definition}
\noindent
$relabel$ is promoted in the obvious way to sets of edges. 

To show that  the order of application of two $\col$ functions does not matter, (ie. that $\col$ commutes with itself)  we need to show that for any nodes $w,x,y,z$ in a graph $G$, $\col(y,z,\col(w,x,G)) = \col(w,x,\col(y,z,G))$. 


We begin with some definitions.

$P_1$ is defined as $\Path(w,x)$.

$P_2$ is defined as $\Path(y,z)$.

$\{v_i\} = \Path(x,w) \inter \Path(y,z)$. Note that in general this is a set of nodes. 



%$E_1$ is the set of nodes after $\col(w,x,G)$ has been carried out.

%$E_2$ is the set of nodes after $\col(y,z,G)$ has been carried out.

% If the new node resulting from $\col(y,z,G)$ is $v_{N1}$,

%$P'_1$ is the path between $w$ and $x$ \textbf{after} $\col(y,z,G)$ has been carried out, resulting in the new node $v_{N1}$. It is defined $P'_1 = (P_1\hide\{v_i\})\union\{v_{N1}\}$.  

%Similarly,  $P'_2$ is the path between $y$ and $z$ \textbf{after} $\col(w,x,G)$ has been carried out, resulting in the new node $v_{N2}$.  It is defined as $P'_2 = (P_2\hide\{v_i\})\union\{v_{N2}\}$.  



We  calculate the nodes and edges of  $\col(y,z,\col(w,x,G))$,  then $\col(w,x,\col(y,z,G))$, showing that they both result in the same final graph. For readability, we introduce simplifying definitions through the proof. 


Edges:

Following the definitions recorded above, $\col(w,x,G)$ results in graph $G_1=(V_1,E_1)$, where

\begin{eqnarray*}
  E_1 & = & E\hide (
                   \{(v',v,l)|(v',v,l) \in in(v), v\in P_1\}
                   \union
                   \{(v,v',l)|(v,v',l) \in out(v), v\in P_1\}
                  )\\
  && \union\ \{(v',v_{N},r(l))|(v',v,l) \in in(v), v\in P_1\}\\
  && \union\ \{(v_{N},v',r(l))|(v,v',l) \in out(v), v\in P_1\}\\
\end{eqnarray*}

\noindent
We now introduce the simplifying definitions:
\[
 in(P_1)   = \{(v',v,l)|(v',v,l) \in in(v), v\in P_1\} \\ 
 out(P_1)  = \{(v,v',l)|(v,v',l) \in out(v), v\in P_1\}\\
 in(v_{N1}) = \{(v',v_{N1},r(l))|(v',v,l) \in in(v), v\in P_1\} \\
 out(v_{N1})= \{(v_{N1},v',r(l))|(v,v',l) \in out(v), v\in P_1\} \\
\]
\noindent
With a slight abuse of notation, $in(\{v_{N1}\})$ (resp. $out(\{v_{N1}\})$)  is written $in(v_{N1})$ (resp. $out(v_{N1})$). These simplifies the definition of $E_1$ to 
\[
  E_1  = ( E\hide(in(P_1) \union out(P_1)) ) \union ( in(v_{N1}) \union out(v_{N1}) )
\]
\noindent  
$P_1$ and $P_2$ intersect (on the nodes $\{v_i\}$), so we record the modification in $P_2$ \textbf{after} $\col(w,x,G)$ has been carried out as

It is defined as $P'_2 = (P_2\hide\{v_i\})\union\{v_{N1}\}$.  If $v_f$ is the replacement abstract node, then


\begin{eqnarray*}
  E_{12} =  & = & E_1\hide (
                   \{(v',v,l)|(v',v,l) \in in(v), v\in P'_2\}
                   \union
                   \{(v,v',l)|(v,v',l) \in out(v), v\in P'_2\}
                  )\\
  && \union\ \{(v',v_f,r(l))|(v',v,l) \in in(v), v\in P'_2\}\\
  && \union\ \{(v_f,v',r(l))|(v,v',l) \in out(v), v\in P'_2\}\\
\end{eqnarray*}

\noindent
We  introduce some more simplifying definitions:
\[
in(P'_2) = \{(v',v,l)|(v',v,l) \in in(v), v\in P'_2\}\\
out(P'_2) = \{(v,v',l)|(v,v',l) \in out(v), v\in P'_2\}\\
in(v_f) = \{(v',v_f,r(l))|(v',v,l) \in in(v), v\in P'_2\}\\
out(v_f) = \{(v_f,v',r(l))|(v,v',l) \in out(v), v\in P'_2\}\\
\]
\noindent
allowing us to re-write the defintion of $E_{12}$ as
\[
  E_{12}  =  (E_1\hide ( in(P'_2) \union out(P'_2)))  \union\ (in(v_f) \union out(v_f))
\]
  

\noindent
and then (by substitution of the $E_1$) as


\[
E_{12}  =  \left( \left(
\begin{array}{l}  E\hide(in(P_1) \union out(P_1)) \\  \union \\ (in(v_{N1}) \union out(v_{N1}) ) \\
\end{array} \right)
   \hide ( in(P'_2) \union out(P'_2)) \right) \\
\hphantom{E_{12}  = \;\; }   \union\ \\
\hphantom{E_{12}  = \;\;}   (in(v_f) \union out(v_f))\\ 

\]

\noindent
and then, by substitution of $P'_2$

\[
E_{12}  =  \left( \left(
  \begin{array}{l}
    E\hide(in(P_1) \union out(P_1)) \\  \union \\ (in(v_{N1}) \union out(v_{N1}) ) \\
  \end{array} \right)
   \hide
   \left( \begin{array}{l}
     in((P_2\hide\{v_i\})\union\{v_{N1}\}) \\\union\\ out((P_2\hide\{v_i\})\union\{v_{N1}\})
   \end{array}
   \right) \right) \\
   \hphantom{E_{12}  = \;\; }   \union\ \\
\hphantom{E_{12}  = \;\;}   (in(v_f) \union out(v_f))\\ 
\]

\noindent
then, since $(P_2\hide\{v_i\}) \inter \{v_{N1}\} = \emptyset$, by distributing $in$ and $out$ over $\union$, we get  

\[
E_{12}  =  \left( \left(
  \begin{array}{l}
    E\hide(in(P_1) \union out(P_1)) \\  \union \\ (in(v_{N1}) \union out(v_{N1}) ) \\
  \end{array} \right)
   \hide
   \left( \begin{array}{l}
     in(P_2\hide\{v_i\})\union in(\{v_{N1}\})) \\\union\\ out(P_2\hide\{v_i\})\union out(\{v_{N1}\})
   \end{array}
   \right) \right) \\
   \hphantom{E_{12}  = \;\; }   \union\ \\
\hphantom{E_{12}  = \;\;}   (in(v_f) \union out(v_f))\\ 
\]

\noindent
We can therefore remove $out(\{v_{N1}\})$ and $out(\{v_{N1}\})$ from both sides of the set hiding operator, leaving 

\[
E_{12}  =  \left( \left(
  \begin{array}{l}
    E\hide(in(P_1) \union out(P_1)) \\  
  \end{array} \right)
   \hide
   \left( \begin{array}{l}
     in(P_2\hide\{v_i\}) \\\union\\ out(P_2\hide\{v_i\})
   \end{array}
   \right) \right) \\
   \hphantom{E_{12}  = \;\; }   \union\ \\
\hphantom{E_{12}  = \;\;}   (in(v_f) \union out(v_f))\\ 
\]
\noindent
and, since $\{v_i\}$ is the set of intersection nodes, and therefore $\{v_i\} \subseteq P_1$, we can write

\[
E_{12}  =  
   E\hide
  (
    in(P_1) \union out(P_1) \union in(P_2) \union out(P_2)
  ) \\
   \hphantom{E_{12}  = \;\;}   \union\ \\
\hphantom{E_{12}  = \;\;}   (in(v_f) \union out(v_f))\\ 
\]

Now, we  revisit the defintion of $v_f$. and compare to $v_g$...

Below, we expand the definition of $in(v_f)$. $out(v_f)$ an exercise!

\[
in(v_f) = \{(v',v_f,r(l))|(v',v,l) \in in(v), v\in P'_2\}
\]
\noindent
by defition of $P'_2$,
\[
in(v_f) = \{(v',v_f,r(l))|(v',v,l) \in in(v), v\in((P_2\hide\{v_i\})\union\{v_{N1}\})\}
\]
\noindent
and,  simplification, 
\[
in(v_f) = \{(v',v_f,r(l))|(v',v,l) \in in((P_2\hide\{v_i\}) \union \{v_{N1}\})\}
\]
\noindent
and, by set manipulation, 
\[
in(v_f) = \{(v',v_f,r(l))|(v',v,l) \in in(P_2\hide\{v_i\}) \union  \{(v',v_{N1},r(l))|(v',v,l)\in in(v), v\in P_1\}\}
\]
\noindent
which, by definition of $in(P_1)$, we can write as
\[
in(v_f) = \{(v',v_f,r(l))|(v',v,l) \in in(P_2\hide\{v_i\}) \union  r(in(P_1))\}
\]
\noindent
where $r(in(P))$ is the relabeling operator $r$ promoted to sets.  Which, since $r$ is idempotent ($r(r(l)) = r(l)$), we can write as
\[
in(v_f) = \{(v',v_f,r(l))|(v',v,l) \in in(P_2\hide\{v_i\}) \union  in(P_1)\}
\]
\noindent
and since $\{v_i\} \subseteq P_1$, 
\[
in(v_f) = \{(v',v_f,r(l))|(v',v,l) \in in(P_2) \union  in(P_1)\}
\]

\mnote{still need to repeat this for $E_{21}$ and show two results are equal. This proof might be best in a TR version}

\pagebreak


%\subsection{The function $\repl$}


We begin with definition of the function $\repl$.
\vspace*{10pt}
\begin{definition}[replace]
\label{def:group-replace}

\[ \repl (V^*, v_{new}, G) = (V', E'), \mbox{ where: } \]
\begin{eqnarray*}
V' & = & V  \setminus V^*  \cup \{v_{new}\}\\
E' & = & E  \setminus (\vartheta_{out}(V^*) \cup \vartheta_{in}(V^*) \cup \vartheta_{int}(V^*))  \\
   & & \qquad \cup\  \vartheta_{out}'(V^*)  \cup \vartheta_{in}'(V^*)
\end{eqnarray*}
\end{definition}


We need to show that the order in which $\repl$ is carried out is irrelevant, i.e., if $v_n$ and $v_m$ are the new nodes, we need to show that
\[
\repl(V_b,v_m,\repl(V_a,v_n,G)) = \repl(V_a,v_n,\repl(V_b,v_m,G))
  \]
  


  
%\subsection{Proof}

To show that  the order of application of the two $\repl$ functions does not matter,   we need to show that given a graph $G=(V,E)$,    $\repl(V_b,v_m,\repl(V_a,v_n,G)) = \repl(V_a,v_n,\repl(V_b,v_m,G))$.
  
\vspace{10pt}
{\bf Approach to proof}
\vspace{10pt}

We  calculate the edges and nodes of
$\repl(V_b,v_m,\repl(V_a,v_n,G))$, then the edges and nodes of
$\repl(V_a,v_n,\repl(V_b,v_m,G))$, 
showing that they both result in the same final graph.
%
We begin by calculating the edges of $\repl(V_a,v_n,G)$.

Following the definition of $\repl$ reproduced above, $\repl(V_a,v_n,G)$ results in graph $G_1=(V_1,E_1)$, where

\begin{eqnarray*}
  V_1 & = & V\hide V_a \union \{v_n\}\\
  E_1 & = & E\hide ( \INN(V_a) \union \OUT(V_a) \union \INT(V_a)) \\
  && \union\ \{(v',v_{n})|(v',v) \in \INN(V_a) \}\\
  && \union\ \{(v_{n},v')|(v,v') \in \OUT(V_a) \}\\
\end{eqnarray*}

where we define
\[
\INN(v_n) = \{(v',v_n)|(v',v) \in \INN(V_a) \}\\
\OUT(v_n) = \{(v_n,v')|(v,v') \in \OUT(V_a) \}
\]

\noindent  
Let $I = V_a \inter V_b$.  We record the modification in $V_b$ \textbf{after} $\repl(V_a,v_n,G)$ has been carried out as $V'_b = (V_b\hide I)\union\{v_{n}\}$.
If, following both applications of $\repl$,  $v_f$ is the final replacement abstract node, $V_{12}$ is the final set of nodes and $E_{12}$ is the final set of edges, then

\begin{eqnarray*}
  V_{12} & = & V_1 \hide V'_b \union \{v_m\} \\
  E_{12} &  = & E_1\hide ( \INN(V'_b) \union \OUT(V'_b) \union \INT(V'_b) )\\
  && \union\ \{(v',v_{f})|(v',v) \in \INN(V'_b) \}\\
  && \union\ \{(v_{f},v')|(v,v') \in \OUT(V'_b) \}
\end{eqnarray*}
\noindent
where we define 

\[
% in(P'_2) = \{(v',v,l)|(v',v,l) \in in(v), v\in P'_2\}\\
% out(P'_2) = \{(v,v',l)|(v,v',l) \in out(v), v\in P'_2\}\\
 \INN(v_f) = \{(v',v_f)|(v',v) \in \INN(V'_b) \}\\
 \OUT(v_f) = \{(v_f,v')|(v,v') \in \OUT(V'_b) \}
 \]

  

 \noindent
 At this point we evaluate $E_{12}$ separately, then $V_{12}$.  By substitution of the $E_1$, we get: 

\[
E_{12}  =  \left( \left(
\begin{array}{l}  E\hide(\INN(V_a) \union \OUT(V_a) \union \INT(V_a)) \\  \union \\ (\INN(v_n) \union \OUT(v_n) ) \\
\end{array} \right)
   \hide ( \INN(V'_b) \union \OUT(V'_b) \union \INT(V'_b) ) \right) \\
\hphantom{E_{12}  = \;\; }   \union\ \\
\hphantom{E_{12}  = \;\;}   (\INN(v_f) \union \OUT(v_f))
\]
\noindent
and then, by substitution of the $V'_b$ (recall $V'_b$ is the modification made to $V'_b$ by the first $\repl$ operation: $V'_b = (V_b\hide I)\union\{v_{n}\}$):

\[
E_{12}  =  \left( \left(
\begin{array}{l}  E\hide(\INN(V_a) \union \OUT(V_a) \union \INT(V_a)) \\  \union \\ (\INN(v_n) \union \OUT(v_n) ) \\
\end{array} \right)
\hide    \left( \begin{array}{l}
  ( \INN((V_b\hide I)\union\{v_{n}\})~ \union\ \\ \OUT((V_b\hide I)\union\{v_{n}\})~ \union\ \\ \INT((V_b\hide I)\union\{v_n\}) ) 
         \end{array}
         \right) \right) \\
\hphantom{E_{12}  = \;\; }   \union\ \\
\hphantom{E_{12}  = \;\;}   (\INN(v_f) \union \OUT(v_f))\\ 
\]

\noindent
then, since $(V_b\hide I) \inter \{v_{n}\} = \emptyset$, we can distribute $\INN$ and $\OUT$  over $\union$. Furthermore,  $\INT(v_n) = \emptyset$, since $v_n$ is a single node.

\[
E_{12}  =  \left( \left(
\begin{array}{l}
  E\hide(\INN(V_a) \union \OUT(V_a) \union \INT(V_a)) \\  \union \\ (\INN(v_n) \union \OUT(v_n) ) \\
\end{array} \right)
   \hide
   \left( \begin{array}{l}
     \INN(V_b\hide I)\union \INN(v_n) ~\union\ \\
     \OUT(V_b\hide I)\union \OUT(v_n)  \\
     \INT(V_b\hide I)) 
   \end{array}
   \right) \right) \\
   \hphantom{E_{12}  = \;\; }   \union\ \\
\hphantom{E_{12}  = \;\;}   (\INN(v_f) \union \OUT(v_f))\\ 
\]

\noindent
We can therefore remove $\INN(\{v_n\})$ and $\OUT(\{v_n\})$ from both sides of the set hiding operator, leaving 

\[
E_{12}  =  \left( \left(
  \begin{array}{l}
  E\hide(\INN(V_a) \union \OUT(V_a) \union \INT(V_a)) ) \\
  \end{array} \right)
   \hide
   \left( \begin{array}{l}
     \INN(V_b\hide I) ~\union\ \\
     \OUT(V_b\hide I) ~\union\ \\
          \INT(V_b\hide I)) 
%     \INT(V_b\hide I)\union \INT(\{v_n\}) 
   \end{array}
   \right) \right) \\
   \hphantom{E_{12}  = \;\; }   \union\ \\
\hphantom{E_{12}  = \;\;}   (\INN(v_f) \union \OUT(v_f))\\ 
\]


\noindent
and, since $I = V_a \inter V_b$,  and therefore $I \subseteq V_a$, we can write

\[
E_{12}  = 
  E\hide(\INN(V_a) \union \OUT(V_a) \union \INT(V_a) \union    \INN(V_b) ~\union\ \OUT(V_b)   ~\union\   \INT(V_b\hide I))  \\
   \hphantom{E_{12}  = \;\; }   \union\ \\
\hphantom{E_{12}  = \;\;}   (\INN(v_f) \union \OUT(v_f))\\ 
\]





Next, we expand the definition of $\INN(v_f)$. Recall \comment{hit the buffers here; continue later}  \comment{wrong tack: instead tur $\INN(V_b\hide I)$ into $\INN(V_b)$. Then do the same with $E_{21}$. 


\[
 \INN(v_f) = \{(v',v_f)|(v',v) \in \INN(V'_b) \}\\
\]
\noindent
so, by definition of $V'_b$ ($V'_b = (V_b\hide I)\union\{v_{n}\}$)
\[
 \INN(v_f) = \{(v',v_f)|(v',v) \in \INN((V_b\hide I)\union\{v_{n}\}) \}\\
 \]
   and..


  %\[
 %\INN(v_f) = \{(v',v_f)|(v',v) \in \INN(V_b\hide I)\union \INN\{v_{n}\}) \}\\
 %\]

   so, since ($\INN(v_n) = \{(v',v_n)|(v',v) \in \INN(V_a) \}$)
   
\[
 \INN(v_f) = \{(v',v_f)|(v',v) \in \INN(V_b\hide I)\union  \{(v',v_n)|(v',v) \in \INN(V_a) \} \}\\
\]
\noindent
and,  since $I = V_a \inter V_b$, 
\[
 \INN(v_f) = \{(v',v_f)|(v',v) \in \INN(V_b)\union  \{(v',v_n)|(v',v) \in \INN(V_a) \} \}\\
\]

XXXXXXX

\[
in(v_f) = \{(v',v_f,r(l))|(v',v,l) \in in((P_2\hide\{v_i\}) \union \{v_{N1}\})\}
\]
\noindent
and, by set manipulation, 
\[
in(v_f) = \{(v',v_f,r(l))|(v',v,l) \in in(P_2\hide\{v_i\}) \union  \{(v',v_{N1},r(l))|(v',v,l)\in in(v), v\in P_1\}\}
\]
\noindent
which, by definition of $in(P_1)$, we can write as
\[
in(v_f) = \{(v',v_f,r(l))|(v',v,l) \in in(P_2\hide\{v_i\}) \union  r(in(P_1))\}
\]
\noindent
where $r(in(P))$ is the relabeling operator $r$ promoted to sets.  Which, since $r$ is idempotent ($r(r(l)) = r(l)$), we can write as
\[
in(v_f) = \{(v',v_f,r(l))|(v',v,l) \in in(P_2\hide\{v_i\}) \union  in(P_1)\}
\]
\noindent
and since $\{v_i\} \subseteq P_1$, 
\[
in(v_f) = \{(v',v_f,r(l))|(v',v,l) \in in(P_2) \union  in(P_1)\}
\]

%\pagebreak

We now carry out similar manipulations for $\col(w,x,col(y,z,G))$, thus showing that the edges of $\col(y,z,col(w,x,G))$ are equal to the edges of  $\col(w,x,col(y,z,G))$ (up to $\alpha$-equivalence). 

\section{part 2}

Following the definitions recorded above, $\col(y,z,G)$ results in graph $G_2=(V_2,E_2)$, where the node introduced is called $v_{N2}$. 

\begin{eqnarray*}
  E_2 & = & E\hide (
                   \{(v',v,l)|(v',v,l) \in in(v), v\in P_2\}
                   \union
                   \{(v,v',l)|(v,v',l) \in out(v), v\in P_2\}
                  )\\
  && \union\ \{(v',v_{N2},r(l))|(v',v,l) \in in(v), v\in P_2\}\\
  && \union\ \{(v_{N},v',r(l))|(v,v',l) \in out(v), v\in P_2\}
\end{eqnarray*}
We now introduce the simplifying definitions:
\begin{eqnarray*}
 in(P_2)   &=& \{(v',v,l)|(v',v,l) \in in(v), v\in P_2\} \\ 
 out(P_2)  &=& \{(v,v',l)|(v,v',l) \in out(v), v\in P_2\}\\
 in(v_{N2}) &=& \{(v',v_{N2},r(l))|(v',v,l) \in in(v), v\in P_2\} \\
 out(v_{N2})&=& \{(v_{N2},v',r(l))|(v,v',l) \in out(v), v\in P_2\} 
\end{eqnarray*}

With a slight abuse of notation, $in(\{v_{N2}\})$ (resp\ $out(\{v_{N2}\})$)  is written $in(v_{N2})$ (resp $out(v_{N2})$). Using these re-writes simplifies the definition of $E_2$ to 
\begin{eqnarray*}
  E_2  &=& ( E\hide(in(P_2) \union out(P_2)) ) \\
       &&\union\\
       && ( in(v_{N2}) \union out(v_{N2}) )
\end{eqnarray*}
       
$P_2$ and $P_1$ intersect (on the set of nodes $\{v_i\}$), so after the path $P_2$ has been removed  we record the modification in $P_1$ as $P'_1$, where $P'_1 = (P_1\hide\{v_i\})\union\{v_{N2}\}$.

We now carry out $col(w,x,G')$. 

We now replace the path $P'_1$ with the replacement abstract node $v_g$, and 
\begin{eqnarray*}
  E_{21} =  & = & E_2\hide (
                   \{(v',v,l)|(v',v,l) \in in(v), v\in P'_1\}
                   \union
                   \{(v,v',l)|(v,v',l) \in out(v), v\in P'_1\}
                  )\\
  && \union\ \{(v',v_g,r(l))|(v',v,l) \in in(v), v\in P'_1\}\\
  && \union\ \{(v_g,v',r(l))|(v,v',l) \in out(v), v\in P'_1\}
\end{eqnarray*}
We  introduce some further simplifying abbreviations:
\[
in(P'_1) = \{(v',v,l)|(v',v,l) \in in(v), v\in P'_1\}\\
out(P'_1) = \{(v,v',l)|(v,v',l) \in out(v), v\in P'_1\}\\
in(v_g) = \{(v',v_g,r(l))|(v',v,l) \in in(v), v\in P'_1\}\\
out(v_g) = \{(v_g,v',r(l))|(v,v',l) \in out(v), v\in P'_1\}\\
\]
\noindent
allowing us to re-write the definition of $E_{21}$ as
\[
  E_{21}  =  (E_2\hide ( in(P'_1) \union out(P'_1)))  \union\ (in(v_g) \union out(v_g))
\]
  

\noindent
and then (by substitution of the $E_2$) as


\[
E_{21}  =  \left( \left(
\begin{array}{l}  E\hide(in(P_2) \union out(P_2)) \\  \union \\ (in(v_{N2}) \union out(v_{N2}) ) \\
\end{array} \right)
   \hide ( in(P'_1) \union out(P'_1)) \right) \\
\hphantom{E_{12}  = \;\; }   \union\ \\
\hphantom{E_{12}  = \;\;}   (in(v_g) \union out(v_g))\\ 

\]

\noindent
and then, by substitution of $P'_2$

\[
E_{21}  =  \left( \left(
  \begin{array}{l}
    E\hide(in(P_2) \union out(P_2)) \\  \union \\ (in(v_{N2}) \union out(v_{N2}) ) \\
  \end{array} \right)
   \hide
   \left( \begin{array}{l}
     in((P_1\hide\{v_i\})\union\{v_{N2}\}) \\\union\\ out((P_1\hide\{v_i\})\union\{v_{N2}\})
   \end{array}
   \right) \right) \\
   \hphantom{E_{12}  = \;\; }   \union\ \\
\hphantom{E_{12}  = \;\;}   (in(v_g) \union out(v_g))\\ 
\]

\noindent
then, since $(P_1\hide\{v_i\}) \inter \{v_{N2}\} = \emptyset$, by distributing $in$ and $out$ over $\union$, we get  

\[
E_{21}  =  \left( \left(
  \begin{array}{l}
    E\hide(in(P_2) \union out(P_2)) \\  \union \\ (in(v_{N2}) \union out(v_{N2}) ) \\
  \end{array} \right)
   \hide
   \left( \begin{array}{l}
     in(P_1\hide\{v_i\})\union in(\{v_{N2}\})) \\\union\\ out(P_1\hide\{v_i\})\union out(\{v_{N2}\})
   \end{array}
   \right) \right) \\
   \hphantom{E_{12}  = \;\; }   \union\ \\
\hphantom{E_{12}  = \;\;}   (in(v_g) \union out(v_g))\\ 
\]

\noindent
We can therefore remove $out(\{v_{N2}\})$ and $out(\{v_{N2}\})$ from both sides of the set hiding operator, leaving 

\[
E_{21}  =  \left( \left(
  \begin{array}{l}
    E\hide(in(P_2) \union out(P_2)) \\  
  \end{array} \right)
   \hide
   \left( \begin{array}{l}
     in(P_1\hide\{v_i\}) \\\union\\ out(P_1\hide\{v_i\})
   \end{array}
   \right) \right) \\
   \hphantom{E_{12}  = \;\; }   \union\ \\
\hphantom{E_{12}  = \;\;}   (in(v_g) \union out(v_g))\\ 
\]
\noindent
and, since $\{v_i\}$ is the set of intersection nodes, and therefore $\{v_i\} \subseteq P_2$, we can write

\[
E_{21}  =  
   E\hide
  (
    in(P_2) \union out(P_2) \union in(P_1) \union out(P_1)
  ) \\
   \hphantom{E_{12}  = \;\;}   \union\ \\
\hphantom{E_{12}  = \;\;}   (in(v_g) \union out(v_g))\\ 
\]



Now, we expand the definition of $in(v_g)$. $out(v_g)$ an exercise!

\[
in(v_g) = \{(v',v_g,r(l))|(v',v,l) \in in(v), v\in P'_1\}
\]
\noindent
by definition of $P'_1$,
\[
in(v_g) = \{(v',v_g,r(l))|(v',v,l) \in in(v), v\in((P_1\hide\{v_i\})\union\{v_{N2}\})\}
\]
\noindent
and,  by simplification, 
\[
in(v_g) = \{(v',v_g,r(l))|(v',v,l) \in in((P_1\hide\{v_i\}) \union \{v_{N2}\})\}
\]
\noindent
and, by set manipulation, 
\[
in(v_g) = \{(v',v_g,r(l))|(v',v,l) \in in(P_1\hide\{v_i\}) \union  \{(v',v_{N2},r(l))|(v',v,l)\in in(v), v\in P_2\}\}
\]
\noindent
which, by definition of $in(P_2)$, we can write as
\[
in(v_g) = \{(v',v_g,r(l))|(v',v,l) \in in(P_1\hide\{v_i\}) \union  r(in(P_2))\}
\]
\noindent
where $r(in(P))$ is the relabeling operator $r$ promoted to sets.  Since $r$ is idempotent, we can write this as
\[
in(v_g) = \{(v',v_g,r(l))|(v',v,l) \in in(P_1\hide\{v_i\}) \union  in(P_2)\}
\]
\noindent
and since $\{v_i\} \subseteq P_2$, 
\[
in(v_g) = \{(v',v_g,r(l))|(v',v,l) \in in(P_1) \union  in(P_2)\}
\]



\comment{We now need to show that $v_f=v_g$ under $\alpha$-equivalence. \comment{haven't said what $v_g$ is yet} We do this by showing that $in(v_f)=in(v_g)$. The demonstration that $out(v_f)=out(v_g)$ proceeds in a similar fashion. }



It is thus clear that $in(v_f)=in(v_g)$. 


\subsection{equality of nodes}

 We now need to show that for any nodes $w,x,y,z$ in a graph $G$, the nodes in the graph resulting from $\col(y,z,\col(w,x,G))$ are the same as the nodes resulting from $\col(w,x,\col(y,z,G))$. 

Recall from Definition~\ref{def:col} that if $x$ and $y$ are nodes, and $P$ is the set of all nodes on the path from $x$ to $y$, then  $\col(x,y,G) =  (V',E')$, where 
  \begin{eqnarray*}
  V' & = & (V\hide P) \union \{v_{N1}\}     \\
  \end{eqnarray*}


Let $P_1$ be defined as $\Path(w,x)$ and $P_2$  as $\Path(y,z)$ and $\{v_i\} = \Path(x,w) \inter \Path(y,z)$.

Consider first $\col(y,z,\col(w,x,G))$, and let $v_{N1}$ be the intermediate node, and $v_f$ be the final node. 

Then by Definition~\ref{def:col}, $V_1 = (V\hide P_1) \union \{v_{N1}\}$. $P_2$ is modified, and $P'_2= P_2\hide \{v_i\} \union \{v_{N1}\}$ and the final set of nodes $V''$ is given by $V_{12} = V_1\hide P'_2 \union \{v_f\}$. 

So
\[
V_{12} \\
=\\
V_1\hide P'_2 \union \{v_f\} \\
=\\
V_1\hide ((P_2\hide \{v_i\}) \union \{v_{N1}\}) \union \{v_f\}\\
=\\
((V\hide P_1) \union \{v_{N1}\})\hide ((P_2\hide \{v_i\}) \union \{v_{N1}\}) \union \{v_f\}\\
=(\textrm{removing}\; v_{N1}\; \textrm{from both sides of the hiding operator})\\
((V\hide P_1))\hide ((P_2\hide \{v_i\}) ) \union \{v_f\}\\ 
= (\textrm{since}\; \{v_i\} \subseteq P_1 \inter P_2)\\
((V\hide P_1))\hide ((P_2) ) \union \{v_f\}\\ 
=\\
(V\hide (P_1 \union P_2)) \union \{v_f\}\\ 
\]

A similar argument holds for $V_{21}$, showing $V_{21} = (V\hide (P_2 \union P_1)) \union \{v_g\}$ and so $V_{21}=V_{12}$ (up to $\alpha$-equivalence).

Thus $\col(y,z,\col(w,x,G)) = \col(w,x,\col(y,z,G))$, as required. 



\end{document}
