%%%%%%%%%%%%
%%
%%%%%%%%%%%%

\section{Background}
\label{sec:prov-background}

\subsection{Core PROV model} \label{sec:prov-core}

We now introduce the core elements of the PROV model, which forms the basis for the grouping operator.
%
We maintain a dual view of provenance, both as a relational model (with binary relations) and as a graph model. Viewed as a relational model, PROV includes the three types of elements: Entities ($\en$), Activities ($\act$), \jwbthree{and Agents. However in this paper we restrict our attention to entities and activities} and \jwbthree{the relations} between them. Agents have proved difficult to incorporate into our framework, but the results we have with entities and activities are worth recording.  
In line with the description in~\citep{w3c-prov-dm} (Section 2), PROV is defined by the following core relations, with common abbreviations in brackets. 

\begin{eqnarray*}
Used~~(\used)  & \subseteq & \act \times \en \\
WasGeneratedBy~~(\wgby) & \subseteq  & \en \times \act \\
WasDerivedFrom~~(\wdf) & \subseteq   & \en \times \en \\
%WasInvalidatedBy~~(\inv) &  \subseteq &  \en \times \act \\
%WasAssociatedWith~~(\waw) & \subseteq & \act \times \ag \\
%ActedOnBehalfOf~~(\delegate) & \subseteq & \ag \times \ag \\ 
%WasAttributedTo~~(\attrTo) & \subseteq & \en \times \ag \\
WasInformedBy~~(\wasInfBy) & \subseteq & \act \times \act
\end{eqnarray*}


\begin{figure}
\centering
\includegraphics[scale=.45]{figures/prov-essentials.pdf} 
\caption{Core elements of the PROV model, adapted from~\citep{w3c-prov-dm}.}
\label{fig:prov-core}
\end{figure}

%\subsection{Bipartite PROV: $\guEA$}  \label{sec:prov-guea}
These are summarized in Fig.~\ref{fig:prov-core}.
%

%\jwbthree{\sloppy In this paper we are going to restrict ourselves to an even simpler model consisting only of $\en$, $\act$, and relations $\used$ and $\wgby$.  An instance  of this model is  therefore a provenance document $D$ consisting of sets $en \in \en$ and $act \in \act$ of symbols, and sets of relation instances $\{ \wgby(e,a)  | e \in \en, a \in \act \} \cup   \{ \used(a,e)  | e \in \en, a \in \act\}$. }


%\jwbthree{In this paper we  restrict ourselves to a model consisting only of $\en$, $\act$, and relations $\used$ and $\wgby$.}

\jwbthree{
We note that, when considering graphs containing the relations $\wasInfBy$ and $\wdf$, we can replace those relations with patterns involving only $\used$ and $\wgby$ using Inference 5 (communication-generation-use-inference)\footnote{\url{https://www.w3.org/TR/prov-constraints/#communication-generation-use-inference}} and Inference 11 (derivation-generation-use-inference)\footnote{\url{https://www.w3.org/TR/prov-constraints/#derivation-generation-use-inference}} in the  PROV-CONSTRAINTS document~\citep{w3c-prov-constraints}.}
\jwbthree{
This replacement has to be handled with care, however. In the case of $\wasInfBy$ and Inference 5, PROV-CONSTRAINTS contains a corresponding inference (Inference 6, generation-communication-use-inference)\footnote{\url{https://www.w3.org/TR/prov-constraints/#generation-communication-use-inference}} that allows the reverse replacement, but there is no such reverse replacement for $\wdf$ and Inference 11, meaning that the original graph cannot be inferred back again, and in general the use of Inference 11  loses information. }


%
%An instance  of \jwbthree{this model  consists} of sets $en \in \en$ and $act \in \act$ of symbols, and sets of relation instances $\{ \wgby(e,a)  | e \in \en, a \in \act \} \cup   \{ \used(a,e)  | e \in \en, a \in \act\}$. 

%
As \jwbthree{the relations that we consider ($\used$ and $\wgby$)} are binary, we \jwbthree{can view a provenance graph $D$} as a bipartite digraph $G=(V,E)$, where $V= \en \cup \act$, and each relation instance maps to a labelled directed edge. By convention, we orient these edges from right to left, to denote that the relation ``points back to the past''. Thus:
$a \xleftarrow{\wgby} e \in E$ iff $\wgby(e,a) \in D$, and $e \xleftarrow{\used} a \in E$ iff $\used(a,e) \in D$.
%
We denote the label associated to edge $(v_i, v_j)$ as $\elabel(v_i,v_j)$. 

%Note that $G$ is a bipartite graph.
We denote a generic such graph by $\guEA$, to indicate that it only contains $\en$ and $\act$ nodes, and $\wgby$ and $\used$ edges. \jwbthree{In the rest of this paper we will equate provenance documents with their graphical models.}
Fig.~\ref{fig:baseline-ug-ae} portrays a simple $\guEA$ graph that we will be using as a running example. 

 %In Sec.~\ref{sec:agents-abstraction} we are going to extend this set to include agents as well as additional relations.

%\comment{Old version is above, new below}

\begin{figure}
\centering
%\includegraphics[scale=.6]{figures/baseline-ug-ae.pdf} 
%\includegraphics[scale=.15]{reworked-fig5.png} 
\includegraphics[scale=.6]{reworked-fig5.pdf} 
\caption{$\guEA$ provenance graph used as a running example to illustrate abstraction by grouping.}  \label{fig:baseline-ug-ae}
\end{figure}

\subsection{Events in $\guEA$}  \label{sec:prov-events}

\label{sec:events}

Central to PROV is the notion that provenance is marked by events. A partial order is defined over events, so that it may or may not be possible to establish whether or not one event precedes another. 
%
Events occur instantaneously, and they mark the lifetime boundaries of Entities (generation, invalidation), Activities (start, end), and Agents (start, end), as well as some of the interactions amongst those elements. These include the generation and usage of an entity by an activity, attribution of an entity to an agent, and more. More specifically, the PROV-CONSTRAINTS document~\citep{w3c-prov-constraints} defines the following types of events (quoted verbatim from Section 2.2):

\begin{itemize} %
	\item\textbf{An activity start event} is the instantaneous event that marks the instant an activity starts.
	
	\item\textbf{An activity end event} is the instantaneous event that marks the instant an activity ends.
	
	\item\textbf{An entity generation event} is the instantaneous event that marks the final instant of an entity's creation timespan, after which it is available for use. The entity did not exist before this event.
	
	\item\textbf{An entity usage event}  is the instantaneous event that marks the first instant of an entity's consumption timespan by an activity. The described usage had not started before this instant, although the activity could potentially have used the same entity at a different time.
	
% \item\textbf{An entity invalidation event} is the instantaneous event that marks the initial instant of the destruction, invalidation, or cessation of an entity, after which the entity is no longer available for use. The entity no longer exists after this event.
	
\end{itemize}
%\end{description}

\paolo{
We denote the start and end events of an activity $a$ as $\mathit{start}(a)$, $\mathit{end}(a)$, respectively, and we
write $ev(\wgby(e,a))$ and $ev(\used(a,e)$) to refer to events associated to instances $\wgby(e,a)$ and $\used(a,e)$) of entity generation and usage, respectively.
}

\paolo{As an example, in the graph of Fig.~\ref{fig:baseline-ug-ae} the generation relation $\wgby(e_4, a_1)$ has an associated generation event $ev(\wgby(e_4, a_1))$, whilst $a_1$ has start and / or end events, written $start(a_1)$ and $end(a_1)$, respectively. Similarly, usage of $e_4$ by $a_2$ is marked by event $ev(\used(a_2, e_4))$. }

\subsection{Constraints and valid $\guEA$ graphs}
\label{sec:prov-constraints}

%As mentioned earlier, our goal in this work is to define transformations of a valid PROV instance into new valid instance, whilst providing obfuscation by abstracting out some of its details. 
\paolo{Validity of a PROV document is defined in terms of a set of constraints, as stated in the PROV-CONSTRAINTS document~\citep{w3c-prov-constraints}.
%
For instance, Constraint 55 (``entity-activity-disjoint'') states that the Entities and Activities are disjoint: 
\[\en \cap \act = \emptyset\]
}
Thus, a \jwb{provenance graph $PG$ in which} both (1) $a_1 \xleftarrow{\used} e_1$ and (2) $e_1 \xleftarrow{\used} a_1$ cannot be valid, because by definition % of $\used$ given earlier,
 (1) entails  $e_1 \in \en$, $a_1 \in \act$, while (2) entails $a_1 \in \en$, $e_1 \in \act$, violating the constraint. We refer to this constraint in the sequel as C1. 
%
%Note that disjointness constraint 55 entails that $\guEA$ graphs are bipartite.

\paolo{In this paper we are mainly concerned with temporal constraints which apply to $\guEA$ instances and determine admissible partial orderings over events.
More precisely, let $\preorder ~~\subset \Ev \times \Ev$ denote a pre-order relation\footnote{Recall that a pre-order is a binary relation with reflexivity and transitivity, but no symmetry or anti-symmetry.} on the set $\Ev$ of events associated to instances of activities and relations as defined above.}
%
\paolo{For a PROV document to be valid, $\preorder$ is required to satisfy the following set of constraints\footnote{For simplicity, \textit{entity invalidation constraints} are not considered.}(using the original numbering in~\citep{w3c-prov-constraints}):}
\begin{itemize}
	
	\item\textbf{C2: generation-generation-ordering (Constraint 39):}  If an entity is generated by more than one activity, then the generation events must all be simultaneous.
	
	Let 
	$gen_1 = ev(\wgby(e, a_1))$, $gen_2 = ev(\wgby(e, a_2)) \in \pg$. Then  \[gen_1  \preorder  gen_2, \quad gen_2 \preorder gen_1\] must hold.
	
	\item\textbf{C3: generation-precedes-usage(Constraint 37):} A generation event for an entity must precede any usage event for that entity.
	%
	For any $a \in \act$ such that $\used(a,e) \in \pg$, \[	ev(\wgby(e, a)) \preorder ev(\used(a,e))\] must hold.
	
%	\item\textbf{C4: generation-precedes-invalidation (Constraint 36):} The generation event (or, more accurately, the set of simultaneous generation events) for an entity must precede the invalidation event.
	
%	For any $a,a' \in \act$ such that $\wgby(e, a))$, $\inv(e,a') \in \pg$ :
%	\[ ev(\wgby(e, a)) \preorder ev(\inv(e,a')) \]
	
%	\item\textbf{C5: usage-precedes-invalidation (Constraint 38):} Any usage event for an entity must precede the invalidation event.
%	
%	For any $a,a' \in \act$ such that $\used(a,e))$, $\inv(e,a') \in \pg$ :
%	\[ ev(\used(a,e)) \preorder ev(\inv(e,a')) \]
	
	\item\textbf{C4: usage-within-activity (Constraint 33):} Any usage of $e$ by $a$ cannot precede the start of $a$ and must precede the end of $a$. For any $e\in \en, a \in \act$ such that $\used(a,e) \in \pg$:
	\[start(a) \preorder ev(\used(a,e))   \preorder end(a)\]
	
	\item\textbf{C5: generation-within-activity (Constraint 34):} The generation of $e$ by $a$ cannot precede the start of $a$ and must precede the end of $a$.
	Let $\wgby(e,a) \in \pg$:
	\[ start(a) \preorder ev(\wgby(e,a))  \preorder end(a)\]
	
%	\item\textbf{C8: invalidation-invalidation-ordering (Constraint 40):}	If an entity is invalidated by more than one activity, the events must all be simultaneous. \jwbtwo{If $\inv(e,a_1), \inv(e,a_2) \in \pg$, then }
%	  \begin{align*}
 %         &\jwbtwo{ev(\inv(e,a_1)) \preorder  ev(\inv(e,a_2)) ~~\mathrm{and}} \\
 %         &\jwbtwo{ev(\inv(e,a_2)) \preorder  ev(\inv(e,a_1))}
  %        \end{align*}
%\jwbtwo{must hold.}

\end{itemize}

Additional relevant constraints state that multiple start (resp. end) events must all be simultaneous, and that the start event of an activity must precede the end event for that activity.
\\

\begin{definition}[Validity]
 A graph $G \in \guEA$ is valid iff it satisfies constraints C1-C5.
	\label{def:valid-guea}
\end{definition}

%\comment{should finish off with a sign-post para here}
\jwb{In the next section we present the $\group$ operator. We will return to the constraints above in Section~\ref{sec:event}, where we propose a new set of \textit{abstract events} that are derived from these and apply to the abstract graphs.}
