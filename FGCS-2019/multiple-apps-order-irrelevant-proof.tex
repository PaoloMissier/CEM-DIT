\subsection{The function $\repl$}


We begin with definition of the function $\repl$.
\vspace*{10pt}
\begin{definition}[replace]
\label{def:group-replace}

\[ \repl (V^*, v_{new}, G) = (V', E'), \mbox{ where: } \]
\begin{eqnarray*}
V' & = & V  \setminus V^*  \cup \{v_{new}\}\\
E' & = & E  \setminus (\vartheta_{out}(V^*) \cup \vartheta_{in}(V^*) \cup \vartheta_{int}(V^*))  \\
   & & \qquad \cup\  \vartheta_{out}'(V^*)  \cup \vartheta_{in}'(V^*)
\end{eqnarray*}
\end{definition}


We need to show that the order in which $\repl$ is carried out is irrelevant, i.e., if $v_n$ and $v_m$ are the new nodes, we need to show that
\[
\repl(V_b,v_m,\repl(V_a,v_n,G)) = \repl(V_a,v_n,\repl(V_b,v_m,G))
  \]
  


  
%\subsection{Proof}

To show that  the order of application of the two $\repl$ functions does not matter,   we need to show that given a graph $G=(V,E)$,    $\repl(V_b,v_m,\repl(V_a,v_n,G)) = \repl(V_a,v_n,\repl(V_b,v_m,G))$.
  
\vspace{10pt}
{\bf Approach to proof}
\vspace{10pt}

We  calculate the edges and nodes of
$\repl(V_b,v_m,\repl(V_a,v_n,G))$, then the edges and nodes of
$\repl(V_a,v_n,\repl(V_b,v_m,G))$, 
showing that they both result in the same final graph.
%
We begin by calculating the edges of $\repl(V_a,v_n,G)$.

Following the definition of $\repl$ reproduced above, $\repl(V_a,v_n,G)$ results in graph $G_1=(V_1,E_1)$, where

\begin{eqnarray*}
  V_1 & = & V\hide V_a \union \{v_n\}\\
  E_1 & = & E\hide ( \INN(V_a) \union \OUT(V_a) \union \INT(V_a)) \\
  && \union\ \{(v',v_{n})|(v',v) \in \INN(V_a) \}\\
  && \union\ \{(v_{n},v')|(v,v') \in \OUT(V_a) \}\\
\end{eqnarray*}

where we define
\[
\INN(v_n) = \{(v',v_n)|(v',v) \in \INN(V_a) \}\\
\OUT(v_n) = \{(v_n,v')|(v,v') \in \OUT(V_a) \}
\]

\noindent  
Let $I = V_a \inter V_b$.  We record the modification in $V_b$ \textbf{after} $\repl(V_a,v_n,G)$ has been carried out as $V'_b = (V_b\hide I)\union\{v_{n}\}$.
If, following both applications of $\repl$,  $v_f$ is the final replacement abstract node, $V_{12}$ is the final set of nodes and $E_{12}$ is the final set of edges, then

\begin{eqnarray*}
  V_{12} & = & V_1 \hide V'_b \union \{v_m\} \\
  E_{12} &  = & E_1\hide ( \INN(V'_b) \union \OUT(V'_b) \union \INT(V'_b) )\\
  && \union\ \{(v',v_{f})|(v',v) \in \INN(V'_b) \}\\
  && \union\ \{(v_{f},v')|(v,v') \in \OUT(V'_b) \}
\end{eqnarray*}
\noindent
where we define 

\[
% in(P'_2) = \{(v',v,l)|(v',v,l) \in in(v), v\in P'_2\}\\
% out(P'_2) = \{(v,v',l)|(v,v',l) \in out(v), v\in P'_2\}\\
 \INN(v_f) = \{(v',v_f)|(v',v) \in \INN(V'_b) \}\\
 \OUT(v_f) = \{(v_f,v')|(v,v') \in \OUT(V'_b) \}
 \]

  

 \noindent
 At this point we evaluate $E_{12}$ separately, then $V_{12}$.  By substitution of the $E_1$, we get: 

\[
E_{12}  =  \left( \left(
\begin{array}{l}  E\hide(\INN(V_a) \union \OUT(V_a) \union \INT(V_a)) \\  \union \\ (\INN(v_n) \union \OUT(v_n) ) \\
\end{array} \right)
   \hide ( \INN(V'_b) \union \OUT(V'_b) \union \INT(V'_b) ) \right) \\
\hphantom{E_{12}  = \;\; }   \union\ \\
\hphantom{E_{12}  = \;\;}   (\INN(v_f) \union \OUT(v_f))
\]
\noindent
and then, by substitution of the $V'_b$ (recall $V'_b$ is the modification made to $V'_b$ by the first $\repl$ operation: $V'_b = (V_b\hide I)\union\{v_{n}\}$):

\[
E_{12}  =  \left( \left(
\begin{array}{l}  E\hide(\INN(V_a) \union \OUT(V_a) \union \INT(V_a)) \\  \union \\ (\INN(v_n) \union \OUT(v_n) ) \\
\end{array} \right)
\hide    \left( \begin{array}{l}
  ( \INN((V_b\hide I)\union\{v_{n}\})~ \union\ \\ \OUT((V_b\hide I)\union\{v_{n}\})~ \union\ \\ \INT((V_b\hide I)\union\{v_n\}) ) 
         \end{array}
         \right) \right) \\
\hphantom{E_{12}  = \;\; }   \union\ \\
\hphantom{E_{12}  = \;\;}   (\INN(v_f) \union \OUT(v_f))\\ 
\]

\noindent
then, since $(V_b\hide I) \inter \{v_{n}\} = \emptyset$, we can distribute $\INN$ and $\OUT$  over $\union$. Furthermore,  $\INT(v_n) = \emptyset$, since $v_n$ is a single node.

\[
E_{12}  =  \left( \left(
\begin{array}{l}
  E\hide(\INN(V_a) \union \OUT(V_a) \union \INT(V_a)) \\  \union \\ (\INN(v_n) \union \OUT(v_n) ) \\
\end{array} \right)
   \hide
   \left( \begin{array}{l}
     \INN(V_b\hide I)\union \INN(v_n) ~\union\ \\
     \OUT(V_b\hide I)\union \OUT(v_n)  \\
     \INT(V_b\hide I)) 
   \end{array}
   \right) \right) \\
   \hphantom{E_{12}  = \;\; }   \union\ \\
\hphantom{E_{12}  = \;\;}   (\INN(v_f) \union \OUT(v_f))\\ 
\]

\noindent
We can therefore remove $\INN(\{v_n\})$ and $\OUT(\{v_n\})$ from both sides of the set hiding operator, leaving 

\[
E_{12}  =  \left( \left(
  \begin{array}{l}
  E\hide(\INN(V_a) \union \OUT(V_a) \union \INT(V_a)) ) \\
  \end{array} \right)
   \hide
   \left( \begin{array}{l}
     \INN(V_b\hide I) ~\union\ \\
     \OUT(V_b\hide I) ~\union\ \\
          \INT(V_b\hide I)) 
%     \INT(V_b\hide I)\union \INT(\{v_n\}) 
   \end{array}
   \right) \right) \\
   \hphantom{E_{12}  = \;\; }   \union\ \\
\hphantom{E_{12}  = \;\;}   (\INN(v_f) \union \OUT(v_f))\\ 
\]


\noindent
and, since $I = V_a \inter V_b$,  and therefore $I \subseteq V_a$, we can write

\[
E_{12}  = 
  E\hide(\INN(V_a) \union \OUT(V_a) \union \INT(V_a) \union    \INN(V_b) ~\union\ \OUT(V_b)   ~\union\   \INT(V_b\hide I))  \\
   \hphantom{E_{12}  = \;\; }   \union\ \\
\hphantom{E_{12}  = \;\;}   (\INN(v_f) \union \OUT(v_f))\\ 
\]





Next, we expand the definition of $\INN(v_f)$. Recall \comment{hit the buffers here; continue later}  \comment{wrong tack: instead tur $\INN(V_b\hide I)$ into $\INN(V_b)$. Then do the same with $E_{21}$. 


\[
 \INN(v_f) = \{(v',v_f)|(v',v) \in \INN(V'_b) \}\\
\]
\noindent
so, by definition of $V'_b$ ($V'_b = (V_b\hide I)\union\{v_{n}\}$)
\[
 \INN(v_f) = \{(v',v_f)|(v',v) \in \INN((V_b\hide I)\union\{v_{n}\}) \}\\
 \]
   and..


  %\[
 %\INN(v_f) = \{(v',v_f)|(v',v) \in \INN(V_b\hide I)\union \INN\{v_{n}\}) \}\\
 %\]

   so, since ($\INN(v_n) = \{(v',v_n)|(v',v) \in \INN(V_a) \}$)
   
\[
 \INN(v_f) = \{(v',v_f)|(v',v) \in \INN(V_b\hide I)\union  \{(v',v_n)|(v',v) \in \INN(V_a) \} \}\\
\]
\noindent
and,  since $I = V_a \inter V_b$, 
\[
 \INN(v_f) = \{(v',v_f)|(v',v) \in \INN(V_b)\union  \{(v',v_n)|(v',v) \in \INN(V_a) \} \}\\
\]

XXXXXXX

\[
in(v_f) = \{(v',v_f,r(l))|(v',v,l) \in in((P_2\hide\{v_i\}) \union \{v_{N1}\})\}
\]
\noindent
and, by set manipulation, 
\[
in(v_f) = \{(v',v_f,r(l))|(v',v,l) \in in(P_2\hide\{v_i\}) \union  \{(v',v_{N1},r(l))|(v',v,l)\in in(v), v\in P_1\}\}
\]
\noindent
which, by definition of $in(P_1)$, we can write as
\[
in(v_f) = \{(v',v_f,r(l))|(v',v,l) \in in(P_2\hide\{v_i\}) \union  r(in(P_1))\}
\]
\noindent
where $r(in(P))$ is the relabeling operator $r$ promoted to sets.  Which, since $r$ is idempotent ($r(r(l)) = r(l)$), we can write as
\[
in(v_f) = \{(v',v_f,r(l))|(v',v,l) \in in(P_2\hide\{v_i\}) \union  in(P_1)\}
\]
\noindent
and since $\{v_i\} \subseteq P_1$, 
\[
in(v_f) = \{(v',v_f,r(l))|(v',v,l) \in in(P_2) \union  in(P_1)\}
\]

%\pagebreak

We now carry out similar manipulations for $\col(w,x,col(y,z,G))$, thus showing that the edges of $\col(y,z,col(w,x,G))$ are equal to the edges of  $\col(w,x,col(y,z,G))$ (up to $\alpha$-equivalence). 

\section{part 2}

Following the definitions recorded above, $\col(y,z,G)$ results in graph $G_2=(V_2,E_2)$, where the node introduced is called $v_{N2}$. 

\begin{eqnarray*}
  E_2 & = & E\hide (
                   \{(v',v,l)|(v',v,l) \in in(v), v\in P_2\}
                   \union
                   \{(v,v',l)|(v,v',l) \in out(v), v\in P_2\}
                  )\\
  && \union\ \{(v',v_{N2},r(l))|(v',v,l) \in in(v), v\in P_2\}\\
  && \union\ \{(v_{N},v',r(l))|(v,v',l) \in out(v), v\in P_2\}
\end{eqnarray*}
We now introduce the simplifying definitions:
\begin{eqnarray*}
 in(P_2)   &=& \{(v',v,l)|(v',v,l) \in in(v), v\in P_2\} \\ 
 out(P_2)  &=& \{(v,v',l)|(v,v',l) \in out(v), v\in P_2\}\\
 in(v_{N2}) &=& \{(v',v_{N2},r(l))|(v',v,l) \in in(v), v\in P_2\} \\
 out(v_{N2})&=& \{(v_{N2},v',r(l))|(v,v',l) \in out(v), v\in P_2\} 
\end{eqnarray*}

With a slight abuse of notation, $in(\{v_{N2}\})$ (resp\ $out(\{v_{N2}\})$)  is written $in(v_{N2})$ (resp $out(v_{N2})$). Using these re-writes simplifies the definition of $E_2$ to 
\begin{eqnarray*}
  E_2  &=& ( E\hide(in(P_2) \union out(P_2)) ) \\
       &&\union\\
       && ( in(v_{N2}) \union out(v_{N2}) )
\end{eqnarray*}
       
$P_2$ and $P_1$ intersect (on the set of nodes $\{v_i\}$), so after the path $P_2$ has been removed  we record the modification in $P_1$ as $P'_1$, where $P'_1 = (P_1\hide\{v_i\})\union\{v_{N2}\}$.

We now carry out $col(w,x,G')$. 

We now replace the path $P'_1$ with the replacement abstract node $v_g$, and 
\begin{eqnarray*}
  E_{21} =  & = & E_2\hide (
                   \{(v',v,l)|(v',v,l) \in in(v), v\in P'_1\}
                   \union
                   \{(v,v',l)|(v,v',l) \in out(v), v\in P'_1\}
                  )\\
  && \union\ \{(v',v_g,r(l))|(v',v,l) \in in(v), v\in P'_1\}\\
  && \union\ \{(v_g,v',r(l))|(v,v',l) \in out(v), v\in P'_1\}
\end{eqnarray*}
We  introduce some further simplifying abbreviations:
\[
in(P'_1) = \{(v',v,l)|(v',v,l) \in in(v), v\in P'_1\}\\
out(P'_1) = \{(v,v',l)|(v,v',l) \in out(v), v\in P'_1\}\\
in(v_g) = \{(v',v_g,r(l))|(v',v,l) \in in(v), v\in P'_1\}\\
out(v_g) = \{(v_g,v',r(l))|(v,v',l) \in out(v), v\in P'_1\}\\
\]
\noindent
allowing us to re-write the definition of $E_{21}$ as
\[
  E_{21}  =  (E_2\hide ( in(P'_1) \union out(P'_1)))  \union\ (in(v_g) \union out(v_g))
\]
  

\noindent
and then (by substitution of the $E_2$) as


\[
E_{21}  =  \left( \left(
\begin{array}{l}  E\hide(in(P_2) \union out(P_2)) \\  \union \\ (in(v_{N2}) \union out(v_{N2}) ) \\
\end{array} \right)
   \hide ( in(P'_1) \union out(P'_1)) \right) \\
\hphantom{E_{12}  = \;\; }   \union\ \\
\hphantom{E_{12}  = \;\;}   (in(v_g) \union out(v_g))\\ 

\]

\noindent
and then, by substitution of $P'_2$

\[
E_{21}  =  \left( \left(
  \begin{array}{l}
    E\hide(in(P_2) \union out(P_2)) \\  \union \\ (in(v_{N2}) \union out(v_{N2}) ) \\
  \end{array} \right)
   \hide
   \left( \begin{array}{l}
     in((P_1\hide\{v_i\})\union\{v_{N2}\}) \\\union\\ out((P_1\hide\{v_i\})\union\{v_{N2}\})
   \end{array}
   \right) \right) \\
   \hphantom{E_{12}  = \;\; }   \union\ \\
\hphantom{E_{12}  = \;\;}   (in(v_g) \union out(v_g))\\ 
\]

\noindent
then, since $(P_1\hide\{v_i\}) \inter \{v_{N2}\} = \emptyset$, by distributing $in$ and $out$ over $\union$, we get  

\[
E_{21}  =  \left( \left(
  \begin{array}{l}
    E\hide(in(P_2) \union out(P_2)) \\  \union \\ (in(v_{N2}) \union out(v_{N2}) ) \\
  \end{array} \right)
   \hide
   \left( \begin{array}{l}
     in(P_1\hide\{v_i\})\union in(\{v_{N2}\})) \\\union\\ out(P_1\hide\{v_i\})\union out(\{v_{N2}\})
   \end{array}
   \right) \right) \\
   \hphantom{E_{12}  = \;\; }   \union\ \\
\hphantom{E_{12}  = \;\;}   (in(v_g) \union out(v_g))\\ 
\]

\noindent
We can therefore remove $out(\{v_{N2}\})$ and $out(\{v_{N2}\})$ from both sides of the set hiding operator, leaving 

\[
E_{21}  =  \left( \left(
  \begin{array}{l}
    E\hide(in(P_2) \union out(P_2)) \\  
  \end{array} \right)
   \hide
   \left( \begin{array}{l}
     in(P_1\hide\{v_i\}) \\\union\\ out(P_1\hide\{v_i\})
   \end{array}
   \right) \right) \\
   \hphantom{E_{12}  = \;\; }   \union\ \\
\hphantom{E_{12}  = \;\;}   (in(v_g) \union out(v_g))\\ 
\]
\noindent
and, since $\{v_i\}$ is the set of intersection nodes, and therefore $\{v_i\} \subseteq P_2$, we can write

\[
E_{21}  =  
   E\hide
  (
    in(P_2) \union out(P_2) \union in(P_1) \union out(P_1)
  ) \\
   \hphantom{E_{12}  = \;\;}   \union\ \\
\hphantom{E_{12}  = \;\;}   (in(v_g) \union out(v_g))\\ 
\]



Now, we expand the definition of $in(v_g)$. $out(v_g)$ an exercise!

\[
in(v_g) = \{(v',v_g,r(l))|(v',v,l) \in in(v), v\in P'_1\}
\]
\noindent
by definition of $P'_1$,
\[
in(v_g) = \{(v',v_g,r(l))|(v',v,l) \in in(v), v\in((P_1\hide\{v_i\})\union\{v_{N2}\})\}
\]
\noindent
and,  by simplification, 
\[
in(v_g) = \{(v',v_g,r(l))|(v',v,l) \in in((P_1\hide\{v_i\}) \union \{v_{N2}\})\}
\]
\noindent
and, by set manipulation, 
\[
in(v_g) = \{(v',v_g,r(l))|(v',v,l) \in in(P_1\hide\{v_i\}) \union  \{(v',v_{N2},r(l))|(v',v,l)\in in(v), v\in P_2\}\}
\]
\noindent
which, by definition of $in(P_2)$, we can write as
\[
in(v_g) = \{(v',v_g,r(l))|(v',v,l) \in in(P_1\hide\{v_i\}) \union  r(in(P_2))\}
\]
\noindent
where $r(in(P))$ is the relabeling operator $r$ promoted to sets.  Since $r$ is idempotent, we can write this as
\[
in(v_g) = \{(v',v_g,r(l))|(v',v,l) \in in(P_1\hide\{v_i\}) \union  in(P_2)\}
\]
\noindent
and since $\{v_i\} \subseteq P_2$, 
\[
in(v_g) = \{(v',v_g,r(l))|(v',v,l) \in in(P_1) \union  in(P_2)\}
\]



\comment{We now need to show that $v_f=v_g$ under $\alpha$-equivalence. \comment{haven't said what $v_g$ is yet} We do this by showing that $in(v_f)=in(v_g)$. The demonstration that $out(v_f)=out(v_g)$ proceeds in a similar fashion. }



It is thus clear that $in(v_f)=in(v_g)$. 


\subsection{equality of nodes}

 We now need to show that for any nodes $w,x,y,z$ in a graph $G$, the nodes in the graph resulting from $\col(y,z,\col(w,x,G))$ are the same as the nodes resulting from $\col(w,x,\col(y,z,G))$. 

Recall from Definition~\ref{def:col} that if $x$ and $y$ are nodes, and $P$ is the set of all nodes on the path from $x$ to $y$, then  $\col(x,y,G) =  (V',E')$, where 
  \begin{eqnarray*}
  V' & = & (V\hide P) \union \{v_{N1}\}     \\
  \end{eqnarray*}


Let $P_1$ be defined as $\Path(w,x)$ and $P_2$  as $\Path(y,z)$ and $\{v_i\} = \Path(x,w) \inter \Path(y,z)$.

Consider first $\col(y,z,\col(w,x,G))$, and let $v_{N1}$ be the intermediate node, and $v_f$ be the final node. 

Then by Definition~\ref{def:col}, $V_1 = (V\hide P_1) \union \{v_{N1}\}$. $P_2$ is modified, and $P'_2= P_2\hide \{v_i\} \union \{v_{N1}\}$ and the final set of nodes $V''$ is given by $V_{12} = V_1\hide P'_2 \union \{v_f\}$. 

So
\[
V_{12} \\
=\\
V_1\hide P'_2 \union \{v_f\} \\
=\\
V_1\hide ((P_2\hide \{v_i\}) \union \{v_{N1}\}) \union \{v_f\}\\
=\\
((V\hide P_1) \union \{v_{N1}\})\hide ((P_2\hide \{v_i\}) \union \{v_{N1}\}) \union \{v_f\}\\
=(\textrm{removing}\; v_{N1}\; \textrm{from both sides of the hiding operator})\\
((V\hide P_1))\hide ((P_2\hide \{v_i\}) ) \union \{v_f\}\\ 
= (\textrm{since}\; \{v_i\} \subseteq P_1 \inter P_2)\\
((V\hide P_1))\hide ((P_2) ) \union \{v_f\}\\ 
=\\
(V\hide (P_1 \union P_2)) \union \{v_f\}\\ 
\]

A similar argument holds for $V_{21}$, showing $V_{21} = (V\hide (P_2 \union P_1)) \union \{v_g\}$ and so $V_{21}=V_{12}$ (up to $\alpha$-equivalence).

Thus $\col(y,z,\col(w,x,G)) = \col(w,x,\col(y,z,G))$, as required. 

