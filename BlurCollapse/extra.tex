\pagebreak

We now carry out similar manipulations for $\col(w,x,col(y,z,G))$, thus showing that the edges of $\col(y,z,col(w,x,G))$ are equal to the edges of  $\col(w,x,col(y,z,G))$ (up to $\alpha$-equivalence). 

\section{part 2}

Following the definitions recorded above, $\col(y,z,G)$ results in graph $G_2=(V_2,E_2)$, where the node introduced is called $v_{N2}$. 

\begin{eqnarray*}
  E_2 & = & E\hide (
                   \{(v',v,l)|(v',v,l) \in in(v), v\in P_2\}
                   \union
                   \{(v,v',l)|(v,v',l) \in out(v), v\in P_2\}
                  )\\
  && \union\ \{(v',v_{N2},r(l))|(v',v,l) \in in(v), v\in P_2\}\\
  && \union\ \{(v_{N},v',r(l))|(v,v',l) \in out(v), v\in P_2\}
\end{eqnarray*}
We now introduce the simplifying definitions:
\begin{eqnarray*}
 in(P_2)   &=& \{(v',v,l)|(v',v,l) \in in(v), v\in P_2\} \\ 
 out(P_2)  &=& \{(v,v',l)|(v,v',l) \in out(v), v\in P_2\}\\
 in(v_{N2}) &=& \{(v',v_{N2},r(l))|(v',v,l) \in in(v), v\in P_2\} \\
 out(v_{N2})&=& \{(v_{N2},v',r(l))|(v,v',l) \in out(v), v\in P_2\} 
\end{eqnarray*}

With a slight abuse of notation, $in(\{v_{N2}\})$ (resp\ $out(\{v_{N2}\})$)  is written $in(v_{N2})$ (resp $out(v_{N2})$). Using these re-writes simplifies the definition of $E_2$ to 
\begin{eqnarray*}
  E_2  &=& ( E\hide(in(P_2) \union out(P_2)) ) \\
       &&\union\\
       && ( in(v_{N2}) \union out(v_{N2}) )
\end{eqnarray*}
       
$P_2$ and $P_1$ intersect (on the set of nodes $\{v_i\}$), so after the path $P_2$ has been removed  we record the modification in $P_1$ as $P'_1$, where $P'_1 = (P_1\hide\{v_i\})\union\{v_{N2}\}$.

We now carry out $col(w,x,G')$. 

We now replace the path $P'_1$ with the replacement abstract node $v_g$, and 
\begin{eqnarray*}
  E_{21} =  & = & E_2\hide (
                   \{(v',v,l)|(v',v,l) \in in(v), v\in P'_1\}
                   \union
                   \{(v,v',l)|(v,v',l) \in out(v), v\in P'_1\}
                  )\\
  && \union\ \{(v',v_g,r(l))|(v',v,l) \in in(v), v\in P'_1\}\\
  && \union\ \{(v_g,v',r(l))|(v,v',l) \in out(v), v\in P'_1\}
\end{eqnarray*}
We  introduce some further simplifying abbreviations:
\[
in(P'_1) = \{(v',v,l)|(v',v,l) \in in(v), v\in P'_1\}\\
out(P'_1) = \{(v,v',l)|(v,v',l) \in out(v), v\in P'_1\}\\
in(v_g) = \{(v',v_g,r(l))|(v',v,l) \in in(v), v\in P'_1\}\\
out(v_g) = \{(v_g,v',r(l))|(v,v',l) \in out(v), v\in P'_1\}\\
\]
\noindent
allowing us to re-write the definition of $E_{21}$ as
\[
  E_{21}  =  (E_2\hide ( in(P'_1) \union out(P'_1)))  \union\ (in(v_g) \union out(v_g))
\]
  

\noindent
and then (by substitution of the $E_2$) as


\[
E_{21}  =  \left( \left(
\begin{array}{l}  E\hide(in(P_2) \union out(P_2)) \\  \union \\ (in(v_{N2}) \union out(v_{N2}) ) \\
\end{array} \right)
   \hide ( in(P'_1) \union out(P'_1)) \right) \\
\hphantom{E_{12}  = \;\; }   \union\ \\
\hphantom{E_{12}  = \;\;}   (in(v_g) \union out(v_g))\\ 

\]

\noindent
and then, by substitution of $P'_2$

\[
E_{21}  =  \left( \left(
  \begin{array}{l}
    E\hide(in(P_2) \union out(P_2)) \\  \union \\ (in(v_{N2}) \union out(v_{N2}) ) \\
  \end{array} \right)
   \hide
   \left( \begin{array}{l}
     in((P_1\hide\{v_i\})\union\{v_{N2}\}) \\\union\\ out((P_1\hide\{v_i\})\union\{v_{N2}\})
   \end{array}
   \right) \right) \\
   \hphantom{E_{12}  = \;\; }   \union\ \\
\hphantom{E_{12}  = \;\;}   (in(v_g) \union out(v_g))\\ 
\]

\noindent
then, since $(P_1\hide\{v_i\}) \inter \{v_{N2}\} = \emptyset$, by distributing $in$ and $out$ over $\union$, we get  

\[
E_{21}  =  \left( \left(
  \begin{array}{l}
    E\hide(in(P_2) \union out(P_2)) \\  \union \\ (in(v_{N2}) \union out(v_{N2}) ) \\
  \end{array} \right)
   \hide
   \left( \begin{array}{l}
     in(P_1\hide\{v_i\})\union in(\{v_{N2}\})) \\\union\\ out(P_1\hide\{v_i\})\union out(\{v_{N2}\})
   \end{array}
   \right) \right) \\
   \hphantom{E_{12}  = \;\; }   \union\ \\
\hphantom{E_{12}  = \;\;}   (in(v_g) \union out(v_g))\\ 
\]

\noindent
We can therefore remove $out(\{v_{N2}\})$ and $out(\{v_{N2}\})$ from both sides of the set hiding operator, leaving 

\[
E_{21}  =  \left( \left(
  \begin{array}{l}
    E\hide(in(P_2) \union out(P_2)) \\  
  \end{array} \right)
   \hide
   \left( \begin{array}{l}
     in(P_1\hide\{v_i\}) \\\union\\ out(P_1\hide\{v_i\})
   \end{array}
   \right) \right) \\
   \hphantom{E_{12}  = \;\; }   \union\ \\
\hphantom{E_{12}  = \;\;}   (in(v_g) \union out(v_g))\\ 
\]
\noindent
and, since $\{v_i\}$ is the set of intersection nodes, and therefore $\{v_i\} \subseteq P_2$, we can write

\[
E_{21}  =  
   E\hide
  (
    in(P_2) \union out(P_2) \union in(P_1) \union out(P_1)
  ) \\
   \hphantom{E_{12}  = \;\;}   \union\ \\
\hphantom{E_{12}  = \;\;}   (in(v_g) \union out(v_g))\\ 
\]



Now, we expand the definition of $in(v_g)$. $out(v_g)$ an exercise!

\[
in(v_g) = \{(v',v_g,r(l))|(v',v,l) \in in(v), v\in P'_1\}
\]
\noindent
by definition of $P'_1$,
\[
in(v_g) = \{(v',v_g,r(l))|(v',v,l) \in in(v), v\in((P_1\hide\{v_i\})\union\{v_{N2}\})\}
\]
\noindent
and,  by simplification, 
\[
in(v_g) = \{(v',v_g,r(l))|(v',v,l) \in in((P_1\hide\{v_i\}) \union \{v_{N2}\})\}
\]
\noindent
and, by set manipulation, 
\[
in(v_g) = \{(v',v_g,r(l))|(v',v,l) \in in(P_1\hide\{v_i\}) \union  \{(v',v_{N2},r(l))|(v',v,l)\in in(v), v\in P_2\}\}
\]
\noindent
which, by definition of $in(P_2)$, we can write as
\[
in(v_g) = \{(v',v_g,r(l))|(v',v,l) \in in(P_1\hide\{v_i\}) \union  r(in(P_2))\}
\]
\noindent
where $r(in(P))$ is the relabeling operator $r$ promoted to sets.  Since $r$ is idempotent, we can write this as
\[
in(v_g) = \{(v',v_g,r(l))|(v',v,l) \in in(P_1\hide\{v_i\}) \union  in(P_2)\}
\]
\noindent
and since $\{v_i\} \subseteq P_2$, 
\[
in(v_g) = \{(v',v_g,r(l))|(v',v,l) \in in(P_1) \union  in(P_2)\}
\]
