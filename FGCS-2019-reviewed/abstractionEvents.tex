%%-*- mode: LaTeX; mode: FlySpell; -*-

\section{Abstraction over events}
\label{sec:event}



\jwb{We now evaluate the consistency of the abstractions we propose on the ordering constraints C2-C7. }
%
The ordering constraints C2-C7 in Section~\ref{sec:prov-constraints} define the space of possible temporal interpretations for a $\guEA$ graph. This is the set of all linear orderings of the events that are consistent with the constraints. 
 %
After grouping over $G$, new events in $G'$ are associated with the new grouping node. For a-grouping, these are
 $start(a_{new})$, $end(a_{new})$, as well as 
$ev(\used(e, a_{new}))$, $ev(\wgby(a_{new}, e))$ for all $e$ that are generated by or used by $a_{new}$. 
For e-grouping, the new events are just $ev(\wgby(e_{new}, a))$ and  $ev(\used(e_{new},a))$ for any $a$ that has generated (resp. used) $e_{new}$.

Since $G' \in \guEA$, constraints C2-C7 apply.
%
These define the validity of $G'$.
%
%The space of possible temporal interpretations is the set of linear orderings of these events that are consistent with the constraints, just as is the case for $G$.
%\jwb{sent repeated from above}
%
In this section we explore the relationship between the temporal interpretations in $G$ and those in $G'$. We do this by extending the notion of abstractions over elements of $G$, to that of abstraction over \textit{events} of $G$. 
%
Specifically, in addition to introducing new abstract entities and activities, we are now going to introduce new abstract events, as well. 
%
A set of temporal interpretations for $G'$ can then be expressed using these new abstract events. The legal interpretations amongst these are those that satisfy constraints C2-C7 on $G'$.
%

%
In this section we define such abstract events in terms of the events in $G$, in such a way that for each legal temporal interpretation in $G'$, expressed using the abstract events, there is a corresponding legal temporal interpretation in $G$.
%
We are going to consider e-grouping and a-grouping in turn.

\subsection{Abstract events}

Consider the scenario in Fig.~\ref{fig:e4-e5}(a), where two sections of a document are independently generated by two editing activities, and then they are independently used by four more activities. %\jwb{then invalidation}
%
This scenario is a slight extension of the abstract pattern of Fig.~\ref{fig:e2-a4} (which we have used to illustrate mixed type grouping), where the document sections are $e_4$, $e_5$. 
%
The e-grouping set $V_{gr} = \{ e_4, e_5\}$ represents the whole document. Let $G'$ be the result of (non-strict) e-grouping, as depicted in Fig.~\ref{fig:e4-e5}(b), where the abstract generation and usage events are given new names, namely $g_{Ni}$ as a shorthand for $\wgby(e_N,a_i)$, and $u_{Ni_j}$ for each usage $j$ of the form $\used(a_i, e_N)$. It is easy to see that, taken in isolation, both graphs are valid: one can define the start, end, generation and usage events so that the ordering constraints are satisfied in $G'$.
%

Our goal is to go one step further, and define the new abstract events in terms of the events in $G$, in such a way that for each temporal interpretation that satisfies C2-C7 in $G'$, there is at least one valid interpretation in $G$.
%
We find that such a mapping of the linear orderings is possible,  but that in some circumstances the range of possible orderings for events in $G'$ is restricted.


\begin{figure*}
\centering
\includegraphics[scale=.5]{figures/e4-e5.pdf} 
\caption{Abstraction over a document content, and associated abstracted events} \label{fig:e4-e5}
\end{figure*}

Note: we have been using symbols like $g_{41}$ in the figure to indicate relationships like $wgby(e_4, a_1)$. 
With slight abuse of notation, but in the interest of simplicity, in the following we are going to use $g_{41}$ to also denote $ev(\wgby(e_4, a_1))$ when it is clear from the context that we refer to the event rather than to the relationship itself.
\\

The following questions provide a useful starting point for reasoning about events. Firstly, given the generation events $g_{41}$, $g_{53}$  of each of the document sections, when was the entire document $e_N$ generated? 
%
When did $a_2, a_4$ use the document? 
%
Secondly,  suppose after the first grouping one performs two additional a-groupings, first with $V_{gr} = \{a_1, a_3\}$, and then with $V_{gr} = \{a_2, a_4, a_5, a_6\}$.
%
 This results in the abstraction depicted in Fig.~\ref{fig:e4-e5}(c), which reads simply ``(abstract) document $e_N$ was used by (abstract) activity $a_N$''. 
 In this abstraction, what happens to the original generation and usage events?

Initial help in answering these questions comes from the PROV-DM recommendation document~\citep{w3c-prov-dm}, namely:
%\begin{description}
\begin{itemize}
\item\textbf{Generation} \textit{is the completion of production of a new entity by an activity} (Sec. 5.1.3)
\item\textbf{Usage} \textit{is the beginning of utilizing an entity by an activity} (Sec. 5.1.4)
\item\textbf{Invalidation} \textit{is the start of the destruction, cessation, or expiry of an existing entity by an activity. The entity is no longer available for use (or further invalidation) after invalidation. Any generation or usage of an entity precedes its invalidation.} (Sec. 5.1.8)
\end{itemize}
%\end{description}

%
Let us consider these definitions in the context of our example. Firstly, the generation of the whole document is only complete upon generation of the last section. Thus, each of the generation events of $e_N$, denoted $g_{N1}$ and $g_{N3}$ in Fig.~\ref{fig:e4-e5}(b), cannot precede $g_{41}$, $g_{53}$. This can be written as the ordering constraints:
\begin{align}
 \label{eq:gprime-order1}
max\{g_{41}, g_{53}\} \leq g_{N1} \\
max\{g_{41}, g_{53}\}  \leq g_{N3}
\end{align}
Furthermore, we know from constraint C2 that $g_{N1}$ and $g_{N3}$ must be simultaneous:
$g_{N1} = g_{N3}$.


Secondly, symmetrically to generation, usage of the document ($u_{N2_1}, u_{N2_2}, u_{N4}$) in Fig.~\ref{fig:e4-e5}(b) begins with the earliest usage by any of the consuming activities:
\begin{align}
u_{N2_1} \leq u_{42} \label{eq:gen-usage1}\\
u_{N2_2} \leq u_{52}  \label{eq:gen-usage2}\\
u_{N4} \leq u_{54}  \label{eq:gen-usage3}
\end{align}
%
Finally, from the definition above, the rule for invalidation follows the same pattern as usage:
%
\begin{align}
i_{N6} &\leq i_{46} \label{eq:inv1}\\
i_{N5} &\leq i_{55}  \label{eq:inv2}
\end{align}
%
Furthermore, C3 requires each generation to precede each usage:
%
\begin{align}
g_{N1} &= g_{N13}  \leq  u_{N2_1}  \quad  \\
g_{N1} &= g_{N13}  \leq  u_{N2_2}  \quad  \\
g_{N1} &= g_{N13}  \leq  u_{N4}  \label{eq:gprime-order2}
\end{align}
%
and C4, C5 require both generation and usage to precede invalidation:
%
\begin{align}
g_{N1} &= g_{N3} \leq i_{N6}  \quad g_{N1} = g_{N3} \leq i_{N5} \\
u_{N2_1} &\leq i_{N6}  \quad u_{N2_1} \leq i_{N5} \\
u_{N2_2} &\leq i_{N6}  \quad u_{N2_2} \leq i_{N5} \label{eq:inv-last}
\end{align}
%
In order to avoid excessive clutter in the example, start and generation constraints are only discussed in the next section, along with all general ordering constraints on $G'$.


Now, consider the  linear orderings in $G$ under the assumption that \textit{every generation event precedes every usage event} and \textit{every usage event precedes every invalidation event}, that is, there is a ``generation phase'' followed by a ``usage phase'' and by an ``invalidation phase'' (Fig.~\ref{fig:e-grouping-orderings}(a)). It is easy to see that, with this assumption, all these orderings are consistent with constraints (\ref{eq:gprime-order1}) through (\ref{eq:inv-last}), provided that we redefine $G'$ events to be \textit{simultaneous} to corresponding $G$ events, as follows:

\begin{figure*}
\centering
\includegraphics[scale=.5]{figures/e-grouping-orderings.pdf} 
\caption{Two possible orderings on $G$ (top), and corresponding orderings on $G'$ (bottom)} \label{fig:e-grouping-orderings}
\end{figure*}


\begin{align}
g_{N1} = g_{N3} = max\{g_{41}, g_{53}\}   \label{eq:max} \\
u_{N2_1} = u_{N2_2}  = u_{N4} = min\{u_{42},  u_{52},  u_{54}\}   \label{eq:usage-min-simple} \\
i_{N5} = i_{N6}  = min \{   i_{46},  i_{55} \}   \label{eq:inval-min-simple}
\end{align}
In this case we conclude that $G'$ is valid by our assumption that each of its generation events precedes each of its usage events.

Consider now the more general case where generation, usage and invalidation events are interleaved for different entities in $V_{gr}$. Fig.~\ref{fig:e-grouping-orderings}(b) shows such an interleaving for our example. In this case, 
$min \{   u_{42},  u_{52},  u_{54} \} = u_{42} \leq g_{53} = max\{g_{41}, g_{53}\}$.
%
This violates (\ref{eq:gen-usage1}) through (\ref{eq:gen-usage3}).  In other words, this more general family of interpretations over $G$ is not represented in $G'$ when the abstract events in $G'$ are defined using the inequalities above. 

In order to account for this general case, we modify  (\ref{eq:usage-min-simple}) and (\ref{eq:inval-min-simple}) as follows:
\begin{align}
u_{N2_1} &= u_{N2_2}  = u_{N4} = max\{ g_{41}, g_{53}, min \{u_{42},  u_{52},  u_{54}\}\}  \label{eq:usage-min} \\
i_{N5} &= i_{N6}  = max\{g_{41}, g_{53}, u_{42},  u_{52},  u_{54}, min\{i_{46},  i_{55}\}\}   \label{eq:inval-min}
\end{align}
%

In the example of Fig.~\ref{fig:e-grouping-orderings}(b), this stricter constraint results in  generation and usage events in $G'$ to all be simultaneous to $g_{53}$, while the invalidation events are shifted later in the event line, to the latest usage. Note that in the special case of Fig.~\ref{fig:e-grouping-orderings} (a), constraints (\ref{eq:usage-min-simple}), (\ref{eq:usage-min}) and (\ref{eq:inval-min-simple}), (\ref{eq:inval-min})  are pairwise equivalent.

The reasoning used in the examples just presented justifies the following definitions for the general inequalities which define abstract events in terms of events in $G$.

\subsection{Abstract events for e-grouping}
\label{sec:abstract-events-for-e-grouping}
%
Let $G=(V,E) \in \guEA$, $V_{gr} \subset V$ be the set of nodes that are to be grouped, and  $e_{new} \in \en$ be the new entity node introduced through e-grouping as per Def.~\ref{eq:t-grouping}.
% 

\paragraph*{\textbf{Abstract Generation events}}
Let $V^*$ to denote $\extend(\clos(V_{gr},G), \en)$, and let $\wgby_{out}$ denote the set of generation relations involving entity nodes in the extension
$V^*$, and activity nodes outside of the extension:
\begin{align*} 
\wgby_{out} = \{ \wgby(e,a) |  e \in V^*,  a \notin V^* \}
\end{align*} 
In the example of Fig.~\ref{fig:e4-e5}, $\wgby_{out} = \{ g_{41}, g_{53} \}$.

%
Correspondingly, let $\wgby'_{out}$ denote the generation relations that involve $e_{new}$:
\begin{align*} 
\wgby'_{out} = \{ \wgby(e_{new},a) | a \notin V^* \}
\end{align*} 
In the example, $\wgby'_{out} = \{ g_{N1}, g_{N3} \}$.

In general, we will denote values in the abstracted prov graph by primed versions of their counterparts in the original graph. The exception to this will be relationships and events involving $e_{new}$, since $e_{new}$ is  a new entity that does not appear in the old graph.

The following equalities, define the orderings of the events associated with the relations in $\wgby'_{out}$.

\vspace*{10pt}
\begin{definition}[Abstract generation events - e-grouping]
\label{def:abstract-gen-e}
For each $g' \in \wgby'_{out}$:
\[
ev(g') = max \{ ev(g) | g \in \wgby_{out} \}  
\]
For all activities $a$ that participate in generating the new event $e_{new}$, we set the generation event to
\[
ev(\wgby(e_{new},a)) = max \{ ev(g) | g \in \wgby_{out} \}
\]
\end{definition}

%\vspace*{10pt}
%\begin{definition}[Abstract usage events - e-grouping] 
%\label{def:abstract-usage-e}
%Let
%\[u'_{min} = min_{ u \in \used_{in} } ( ev(u) \}\] and let 
%$g'_{max} = max_{g \in \wgby_{out}} ( ev(g) \} $.\\
%For each $u' \in \used'_{in}$:
%\begin{equation}
%v(u') = max \{ g'_{max} , u'_{min} \}
%\end{equation}
%\end{definition}

\paragraph*{\textbf{Abstract Usage events}}
Usage events for e-grouping are defined similarly by generalization from (\ref{eq:usage-min}), as follows.
%
Let $\used_{in}$ denote the set of usage relations involving entity nodes in the extension
$V^*$, and activity nodes outside of the extension:
\begin{align*} 
\used_{in} = \{ \used(a,e) |  e \in V^*,  a \notin V^* \}
\end{align*} 
In the example of Fig.~\ref{fig:e4-e5}, $\used_{in} = \{ u_{42}, u_{52}, u_{54} \}$.

%
Correspondingly, let 
$\used'_{in}$ denote the usage relations that involve $e_{new}$:
\begin{align*} 
\used'_{in} = \{ \used(a, e_{new}) | a \notin V^* \}
\end{align*} 
In the example, $\used'_{in} = \{ u_{N2_1}, u_{N2_2}, u_{N4}  \}$.

The following equalities, which generalise (15), define the events associated with the relations in $\used'_{in}$.

\vspace*{10pt}
\begin{definition}[Abstract usage events - e-grouping] 
\label{def:abstract-usage-e}
Let
\[u'_{min} = min \{ ev(u) | u \in \used_{in} \}\] and let 
$g'_{max} = max \{ ev(g) | g \in \wgby_{out}\} $.\\
For each $u' \in \used'_{in}$:
\begin{equation*}
ev(u') = max \{ g'_{max} , u'_{min} \}
\end{equation*}
and so
\begin{equation}
ev(\used(a,e_{new})) = max \{ g'_{max} , u'_{min} \}
\end{equation}
\end{definition}

\paragraph*{\textbf{Abstract Invalidation events}}
The events equalities for invalidation, exemplified in (\ref{eq:inval-min}), follow the pattern used above for usage. The only difference is that $\used'_{in}$ is replaced by 
\[ \inv'_{in} = \{ \inv(a, e_{new}) | a \notin V^* \} \]
In our example, $ \inv_{in} = (  i_{46}, i_{55} \}$,  $\inv'_{in} = ( i_{N6}, i_{N5} \}$.
%
The corresponding definition is as follows.

\vspace*{10pt}
\begin{definition}[Abstract invalidation events] 
\label{def:abstract-inv}
Let
\[i'_{min} = min \{ ev(i) | i \in \inv_{in} \}\]
and 
\[u'_{max} = max \{ ev(u) | u \in \used_{in} \}\]
For each $i' \in \inv'_{in}$:
\[
ev(i') = max \{ g'_{max} , u'_{max},  i'_{min}\}
\]
and thus for each activity $a$  that participates in the invalidation of $e_{new}$, invalidation is the latest of the new generation event $g'_{max}$, the latest usage event $u'_{max}$ and the minimum of the original invalidation events.  
\begin{equation}
\inv(a,e_{new}) = max \{ g'_{max} , u'_{max},  i'_{min} \}
\end{equation}
\end{definition}


{\bf Start events:} Now that the event of usage and generation of $e_{new}$ has been fixed, we need to ensure that the activities involved in these events continue to meet the constraints that apply to them. This includes start and end events for activities related to our new entity $e_{new}$ by either $\wgby$ or $\used$.  


%In addition, a-grouping also produces new abstract start and end events for the new activity $a_{new}$. 
% The situation is illustrated in the timelines of Fig.~\ref{fig:a1-a2}. The right side of figure (a) shows a possible interleaving of events in $G$, which is consistent with constraints C2-C7. Intuitively, the start (resp. end) event for the abstract activity $a_N$ cannot follow (resp. precede) the earliest (resp., latest) usage/generation event associated with $a_1, a_2$.
% %
% Similar to the case for e-grouping,
We appeal to the informal definitions of start and end in~\citep{w3c-prov-dm} to derive inequalities for the abstract start and end events for our new activity $a_{new}$.
% 
\begin{itemize}
\item \textbf{Start} \textit{is when an activity is deemed to have been started by an entity, known as trigger. The activity did not exist before its start. Any usage, generation, or invalidation involving an activity follows the activity's start} (See~\citep{w3c-prov-dm},  Section 5.1.6)

\item \textbf{End} \textit{is when an activity is deemed to have been ended by an entity, known as trigger. The activity no longer exists after its end. Any usage, generation, or invalidation involving an activity precedes the activity's end} (See~\citep{w3c-prov-dm}, Section 5.1.7)
\end{itemize}
(For simplicity we are going to leave the trigger entity implicit, and simply refer to the start and end events as $\start(a)$ and $\ed(a)$).


\begin{definition}[Start events - e-grouping] 
\label{def:abstract-start-e}
Consider events $a$ such that $\wgby(e_{new},a)$. For each such $a$, we set the new start event $\start'(a)$ as the lesser of the original start event and the generation event $ev(\wgby(e_{new},a))$. Thus
\begin{equation}
\start'(a) = min\{\start(a),ev(\wgby(e_{new},a))\}
\end{equation}
\end{definition}

{\bf End events:} For all activities $a$ that use the newly created entity, we must ensure that ensure that the end of the activity does not precede any $\used(a,e_{new})$ events.
\begin{definition}[End events - e-grouping]
  \label{def:abstract-end-e}
  If the set of all usage events by $a$ of $e_{new}$ is denoted $\{ev(\used(a,e_{new}))\}$, we set the new end events $\ed'(a)$ to be
  \begin{equation}
  \ed'(a) = max\{\ed(a), max\{ev(\used(a,e_{new}))\}\}
\end{equation}
\end{definition}

A proof of that, given the definitions above, the  constraints of Section~\ref{sec:prov-constraints} are satisfiable, is given in~\ref{sec:consistency-constraints-e-grouping}.


\subsection{Abstract events for a-grouping}
\label{sec:abstract-events-for-a-grouping}
% \begin{figure}
% \centering
% \includegraphics[scale=.5]{figures/a1-a2.pdf} 
% \caption{Abstraction for start/end events} \label{fig:a1-a2}
% \end{figure}

Generation and usage abstract events follow a very similar pattern as those for e-grouping, except that the new node introduced by grouping is an activity node: $a_{new} \in \act$.
%
%This is illustrated in Fig.~\ref{fig:a1-a2}.
As a consequence, the abstract event definitions given in the previous section also follow the same pattern, but with the roles of entities and activities reversed. They are summarized here below.  We now use $V^*$ to mean the group of nodes collected by $\extend(\clos(V_{gr},G), \act)$.

%\jwb{I (personally) find the notations below confusing, and would rather get rid of them. I haven't used them much in the rest of this section.}
%
%\begin{align*} 
%\wgby_{in} & =  \{ \wgby(e,a) |  e \notin V^*, a \in V^* \} \\
%\wgby'_{in} &  = \{ \wgby(e, a_{new}) | e \notin V^* \} \\
%\used_{out} & = \{ \used(a,e) |  e \notin V^*,  a \in V^* \} \\
%\used'_{out} & = \{ \used(a_{new}, e) | e \notin V^* \}
%\end{align*} 
%
%
%

The new start (resp. end) event is taken to be the minimum (resp. maximum) relevant start (resp. end)  event.
\begin{definition}[Abstract start and end events - a-grouping] 
\label{def:abstract-start-and-end-a}
\begin{align*}
  \start(a_{new}) & = min\{\start(a) | a \in V^*\} \\
  \ed(a_{new}) & = max\{\ed(a) | a \in V^*\} \\
\end{align*}
\end{definition}

\begin{definition}[Abstract generation events - a-grouping]
\label{def:abstract-gen-a}
%Since generation events must be simultaneous, we can assume that they are simultaneous in the original graph. 
%
%For any entity $e$ not in $V^*$ that is generated by an activity $a$ in $V^*$, the new generation event cannot be before the start of the abstracted activity $a_{new}$, taken as the minimum of the original start events s in Definition~\ref{def:abstract-start-and-end-a}. However,
Constraint C2 applies to the original graph, so for all activities $a,b$ that participate in the generation of $e$, $ev(\wgby(e,a)) = ev(\wgby(e,b))$. 


%\jwb{Should we do start and end definitions first? Start and end only apply to ctivities, so the definition below already excludes entities}
The new generation event is thus given as
%  \begin{align*}
%   ev(\wgby(e,a_{new})) =  max\{ &  \start(a_{new}), \\ 
%                                & min\{ev(\wgby(e,a))  | a \in V^*\} \}
% \end{align*}
  \begin{align*}
   ev(\wgby(e,a_{new})) =  ev(\wgby(e,a))
 \end{align*}

\end{definition}

\vspace*{10pt}
\begin{definition}[Abstract usage events - a-grouping] 
\label{def:abstract-usage-a}
For an entity $e$ not in $V^*$ which is used by an activity $a$ in $V^*$, the assigned usage event for $a_{new}$ ($ev(\used(a_{new},e))$) is given by  
\begin{align*}
ev(\used(a_{new},e)) = max\{ & \start(a_{new}) , \\
                            & min\{ev(\used(a,e))) | a \in V^*\}\}
\end{align*}
\end{definition}


\vspace*{10pt}
\begin{definition}[Abstract invalidation events - a-grouping] 
  \label{def:abstract-inv-a}
  By Constraint C8, invalidation events from two distinct activities are simultaneous.
 %
For an entity $e$ which is invalidated by an activity $a$ in $V^*$, the event of the invalidation event remains the same when the activity is abstracted.
\[
ev(\inv(a_{new},e)) = min\{ev(\inv(a,e)) | a \in V^*\}
\]
\end{definition}

A proof of that, given the definitions above, the  constraints of Section~\ref{sec:prov-constraints} are satisfiable, is given in~\ref{sec:consistency-constraints-e-grouping}.


